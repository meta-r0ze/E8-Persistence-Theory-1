\section{Conclusion: The Invariant Substrate}

In this work, we have established that the fundamental constants of nature are not arbitrary tuning parameters, but the necessary boundary conditions of a discrete $E_8$ gauge theory projected onto a 4D manifold.

We have demonstrated a \textbf{Derivation Hierarchy} where a single master equation ($\alpha^{-1}$) encodes the complete geometric specification of the vacuum. As summarized in the System Specification (Section III), all fundamental limits derive from the lattice invariants:

\begin{equation}
\mathbb{S} = \{ D=4, \Delta=43, \sigma=5, \nu=16, \chi=2 \}
\end{equation}

With the static geometry of the vacuum established (System I, System II, System III), the subsequent papers in this series will derive the resonant excitations of this substrate. Table \ref{tab:system4} outlines this Architecture of Matter, demonstrating how the constants derived here serve as the construction rules for the stable particle spectrum.

\begin{table*}[h]
\centering
\caption{\textbf{System IV: Architecture of Matter.} The emergence of stable particles, atoms, and nuclei as resonant solutions of the Effective Field Limits.}
\label{tab:system4}
\renewcommand{\arraystretch}{1.5}
\setlength{\tabcolsep}{6pt}
\begin{tabular}{@{} l l l r l @{}} 
\toprule
\textbf{IE Pillar} & \textbf{Component} & \textbf{Construction Rule} & \textbf{Archetype} & \textbf{System Function} \\
\midrule
\textbf{Substrate ($S$)} & Invariant Substrate & System I ($\mathbb{S}$) & \textbf{$E_8$} & Metric constraints of the 4D projection. \\
\textbf{Substrate ($S$)} & Input Impedance & System II ($\mathbb{O}$) & \textbf{$\alpha^{-1}$} & The Baseline Cost \\
\textbf{Substrate ($S$)} & Field Limits & System III ($\mathbb{C}$) & \textbf{Standard Model} & \textbf{The Runtime Rules} \\
\midrule
\textbf{Energy Vessel ($\Delta E$)} & Lattice Phonon & Inverse Resonance ($1/\Delta^n$) & \textbf{Neutrino} & Energy Sink (Thermodynamic Balance) \\
\addlinespace
\textbf{Info. Model ($\Delta I$)} & Resonant Knot & Geometric Lock ($\Delta^2 - \pi D$) & \textbf{Proton} & Baryonic Identity (Stable Memory) \\
\textbf{Info. Model ($\Delta I$)} & Physical Radius & Projection Scale ($D \cdot \lambda_C$) & \textbf{Charge Radius} & Spatial Extent (Interaction Volume) \\
\addlinespace
\textbf{Protocol ($MI$)} & Ground State & Zero-Entropy Address ($\Delta^0$) & \textbf{Electron} & Charge Carrier (Chemical Agent) \\
\textbf{Protocol ($MI$)} & Atomic Orbital & Impedance Matching ($\alpha^2 m_e$) & \textbf{Hydrogen} & Bonding Interface (Rydberg). \\
\addlinespace
\multicolumn{5}{l}{\textit{The Stabilizing Governor ($G$) — Nuclear Limits (Magic Numbers)}} \\
\textbf{Governor ($G$)} & Magic Number 2 & Boundary ($\chi$) & \textbf{Helium (2)} & Minimal Topological Closure. \\
\textbf{Governor ($G$)} & Magic Number 8 & Manifold Double ($2D$) & \textbf{Oxygen (8)} & Spinor Capacity. \\
\textbf{Governor ($G$)} & Magic Number 20 & Projection ($D \cdot \sigma$) & \textbf{Calcium (20)} & Symmetric Packing. \\
\textbf{Governor ($G$)} & Magic Number 28 & Capacity ($H_{sys} + \sigma$) & \textbf{Nickel (28)} & System Saturation. \\
\addlinespace
\textbf{Governor ($G$)} & Magic Number 50 & Harmonic ($\Delta + \sigma + \chi$) & \textbf{Tin (50)} & Resonant Stability. \\
\textbf{Governor ($G$)} & Magic Number 82 & Harmonic ($2\Delta - D$) & \textbf{Lead (82)} & Heavy Saturation. \\
\textbf{Governor ($G$)} & Magic Number 126 & Harmonic ($3\Delta - (\sigma - \chi)$) & \textbf{Shell (126)} & Interaction Limit. \\
\addlinespace
\multicolumn{5}{l}{\textit{The Thermodynamic Taxes (Manifestation in Matter)}} \\
\textbf{Temporal Tax ($T$)} & Fine Structure & Spin-Orbit Coupling ($\alpha^4$) & \textbf{Splitting} & Entropic cost of orbital movement \\
\textbf{Margin ($PM$)} & Mass Defect & Binding Ratio ($\chi\Delta / D\sigma$) & \textbf{Deuteron} & Energy released to purchase stability \\
\bottomrule
\end{tabular}
\end{table*}

The universe does not choose these structures; the lattice architecture selects them as the only solvent methods of information propagation. By replacing free parameters with geometric necessities, we move from a descriptive model of physics to a predictive one.