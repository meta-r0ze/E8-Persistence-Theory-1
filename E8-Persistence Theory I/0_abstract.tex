\begin{abstract}
The Standard Model of particle physics relies on over 20 free parameters that must be determined by measurement rather than first principles. In this work, we test the hypothesis that these constants are the necessary boundary conditions of a resource-constrained information processing system. We model the vacuum as a discrete $E_8$ lattice projected onto a 4-dimensional manifold and apply the \textbf{Persistence Principle}, the minimization of Entropic Action within a finite-capacity system to identify five integer invariants $\mathbb{S} = \{ \Delta=43, \nu=16, \sigma=5, D=4, \chi=2 \}$ that define the geometric kernel of the vacuum.

From these integers, we derive the Fine-Structure Constant ($\alpha^{-1}$) as the calculated \textbf{Geometric Impedance} of the substrate. The derivation yields $\alpha^{-1}_{geo} \approx 137.035999212$, which lies within the uncertainty interval of current experimental measurements. We further demonstrate that the running of the coupling to the electroweak scale ($\alpha^{-1}(M_Z) \approx 128.9$) emerges naturally from the geometric screening of the lattice boundary.

Finally, we show that the Strong Coupling ($\alpha_s$), Weak Mixing Angle ($\sin^2 \theta_W$), Gravitational parameters ($\alpha_G, M_P$), and the complete Higgs sector—including the VEV ($v$), scalar mass ($m_H$), self-coupling ($\lambda$), and electron Yukawa ($y_e$) are strictly derived harmonics of this geometric impedance. This framework provides the \textbf{Geometric Initialization} of the Standard Model, replacing arbitrary parameter tuning with a self-consistent derivation hierarchy.
\end{abstract}
