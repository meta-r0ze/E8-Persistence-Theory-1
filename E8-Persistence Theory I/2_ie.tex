\section{Theoretical Context: Informational Energetics}\label{:IE}

This framework synthesizes insights from non-equilibrium thermodynamics, algorithmic information theory, and robust control theory, bridging them with empirical principles from evolutionary biology, computational neuroscience, and high-energy physics, and applying them through the practical lenses of institutional economics, quantitative finance, and reliability engineering.

We proceed from the hypothesis that physical reality acts as a resource-constrained information processing system. A subset of complex adaptive systems, persistent systems must optimize for persistence against entropy. We posit that the vacuum itself is also governed by a \textbf{Persistence Principle}: the minimization of computational cost (Entropic Action) relative to structural complexity.

From this perspective, the laws of physics are not arbitrary rules but stability conditions, the geometric forms that survive the vacuum's entropic selection.

The fundamental constants are not inputs; they are the survivors of a geometric selection process, representing the unique topological forms capable of persisting against the entropic flux of the vacuum.

\subsection{The Axiom of Persistence}
The fundamental imperative of any persistent system is to maximize its existence duration within a high-entropy environment. This requires minimizing the \textbf{Entropic Action ($S_\Phi$)}—the metabolic cost of maintaining a distinct identity. To satisfy this axiom, any persistent entity must possess four structural pillars to manage the flow of information and energy, plus accounting for the thermodynamic overhead of operation.

\subsection{The Structural Pillars}
We define \textbf{Persistence} ($P$) as the sum of the metabolic costs required to maintain these pillars.

\begin{equation}
\label{eq:IE_impedance}
\begin{split}
P = {} & \\ 
  \underbrace{\Delta E}_{\text{Capacity}}
+ \underbrace{\Delta I}_{\text{Identity}}
- \underbrace{MI}_{\text{Efficiency}}
- \underbrace{G}_{\text{Stability}}
+ \underbrace{T}_{\text{Overhead}}
+ \underbrace{PM}_{\text{Margin}}
\end{split}
\end{equation}

\noindent The six components represent the universal structural requirements of persistence. To manifest physically, these abstract requirements must map to specific features of a system. These six components represent the minimal complete set: fewer leaves the system unable to persist; additional components reduce to combinations of these. Positive terms represent metabolic costs; negative terms represent efficiency gains that reduce the persistence burden.

When projecting a $E_8$ lattice onto a 4D Manifold, which is derived as the unique substrate solution in \cref{sec:DerivationOfTheSubstrate} these universal components crystallize into the following specific geometric functions:

\begin{enumerate}
    \item \textbf{The Energy Vessel ($\Delta E$):} \textit{The Capacity.} The structure that holds state. In System I, this maps to \textbf{The Fundamental Resonance} ($\Delta=43$).
    \textbf{Mechanism:} In a wave-based substrate, storage capacity is strictly limited by the fundamental non-repeating frequency (Heegner Number) required to prevent aliasing.

    \item \textbf{The Information Model ($\Delta I$):} \textit{The Identity.} The structure used to interact with the environment. In System I, this maps to \textbf{The Interaction Symmetry} ($\sigma=5$).
    \textbf{Mechanism:} The complexity of a particle's identity is bounded by the rank of its internal symmetry group ($SU(5)$ precursor).
    
    \item \textbf{The Coordination Protocol ($MI$):} \textit{The Efficiency.} The channel regulating flow. In System I, this maps to \textbf{The Chiral Channel} ($\nu=16$).
    \textbf{Mechanism:} Efficiency requires directed information flow; geometrically, this requires truncating the total degrees of freedom to the chiral rank (Geometric Diode).
    
    \item \textbf{The Stabilizing Governor ($G$):} \textit{The Stability.} The constraint preventing divergence. In System I, this maps to \textbf{The Topological Boundary} ($\chi=2$).
    \textbf{Mechanism:} Infinite dissipation is prevented by enforcing topological closure on the energy vessel (Gauss-Bonnet limit).
    
    \item \textbf{The Temporal Tax ($T$):} \textit{The Overhead.} The cost of updates. In System I, this maps to \textbf{Metric Time} ($-1$).
    \textbf{Mechanism:} State transitions require an irreversible metric signature component to enforce the arrow of time.
    
    \item \textbf{The Persistence Margin ($PM$):} \textit{The Floor.} The buffer for existence. In System I, this maps to \textbf{Metric Space} ($+3$).
    \textbf{Mechanism:} Volumetric capacity is required to store the knots defined by the other pillars.
\end{enumerate}

\noindent \textbf{The Finite Capacity Constraint:} Standard Quantum Field Theory assumes a vacuum of infinite capacity, leading to divergences. By mapping the \textbf{Stabilizing Governor} to the topological boundary ($\chi=2$), we impose a physical mechanism that prevents the Energy Vessel from diverging. The Persistence Principle acts as the selection filter, ensuring only geometric configurations with this functioning Governor survive.

With the universal architecture of persistence established, we now derive the specific geometric invariants of the vacuum substrate.

% TODO citations to add
% Non-equilibrium thermodynamics → cite Prigogine or England
% Algorithmic information theory → cite Kolmogorov, Chaitin, or Solomonoff
% Robust control theory → cite Doyle or Csete & Doyle (2002) on biological robustness
% Evolutionary biology → cite Kauffman (self-organization) or England (dissipation-driven adaptation)