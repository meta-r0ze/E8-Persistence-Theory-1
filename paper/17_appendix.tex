\appendix

\section{Formal Derivation of the Causality Constraint} \label{sec:DerivationOfTheCausalityConstraint}
In the main text, we asserted that a causal projection requires the resonant frequency $\Delta$ to be greater than the total state space $N=32$. This appendix provides the rigorous mathematical justification for that constraint. We first formally define the projection operators that separate the matter (chiral) and mirror (symmetric) sectors, prove their geometric independence, and then derive the non-aliasing condition that emerges from this structure.

\subsection{The Projection Operators}

The E\textsubscript{8} lattice embeds in $\mathbb{R}^8$. We define the projection from the 8D lattice space to the 4D spacetime manifold using the orthogonal decomposition inherent to the $E_8 \to D_4 \oplus D_4$ split. Let $x \in \mathbb{R}^8$ be a lattice vector. We define the Left-Chiral ($P_L$) and Right-Chiral ($P_R$) projection operators as:
\[
P_L(x) = \frac{1}{\sqrt{2}}(x_1 - x_2, x_3 - x_4, x_5 - x_6, x_7 - x_8)
\]
\[
P_R(x) = \frac{1}{\sqrt{2}}(x_1 + x_2, x_3 + x_4, x_5 + x_6, x_7 + x_8)
\]

\subsection{Orthogonality Proof}

For the projections to define distinct physical sectors (Matter vs. Mirror), they must be orthogonal. We compute the inner product. Since the basis vectors are orthogonal roots in the fundamental domain of E\textsubscript{8} ($|x_i|^2=1, x_i \cdot x_{i+1}=0$):
\begin{align*}
    P_L \cdot P_R &= \frac{1}{2} \sum_{i=1,3,5,7} (x_i - x_{i+1})(x_i + x_{i+1}) \\
    &= \frac{1}{2} \sum (x_i^2 - x_{i+1}^2) = \frac{1}{2} \sum (1 - 1) = 0
\end{align*}
Thus, $P_L \perp P_R$. The sectors are geometrically distinct.

\subsection{The Generalized Nyquist Limit}

The total information content of the lattice node is the sum of its chiral components.
\begin{itemize}
    \item Dimension of Left Sector: $\text{dim}(P_L) = 4$ spatial dimensions $\times$ 4 spinor components $= 16$ degrees of freedom ($\nu$). The image of $P_L$ is 4-dimensional. However, the chiral spinor representation on this 4D manifold has dimension $v = 16$ (the Weyl spinor of Spin(10)).
    \item Dimension of Right Sector: $\text{dim}(P_R) = 16$ degrees of freedom.
    \item \textbf{Total State Space: $N = 32$.}
\end{itemize}

For the lattice to project these $N=32$ degrees of freedom onto a single integer timeline (defined by the resonant frequency $\Delta$) without \textbf{Aliasing}, we invoke the Pigeonhole Principle.

\paragraph{Theorem (Generalized Nyquist for Discrete Lattices):} Let a lattice have $N$ distinct state channels, and let $\Delta$ be the fundamental period. For a bijective projection from the $N$-channel state space onto the cyclic group $\mathbb{Z}/\Delta\mathbb{Z}$, we require $\Delta > N$. If $\Delta \le N$, then by the Pigeonhole Principle, at least two distinct channels must map to the same integer index in $\mathbb{Z}/\Delta\mathbb{Z}$. This creates an aliasing event, causing phase ambiguity between the Left and Right sectors.

Thus, the condition $\Delta > 32$ is a hard topological constraint of the projection.


\section{Geometric Origin of Hypercharge}\label{sec:OriginOfHypercharge}

In the main text, we derived the structure of the non-Abelian gauge groups $SU(3)_C$ and $SU(2)_L$ from the integer invariants $\sigma$ and $\chi$. This appendix extends that logic to the Abelian group $U(1)_Y$, demonstrating that the specific hypercharge quantum numbers of the Standard Model are not arbitrary assignments but are the precise values required to maintain consistency between a particle's electric charge and its geometric role within the lattice.

We use the Gell-Mann--Nishijima relation, $Q = I_3 + Y/2$, as a constraint equation. By deriving Weak Isospin ($I_3$) geometrically from the topological boundary $\chi=2$, we can calculate the necessary hypercharge $Y = 2(Q - I_3)$ for each particle and show that it corresponds to a simple ratio of the geometric invariants $\{D, \sigma, \chi\}$.

\paragraph{Geometric Isospin ($I_3$):} The $SU(2)_L$ symmetry arises from the topological boundary $\chi=2$, which mandates a doublet structure for left-handed particles. We therefore make the following geometric assignments:
\begin{itemize}
    \item Particles in a left-handed doublet are assigned $I_3 = \pm 1/2$.
    \item Particles that are right-handed singlets are assigned $I_3 = 0$.
\end{itemize}

\paragraph{Derivation of Fermion Hypercharges:}
The following table demonstrates that the calculated hypercharge for every Standard Model fermion perfectly matches a unique, simple ratio of the geometric invariants. This provides a powerful confirmation of the framework, linking quantum numbers directly to the geometry of the substrate.

\begin{table}[h!]
\centering
\caption{Derivation of Fermion Hypercharges from Geometric Ratios}
\label{tab:hypercharge_derivation}
\renewcommand{\arraystretch}{1.2}
\begin{tabular}{@{}lccccc@{}}
\toprule
\textbf{Particle} & \textbf{$SU(2)_L$ Rep.} & \textbf{$I_3$} & \textbf{$Q$} & \textbf{Calculated $Y$} & \textbf{Geometric Ratio} \\
\midrule
Neutrino ($L_L$)    & Doublet & $+1/2$ & $0$    & $-1$ & $-D/D$ \\
Electron ($L_L$)    & Doublet & $-1/2$ & $-1$   & $-1$ & $-D/D$ \\
Electron ($e_R$)    & Singlet & $0$    & $-1$   & $-2$ & $-\chi$ \\
\addlinespace
Up Quark ($Q_L$)    & Doublet & $+1/2$ & $+2/3$ & $+1/3$ & $1/(\sigma-\chi)$ \\
Down Quark ($Q_L$)  & Doublet & $-1/2$ & $-1/3$ & $+1/3$ & $1/(\sigma-\chi)$ \\
Up Quark ($u_R$)    & Singlet & $0$    & $+2/3$ & $+4/3$ & $D/(\sigma-\chi)$ \\
Down Quark ($d_R$)  & Singlet & $0$    & $-1/3$ & $-2/3$ & $-\chi/(\sigma-\chi)$ \\
\bottomrule
\end{tabular}
\end{table}

\paragraph{Derivation of Higgs Hypercharge:}
The Higgs boson doublet contains a neutral component ($H^0$) and a charged component ($H^+$). For the charged component, $Q=+1$ and $I_3=+1/2$. Applying the same convention as the fermion sector yields its hypercharge:
\[
Y_H = 2(Q - I_3) = 2(1 - 1/2) = +1
\]
Geometrically, this unitary value corresponds to the scalar ground state of the lattice:
\[
Y_H = \frac{1}{\Delta^0} = 1
\]

\paragraph{Physical Interpretation of the Ratios:}
The specific ratios assigned to each particle class are not arbitrary; they reflect the particle's fundamental coupling to the geometric structures of the vacuum.
\begin{itemize}
    \item \textbf{Leptons:} As color-singlets, their hypercharges are determined by their coupling to the fundamental manifold topology ($D$) and its boundary ($\chi$), not the color sector ($\sigma-\chi$).
    \item \textbf{Quarks:} As color-triplets, their hypercharges are necessarily normalized by the Interaction Remainder ($\sigma-\chi=3$), reflecting their fundamental coupling to the $SU(3)_C$ gauge structure.
    \item \textbf{Higgs:} As the scalar field that stabilizes the electroweak boundary, its hypercharge is derived from the unitary scalar ground state ($\Delta^0=1$), signifying its foundational role.
\end{itemize}

\textbf{Conclusion:} The hypercharge assignments are not random. They are the unique rational numbers that ensure the electric charge of a particle is consistent with its geometric isospin (determined by $\chi$), its relationship to the color sector (via ratios involving the remainder $\sigma-\chi=3$), and its role within the manifold ($D=4$). The entire charge structure of the Standard Model is shown to be a direct consequence of the five geometric invariants.






\section{The \boldmath{$E_8$} Root Inventory and State Partition} \label{sec:RootInventoryAndStatePartition}

To demonstrate that the Standard Model particle content is not an arbitrary selection, we provide a rigorous accounting of the 248 root vectors of $E_8$. We show that the 48 fundamental fermions are the unique subset of the root system that satisfies the Persistence Principle under the maximal subgroup decomposition $E_8 \supset E_6 \times SU(3)$.

\subsection{The Fundamental Decomposition}

The $E_8$ Lie algebra (dimension 248) decomposes under $E_6 \times SU(3)$ as:

\begin{equation}
\mathbf{248} = (\mathbf{78}, \mathbf{1}) \oplus (\mathbf{1}, \mathbf{8}) \oplus (\mathbf{27}, \mathbf{3}) \oplus (\overline{\mathbf{27}}, \overline{\mathbf{3}})
\end{equation}

This partition assigns every root vector in the lattice to a specific physical sector based on its transformation properties:

\begin{itemize}
    \item \textbf{The Gauge Core $(\mathbf{78}, \mathbf{1})$:} Contains the unification field generators, including the Standard Model gauge group $SU(3) \times SU(2) \times U(1)$.
    \item \textbf{The Flavor Core $(\mathbf{1}, \mathbf{8})$:} Contains the generators of the generational symmetry (identifying the 3-fold generation index).
    \item \textbf{Matter Precursors $(\mathbf{27}, \mathbf{3})$:} Contains three families of matter, transforming as the fundamental representation of $E_6$.
    \item \textbf{Mirror Precursors $(\overline{\mathbf{27}}, \overline{\mathbf{3}})$:} Contains three families of conjugate mirror matter.
\end{itemize}

\subsection{The Persistence Filter (Decomposition of the $\mathbf{27}$)}

The 81 roots identified as ``Matter'' consist of three copies of the $\mathbf{27}$. Under the geometric descent to the Standard Model via $SO(10)$, the $\mathbf{27}$ decomposes as:

\begin{equation}
\mathbf{27} \to \mathbf{16} \oplus \mathbf{10} \oplus \mathbf{1}
\end{equation}

The Persistence Principle acts as a geometric filter on these sub-representations based on the topological boundary condition $\chi=2$ established in Section IV.D:

\begin{enumerate}
    \item \textbf{The $\mathbf{16}$ (Chiral Spinor):} Contains the Standard Model fermions ($Q_L, u_R, d_R, L_L, e_R, \nu_R$). These states possess the required topological closure ($\chi=2$) to maintain a persistent knot structure. \textbf{Status: Retained.}
    
    \item \textbf{The $\mathbf{10}$ (Vector):} Contains vector-like quarks and leptons. These states possess $\chi=0$ (vector pairing), preventing topological closure. Without a boundary condition, they cannot maintain a finite mass scale and decouple at the Planck limit ($\lambda \to \infty$). \textbf{Status: Filtered.}
    
    \item \textbf{The $\mathbf{1}$ (Singlet):} Contains the sterile neutrino modulus. Lacking gauge tethering to the manifold ($D=4$), this state cannot maintain a coherent address in the causal update sequence. \textbf{Status: Filtered.}
\end{enumerate}

\subsection{The Balance Sheet (248 $\to$ 48)}

We provide the complete ledger of the $E_8$ roots, classifying them into \textbf{Persistent (SM)} and \textbf{Filtered (Decoupled)} categories.

\paragraph{The Excluded Sectors (200 Roots)}
The following roots are geometrically prohibited from forming low-energy states:
\begin{itemize}
    \item \textbf{Mirror Sector (81 Roots):} The $\overline{\mathbf{27}}$ sector is orthogonal to the chiral projection $P_L$ defined in Eq.~(4). It constitutes the ``Dark Sector'' which may interact gravitationally but lacks the geometric alignment to couple to the Standard Model gauge fields.
    \item \textbf{Vector/Sterile Sector (33 Roots):} The $\mathbf{10}$ and $\mathbf{1}$ components of the $\mathbf{27}$ (11 roots per generation $\times$ 3 generations).
    \item \textbf{Heavy Gauge Sector (66 Roots):} The components of the $\mathbf{78}$ excluding the 12 Standard Model bosons. These correspond to GUT-scale X/Y bosons which acquire Planck-scale masses.
    \item \textbf{Flavor Generators (8 Roots):} Internal symmetry operators that do not manifest as propagating fields.
    \item \textbf{Total Excluded:} $81 + 33 + 66 + 8 = 188$ roots.
\end{itemize}

\paragraph{The Persistent Sector (60 Roots)}
The remaining roots satisfy all geometric invariants ($\nu=16, \chi=2, D=4$):
\begin{itemize}
    \item \textbf{Standard Model Bosons (12 Roots):} The photon, $W^\pm$, $Z$, and 8 gluons.
    \item \textbf{Standard Model Fermions (48 Roots):}
    \begin{itemize}
        \item 3 Generations $\times$ 16 Chiral States.
        \item ($\nu = 16$ states = 6 Quarks + 2 Leptons + Antiparticles).
    \end{itemize}
\end{itemize}

\textbf{Conclusion:} The 48 fermions of the Standard Model are not an arbitrary collection; they are the unique subset of $E_8$ roots that possess both Chiral Persistence ($P_L$) and Topological Stability ($\chi=2$). All other roots are strictly filtered by the geometry of the projection.