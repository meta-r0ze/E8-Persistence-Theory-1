\subsection{System IV-B: The Topological Structure of the Strong Force (The \texorpdfstring{$\theta_{QCD}$}{theta_QCD} Prohibition)}
\label{sec:StrongCP}

Having established the magnitude and dynamic scaling of the Strong Force, we now address its fundamental topological structure. The same geometric invariants that define the SU(3) color group also structurally prohibit the CP-violating $\theta$-term in the QCD Lagrangian, providing a geometric resolution to the Strong CP problem.

\paragraph{The Standard Model Ansatz}
The QCD Lagrangian contains a CP-violating topological term, $g^2\theta_{QCD}/(32\pi^2) G\tilde{G}$, where $\theta$ is an angular parameter. Experimental limits on the neutron electric dipole moment (nEDM), however, constrain $|\theta_{QCD}| < 10^{-10}$. Standard physics offers no fundamental reason for this extreme fine-tuning, motivating theories such as the Peccei-Quinn mechanism and the axion.

\paragraph{The E8-Persistence Derivation}
The $E_8$-Persistence framework resolves the Strong CP problem not by dynamic suppression, but by \textbf{Topological Prohibition}. The interaction subspace that hosts the Strong Force is geometrically orthogonal to the spacetime boundary where topological winding numbers are defined. This prohibition is enforced by two independent and mutually reinforcing principles.

\subparagraph{Argument 1: The Entropic Ground State}
A non-zero $\theta_{QCD}$ requires the gauge field to possess a topological winding number around the spacetime manifold ($\pi_3(S^3)$). This imposes two distinct and prohibitive informational costs:
\begin{enumerate}
    \item \textbf{Specification Cost:} The vacuum would need to encode the specific value of $\theta \in [0, 2\pi)$—a continuous parameter that requires infinite precision and thus carries unbounded Shannon entropy.
    \item \textbf{Coupling Cost:} For the winding number to be meaningful, the color field (residing in the internal $\sigma-\chi$ subspace) would have to maintain a persistent correlation with the spacetime boundary ($\chi=2$ surface). This cross-sector coupling requires continuous informational overhead to maintain coherence against decoherence.
\end{enumerate}
The decoupled state where $\theta_{QCD}$ is geometrically undefined avoids both costs ($S_{spec}=0$, $S_{couple}=0$). By the Principle of Least Entropic Action, the vacuum must relax to the state of minimal information content, which is the geometric ground state where the winding number does not exist.
\begin{equation}
    \theta_{QCD} \equiv 0 \quad \text{(Entropic Ground State)}
\end{equation}

\subparagraph{Argument 2: The Unitary CP Budget}
The lattice's finite capacity imposes a strict conservation law on time-reversal symmetry breaking, quantified by the Jarlskog invariant, $J_{total}$. The geometric structure allocates the entire CP capacity to the chiral sector (the Weak Boundary, $\chi=2$), where it manifests as the CKM phase: $J_{total} = J_{weak} \approx 3 \times 10^{-5}$. This exhausts the available budget. The internal symmetry space ($\sigma-\chi=3$) that hosts the Strong Force is geometrically forbidden from generating time asymmetry because it lacks a projection onto the temporal boundary. The color sector is \textbf{informationally isolated} from the Arrow of Time. Therefore, its contribution to the CP budget must be identically zero:
\begin{equation}
    J_{strong} \equiv 0 \quad \Rightarrow \quad \theta_{QCD} = 0
\end{equation}

\paragraph{Comparison to the Axion Mechanism}
This geometric resolution is fundamentally different from the dynamical relaxation proposed by the Peccei-Quinn mechanism. The two solutions are mutually exclusive, as summarized in Table \ref{tab:cp_solutions}.
\begin{table}[h]
\centering
\caption{Strong CP Solutions: Axion vs. Geometric Prohibition}
\begin{tabular}{lcc}
\toprule
\textbf{Feature} & \textbf{Peccei-Quinn Mechanism} & \textbf{E8-Persistence} \\
\midrule
Mechanism & Dynamic relaxation & Topological prohibition \\
New Particle Required & Yes (the Axion) & No \\
New Symmetry Required & Yes (U(1)$_{PQ}$) & No \\
Resulting $\theta_{QCD}$ value & Dynamically driven to be small & Geometrically fixed to be zero \\
Testable via & Axion searches & Null results in axion searches \\
\bottomrule
\end{tabular}
\label{tab:cp_solutions}
\end{table}

\paragraph{Falsifiable Predictions}
This geometric prohibition generates two hard-falsifiable predictions:
\begin{enumerate}
    \item \textbf{No QCD Axion:} Because $\theta_{QCD}=0$ is the geometric ground state, no dynamical relaxation mechanism is needed. The QCD axion, as proposed to solve the Strong CP problem, is therefore predicted not to exist. A confirmed discovery of the QCD axion with the expected properties would falsify this framework. (Note: This does not exclude other axion-like particles (ALPs) whose existence is not tied to the Peccei-Quinn mechanism.)
    
    \item \textbf{Zero Strong-Sector nEDM:} The contribution to the neutron electric dipole moment from $\theta_{QCD}$ is exactly zero. The only non-zero contribution arises from the CKM phase (a weak interaction effect), calculated to be $d_n^{weak} \sim 10^{-32} \, e\cdot\text{cm}$ \cite{pospelov_electric_2005}. Given the current experimental limit of $|d_n| < 1.8 \times 10^{-26} \, e\cdot\text{cm}$ \cite{abel_measurement_2020}, any confirmed measurement of $d_n > 10^{-30} \, e\cdot\text{cm}$ would constitute a major deviation from the Standard Model background and would falsify our geometric framework.
\end{enumerate}