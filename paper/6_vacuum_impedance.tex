\section{The Persistence Condition: Vacuum Impedance (\texorpdfstring{$\alpha^{-1}$}{alpha\string^-1})} \label{sec:Vacuum_Impedance}
\CatchFileBetweenTags{\AlphaInvVal}{calculations/results.tex}{AlphaInvVal}

Having established the Entropic Lagrangian, we determine the vacuum's primary boundary condition: the Fine-Structure Constant ($\alpha$) at the zero-momentum limit ($q^2 \to 0$).

\textbf{The Standard Model Ansatz:} The Fine-Structure Constant $\alpha$ is an empirical parameter ($\approx 1/137$) that describes the strength of the electromagnetic interaction. While it is measured with extreme precision, the Standard Model offers no mechanism to derive its magnitude. It remains a ``magic number" required to fit the data, but devoid of structural origin.

\textbf{The $E_8$-Persistence Derivation:} We derive $\alpha^{-1}$ as the \textbf{Geometric Impedance} ($Z_{\Phi}$) of the substrate, the minimum Action cost required to sustain a coherent topological defect against the entropic flux of the lattice.

\textbf{The Geometric Impedance Equation}
\begin{equation}\label{eq:alpha_inverse}
\begin{split}
\alpha^{-1} \equiv Z_{\Phi} = \underbrace{\pi\Delta}_{\Delta E}
+ \,\underbrace{\chi}_{\Delta I}
- \,\underbrace{\frac{1}{D\Delta - \sigma}}_{MI}
- \,\underbrace{\frac{\chi}{\Delta}}_{G} & \\
+ \,\underbrace{\frac{1}{N^3} \cdot \frac{\chi}{\sigma} \cdot \left( 1 - \frac{\sigma}{D\Delta} \right)}_{T}
+ \,\underbrace{\frac{1}{H_{full} \cdot (\sigma + 1) \cdot \Delta^2}}_{PM}
\end{split}
\end{equation}
We now derive each term in sequence. 

\subsection{The Impedance Ansatz}
For a topological defect (particle) to exist stably, its geometric structure must balance against the vacuum's resistance to deformation. We define the Geometric Impedance $Z$ as the Entropic Action cost per unit of topological charge ($Q_{top}$):

\begin{equation}
\alpha^{-1} \equiv Z_{\Phi} = \frac{S_\Phi}{Q_{top}}
\end{equation}

For the electromagnetic field, the topological charge is quantized by the boundary condition $\chi = 2$. The total impedance decomposes into independent geometric contributions, each corresponding to a distinct structural requirement for persistence derived from the Informational Energetics pillars.

\begin{equation}
\alpha^{-1} = Z_{base} + Z_{corrections}
\end{equation}

\subsection{The Base Geometry (\texorpdfstring{$Z_{base}$}{Zbase})}
The dominant contribution to the impedance comes from the fundamental geometry of the interaction loop.

\subsubsection{The Resonant Circumference (Energy Vessel)}
\begin{equation}
\underbrace{\pi\Delta}_{\Delta E}
\end{equation}
A topological defect must complete a closed gauge cycle to maintain invariance. The minimum non-trivial loop wraps the fundamental linear resonance ($\Delta$) around the circular topology of the gauge field ($\pi$). Thus, $\pi$ acts as the Geometric Conversion Factor, translating the discrete linear resonance into continuous gauge flux.

\textit{Justification:} If removed, the gauge field has no geometric extent and the universe would contain no electromagnetic field.

\subsubsection{The Topological Boundary (Information Model)}
\begin{equation}
\underbrace{+ \chi}_{\Delta I}
\end{equation}

A particle is distinguished from the vacuum by its boundary. By the Gauss-Bonnet theorem, a closed, stable surface in this manifold requires an Euler characteristic of $\chi=2$. This enters as an additive constant representing the minimum action cost to define ``Self'' vs. ``Environment.'' converting a continuous wave into a discrete entity. If removed, charges cannot be quantized; the universe would be a featureless superfluid.

\paragraph{Synthesis: The Minimal Wilson Loop ($Z_0$)}
The combination of the Resonant Circumference ($\pi\Delta$) and the Topological Boundary ($\chi$) generates the fundamental observable of Lattice Gauge Theory: the \textbf{Wilson Loop}.

In standard physics, the Wilson Loop $W_C$ measures the phase change of a field around an arbitrary path. In the $E_8$ lattice, the path is not arbitrary; it is constrained by the resonant geometry. The \textbf{Base Impedance} ($Z_{base}$) is the action cost of the minimal possible Wilson Loop supported by the substrate:

\begin{equation}
Z_{base} = \underbrace{\pi\Delta}_{\text{Circumference}} + \underbrace{\chi}_{\text{Boundary}} = \pi(43) + 2 \approx \mathbf{137.088\dots}
\end{equation}
\textbf{Note:} This base value matches the experimental Fine-Structure Constant to within \textbf{0.03\%}. The remaining four terms in the equation are the thermodynamic corrections required to stabilize this loop within a finite-capacity lattice.

\subsection{The Systemic Corrections (\texorpdfstring{$Z_{corrections}$}{Zcorrections})}
The physical lattice is not ideal; it is discrete and resource-constrained. We derive the four perturbation terms required to stabilize the ideal knot within the finite $E_8$ projection.

\subsubsection{Alignment Efficiency (Coordination Protocol)}
\begin{equation}
- \underbrace{\frac{1}{D\Delta - \sigma}}_{MI}
\end{equation}
The lattice possesses 5-fold internal symmetry ($\sigma=5$) which must project onto a 4-dimensional spacetime manifold ($D=4$). This geometric mismatch creates friction. The system minimizes this drag by aligning the manifold geometry ($D\Delta$) with the internal symmetry axes. This term represents the Strain Relief provided by this alignment.

$$ C_{res} = D\Delta - \sigma = 172 - 5 = 167 $$
In network theory, Impedance ($Z$) is the inverse of Admittance ($Y$). Since $C_{res}$ represents the available degrees of freedom (Admittance) for alignment, the impedance reduction is the reciprocal:
\begin{equation}
Z_{MI} = -\frac{1}{C_{res}} = -\frac{1}{167} \approx -0.00599
\end{equation}

If removed, \textbf{GST Violation.} The geometric link between the Gauge Sector and the Flavor Sector breaks. The Weak Mixing Angle would decouple from the Cabibbo Angle, violating the Gatto-Sartori-Tonin relation.

\textit{Forward Link:} This term structurally locks the Electromagnetic force to the Weak force (See Paper III: The GST Relation).

\subsubsection{Metric Shear (Stabilizing Governor)}
\begin{equation}
- \underbrace{\frac{\chi}{\Delta}}_{G}
\end{equation}

The continuous projection of the $E_8$ lattice defines a ``Continuous Limit,'' a frictionless geometric superfluid where energy scales linearly with frequency. However, to support discrete matter, the vacuum must enforce a \textbf{Topological Boundary} ($\chi=2$) against the \textbf{Resonant Depth} ($\Delta=43$).

This conflict between the continuous field and the discrete boundary creates a negative pressure on the system. By Hooke's Law, the restoring force is proportional to the strain, defined here as the ratio of the discrete boundary size to the continuous field depth. This term acts as a \textbf{Metric Shear}, a subtractive impedance required to prevent Ultraviolet Divergence.

\begin{equation}
Z_{G} = -\frac{\chi}{\Delta} = -\frac{2}{43} \approx -0.04651
\end{equation}

\paragraph{The Continuous Projection Limit}
We independently validate the integer derivation by analyzing the continuous limit of the 4D projection. The continuous projection of the $E_8$ lattice into 4D space via $H_4$ geometry defines the continuous vacuum impedance based on the Golden Ratio ($\phi$):
\begin{equation}
\alpha^{-1}_{cont} = (D \cdot \sigma) \cdot \phi^4 = 20 \times 6.854 \approx \mathbf{137.082}
\end{equation}
This value represents the vacuum without particles. To instantiate the discrete topology required for matter ($\chi=2$), the system must pay exactly the Governor cost derived above:
\begin{equation}
\alpha^{-1}_{physical} \approx \alpha^{-1}_{cont} - Z_G = 137.082 - 0.047 \approx \mathbf{137.035}
\end{equation}
\textbf{Conclusion:} The Governor is the \textbf{Metric Shear} required to lock continuous geometry into discrete topology. This structural duality—\textbf{Integer Knots vs. Golden Waves}—forms the geometric basis for the flavor mixing disparities derived in Paper III.

\subsubsection{Electroweak Transition (The Temporal Tax)}
\begin{equation}
+ \underbrace{\frac{1}{N^3} \cdot \frac{\chi}{\sigma} \cdot \left( 1 - \frac{\sigma}{D\Delta} \right)}_{T}
\end{equation}

The Weak interaction enables state transitions (Time). Unlike the gauge field which exists everywhere, a state transition (Time) is a localized update. The impedance cost $T$ is the probability that a random fluctuation successfully accesses the transition channel. 

A state transition must satisfy three independent geometric constraints, each contributing a probability factor:

\begin{enumerate}
    \item \textbf{Volumetric Addressing ($1/N^3$):} The system must select the specific node $(x,y,z)$ within the lattice's 3-dimensional projection. The probability of selecting the correct coordinate from the total state capacity ($N=32$) is $1/N^3$.
    \item \textbf{Boundary Selection ($\chi/\sigma$):} The transition probability scales with the Topological Boundary ($\chi=2$) to the Interaction Symmetry ($\sigma=5$). Only signals coupling to the boundary can effect a persistent change.
    \item \textbf{Bandwidth Availability ($1 - \sigma/D\Delta$):} The fraction of the projected manifold capacity available for signal propagation after the symmetry overhead is subtracted.
\end{enumerate}
This is Landauer's Limit applied to the lattice.

\begin{equation}
Z_T = \frac{1}{N^3} \cdot \frac{\chi}{\sigma} \cdot \left(1 - \frac{\sigma}{D\Delta}\right) \approx +1.185 \times 10^{-5}
\end{equation}

\paragraph{Geometric Consistency Condition}
We observe that the derived Temporal Tax ($T$) satisfies a quadratic consistency relation with the vacuum coupling ($\alpha$) and the lattice symmetry ($\sigma=5$):
\begin{equation}
T \approx \frac{\alpha^2}{2\sqrt{\sigma}} = \frac{\alpha^2}{2\sqrt{5}}
\end{equation}
This creates a closed consistency loop: the lattice geometry determines $T$, which contributes to the total impedance $\alpha^{-1}$, which in turn results in a field strength that satisfies $T \approx \alpha^2/2\sqrt{5}$. The observed value $\alpha \approx 1/137$ emerges as the unique stable solution to this constraint.

\begin{itemize}
    \item \textbf{Quantization Noise:} The slight divergence ($0.5\%$) between the Integer Derivation and this continuous consistency condition represents the \textbf{Quantization Noise} of mapping an irrational symmetry geometry ($\sqrt{5}$) onto a discrete integer lattice. The integer set $\{43, 16, 5, 4, 2\}$ is the unique "Best Rational Approximation" that maintains lattice solvency.
    \item \textit{Forward Link:} This term structurally links the vacuum geometry to the Weak Force ($T \approx \alpha^2 \sin^2 \theta_W$) and Time Asymmetry ($J \approx \phi^2 T$). (Derivation provided in \cref{sec:JarlskogInvariant}).
\end{itemize}

\subsubsection{Mass Resolution Floor (The Persistence Margin)}
\begin{equation}
+ \underbrace{\frac{1}{H_{full} \cdot (\sigma + 1) \cdot \Delta^2}}_{PM}
\end{equation}

This defines the minimum resolvable mass signal against thermal noise. The lattice has a finite resolution limit defined by the Full Persistence Budget ($H_{full} = 31$) and the \textbf{Weak Interaction Aperture}.
\textit{Justification:} As verified in Paper II, the Weak Force acts through an aperture of $\sigma+1=6$ (Symmetry + Vacuum Unit). The resolution floor is the inverse of the total system capacity ($H_{full}$) scaled by this aperture and the resonant area ($\Delta^2$). If removed the electron coupling falls below the resolution limit of the vacuum; the particle dissolves into radiation.

\begin{equation}
Z_{PM} = \frac{1}{H_{full} \cdot (\sigma + 1) \cdot \Delta^2} \approx +2.91 \times 10^{-6}
\end{equation}

\textit{Forward Link:} This Establishes the \textbf{Geometric Baseline} for the Electron mass (See Paper II).








\subsection{Numerical Result}
Summing the base geometry and the systemic corrections:
\begin{equation}
\alpha^{-1}_{calc} = 135.0885 + 2 - 0.00599 - 0.04651 + 0.0000119 + 0.0000029
\end{equation}
\begin{equation}
\alpha^{-1}_{calc} = \AlphaInvVal
\end{equation}

\begin{itemize}
    \item \textbf{Geometric Prediction:} \AlphaInvVal
    \item \textbf{Experimental Average (CODATA 2022):} $137.035999178(8)$ \cite{mohr_codata_2025}
    \item \textbf{Morel (2020) Value}: $137.035999206(11)$ \cite{morel_determination_2020}
    \item \textbf{Precision:} The geometric derivation lies within the \textbf{$0.8\sigma$ uncertainty interval} of the experimental consensus.
\end{itemize}


\subsection{Theorem of Impedance Uniqueness}

We formally assert that the derived equation for $\alpha^{-1}$ is not merely consistent with observation, but is the unique solution mandated by the substrate geometry.

\textbf{Theorem:} Given a discrete $E_8$ lattice projected onto a causal $D=4$ manifold subject to the Persistence Principle, the Vacuum Impedance $\alpha^{-1}$ is uniquely determined by the linear sum of the \textbf{Minimal Complete Basis} of geometric action costs.

\textit{Proof:}
The Impedance Functional $Z[\Psi]$ must span all available degrees of freedom in the projection to maintain unitarity. We decompose the projection geometry into its irreducible sectors:

\begin{enumerate}
    \item \textbf{The Metric Sector (1-Form):} The cost of spatial extension.
    \begin{itemize}
        \item \textit{Constraint:} Must couple the linear lattice depth ($\Delta$) to the gauge topology ($\pi$).
        \item \textit{Unique Term:} $\pi\Delta$ (The Circumference).
    \end{itemize}

    \item \textbf{The Topological Sector (0-Form):} The cost of distinct existence.
    \begin{itemize}
        \item \textit{Constraint:} Must satisfy the Gauss-Bonnet boundary condition for a closed knot.
        \item \textit{Unique Term:} $+\chi$ (The Euler Characteristic).
    \end{itemize}

    \item \textbf{The Symmetry Sector (Group Theoretic):} The cost of dimensional reduction.
    \begin{itemize}
        \item \textit{Constraint:} Must minimize friction between the internal symmetry ($\sigma$) and the manifold ($D$).
        \item \textit{Unique Term:} $-1/(D\Delta - \sigma)$ (The Admittance of the Reserve Capacity). Inverse scaling is required for efficiency/drag reduction.
    \end{itemize}

    \item \textbf{The Conformal Sector (Scale Invariance):} The cost of discrete quantization.
    \begin{itemize}
        \item \textit{Constraint:} Must balance the continuous field pressure ($\Delta$) against the discrete boundary ($\chi$) to prevent divergence.
        \item \textit{Unique Term:} $-\chi/\Delta$ (The Metric Shear). Ratio scaling is required for pressure/stress.
    \end{itemize}

    \item \textbf{The Entropic Sector (Probabilistic):} The cost of state selection.
    \begin{itemize}
        \item \textit{Constraint:} Must account for the non-zero entropy of selecting a specific node state ($Z_T$) and the resolution floor ($Z_{PM}$).
        \item \textit{Unique Terms:} The joint probabilities defined by the volumetric ($N^3$) and aperture ($\sigma+1$) limits.
    \end{itemize}
\end{enumerate}

\textbf{Completeness Argument:} The set of invariants $\mathbb{S} = \{D, \Delta, \nu, \sigma, \chi\}$ completely defines the projection $E_8 \to D_4$. There are no remaining independent integers in the system to construct additional terms. Any further terms would effectively double-count a degree of freedom, violating the Principle of Least Action.

Therefore, the summation $\alpha^{-1} = \sum Z_i$ represents the unique eigenvalues of the persistence equation. It is the sum of the geometric costs required to maintain a persistent, causal, solvent vacuum.
 \hfill $\square$