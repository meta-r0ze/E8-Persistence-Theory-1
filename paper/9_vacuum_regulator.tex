\section{The Vacuum Regulator: The Higgs Sector (\texorpdfstring{$v, \lambda, \mu^2, m_H, y_e$}{vlambdamu2mhye})} 
\label{sec:Vacuum_Regulator}
\CatchFileBetweenTags{\AlphaInvVal}{calculations/constants.tex}{AlphaInvVal}
\CatchFileBetweenTags{\MeMeVPrint}{calculations/constants.tex}{MeMeVPrint}

% Delta E
\CatchFileBetweenTags{\HiggsVEVVal}{calculations/constants.tex}{HiggsVEVVal}
\CatchFileBetweenTags{\HiggsVEVExperimentalValue}{calculations/constants.tex}{HiggsVEVExperimentalValue}
\CatchFileBetweenTags{\HiggsVEVAccText}{calculations/constants.tex}{HiggsVEVAccText}

\CatchFileBetweenTags{\FermiConstVal}{calculations/constants.tex}{FermiConstVal}
\CatchFileBetweenTags{\FermiConstExperimentalValue}{calculations/constants.tex}{FermiConstExperimentalValue}
\CatchFileBetweenTags{\FermiConstAccText}{calculations/constants.tex}{FermiConstAccText}

% MI
\CatchFileBetweenTags{\HiggsLambdaVal}{calculations/constants.tex}{HiggsLambdaVal}
\CatchFileBetweenTags{\HiggsLambdaExperimentalValue}{calculations/constants.tex}{HiggsLambdaExperimentalValue}
\CatchFileBetweenTags{\HiggsLambdaAccText}{calculations/constants.tex}{HiggsLambdaAccText}

% T
\CatchFileBetweenTags{\HiggsMassVal}{calculations/constants.tex}{HiggsMassVal}
\CatchFileBetweenTags{\HiggsMassExperimentalValue}{calculations/constants.tex}{HiggsMassExperimentalValue}
\CatchFileBetweenTags{\HiggsMassAccText}{calculations/constants.tex}{HiggsMassAccText}

% PM
\CatchFileBetweenTags{\ElectronYukawaVal}{calculations/constants.tex}{ElectronYukawaVal}
\CatchFileBetweenTags{\ElectronYukawaExperimentalValue}{calculations/constants.tex}{ElectronYukawaExperimentalValue}
\CatchFileBetweenTags{\ElectronYukawaAccText}{calculations/constants.tex}{ElectronYukawaAccText}

\textbf{The Standard Model Ansatz:} In the Standard Model, the electroweak sector is parameterized by independent inputs ($v, \lambda, \mu^2$) to generate the Higgs potential $V(\phi) = -\mu^2|\phi|^2 + \lambda|\phi|^4$. While effective, this offers no structural reason for the specific energy scale ($v \approx 246$ GeV) or the coupling strength ($\lambda \approx 0.13$).

\textbf{The $E_8$-Persistence Derivation:} We have established the Lattice Hardware (System I) and the Vacuum Impedance (System III). However, a raw feed from the lattice resonance ($\Delta$) is too energetic to couple directly to matter. The universe requires a \textbf{Step-Down Transformer} to convert the high-frequency lattice potential into a stable mass scale.

We identify the Higgs Field not merely as a boson, but as a \textbf{Nested Persistent System} a fractal replica of the vacuum architecture designed to regulate the electroweak scale. It replicates the six pillars of Informational Energetics to create a stable ``energy vessel'' ($v$) within the larger lattice.


\subsection{Energy Vessel (\texorpdfstring{$\Delta E_H$}{dE}): The Vacuum Expectation Value (\texorpdfstring{$v$}{v})}
The VEV represents the capacity of the subsystem. Because the Electron ($\Delta^0$) is the unique Unitary Ground State ($N=0$) of the lattice, it acts as the fundamental \textbf{Mass Unit} against which the vacuum potential is normalized.

\subsubsection{Step 1: The Bare Geometric Floor (\texorpdfstring{$v_{geo}$}{vgeo})}
We first calculate the static potential minimum defined by the lattice invariants:
\begin{equation}
v_{geo} = (\chi \Delta^2 - I_s) \cdot \alpha^{-1} \cdot m_e 
\end{equation}

\noindent \textbf{Structural Overhead ($I_s$):}
$$ I_s = (\Delta \cdot D) + \nu = (43 \times 4) + 16 = \mathbf{188} $$

\noindent Substituting the invariants:

\begin{equation}
v_{geo} = (2 \cdot 43^2 - 188) \cdot \AlphaInvVal \cdot \MeMeVPrint \text{ MeV}
\end{equation}
\begin{equation}
v_{geo} \approx 245.789 \text{ GeV}
\end{equation}

\subsubsection{Step 2: Radiative Correction (Topological Screening)}
The field is screened by the electromagnetic topology. The screening medium consists of the spatial manifold ($D=4$) plus the topological boundary charge ($\chi=2$) distributed across the full spherical phase space of the gauge field ($4\pi$).
\begin{equation}
D_{eff} = D + \frac{\chi}{4\pi} \approx 4.15915
\end{equation}

\begin{equation}
v_{screened} = v_{geo} \left( 1 + \frac{\alpha}{D_{eff}} \right) \approx 246.2201 \text{ GeV}
\end{equation}

\subsubsection{Step 3: The Thermodynamic Noise Floor}
Finally, we account for the finite resolution of the lattice. As derived in System II, the vacuum possesses a \textbf{Persistence Margin} ($PM$) representing the minimum fluctuation amplitude. This noise reduces the effective depth of the potential well.
Because the vacuum stability floor supports $n_{gen}=3$ generations ($\sigma - \chi = 3$), the noise is partitioned linearly across the generation manifold.

\begin{equation}
v_{phys} = v_{screened} \left( 1 - \frac{PM}{3} \right)
\end{equation}

\textbf{Calculation:}
Substituting $PM \approx 2.91 \times 10^{-6}$:
\begin{equation}
v_{phys} = 246.2201 \text{ GeV} \times (1 - 9.7 \times 10^{-7}) \approx \textbf{246.219876} \text{ GeV}
\end{equation}

\begin{itemize}
    \item \textbf{Geometric Prediction:} $\HiggsVEVVal$ GeV
    \item \textbf{Experimental Value:} \HiggsVEVExperimentalValue
    \item \textbf{Accuracy:} \HiggsVEVAccText
\end{itemize}

\subsubsection*{Derived Limit: The Fermi Constant ($G_F$)}
$G_F$ is the inverse squared cross-section of this stability floor. The normalization factor $\sqrt{2}$ is identified as the square root of the Topological Boundary ($\chi=2$):
\begin{equation}
G_F = \frac{1}{\sqrt{\chi} v^2} \approx \mathbf{\FermiConstVal \text{ GeV}^{-2}}
\end{equation}
(\textbf{Accuracy:} \FermiConstAccText)



\subsection{Information Model (\texorpdfstring{$\Delta I_H$}{dI}): The Scalar Charge (\texorpdfstring{$Y$}{Y})}
The \textbf{Information Model} defines the identity signature of the system within the gauge group. For the Higgs field, this corresponds to its Hypercharge ($Y$).

As established in \textbf{Appendix \cref{sec:OriginOfHypercharg}}, quantum numbers in this framework are geometric ratios. The Higgs boson is the scalar excitation of the vacuum's Unitary Ground State ($\Delta^0 = 1$). Consequently, its Hypercharge is derived as the inverse of this ground state resonance:

\begin{equation}
Y_H = \frac{1}{\Delta^0} = 1
\end{equation}

This unitary charge ($Y=1$) combined with the topological boundary constraint ($\chi=2$) mandates the $SU(2)_L$ doublet structure ($\mathbf{2}$) required for the Information Model to interface with the Chiral Diode ($\nu=16$).

\subsection{Coordination Protocol (\texorpdfstring{$MI_H$}{MI}): Self-Coupling (\texorpdfstring{$\lambda$}{lambda})}
The self-coupling $\lambda$ represents the \textbf{Coordination Protocol} of the Higgs system. It determines the bandwidth allocated for self-interaction to maintain field coherence.

$\lambda$ is defined as the fraction of the Total Systemic Capacity ($H_{sys}$) reserved for the Interaction Remainder (Color/Strong Force). However, to sustain temporal coherence against the lattice oscillation ($\Delta$), the system must pay a \textbf{Resonant Tax} of one unit of inverse-bandwidth ($1/\Delta$). This represents the metabolic cost of maintaining coherence across the fundamental oscillation period: as the lattice "breathes" at frequency $\Delta$, the Higgs field must continuously re-synchronize its phase, consuming bandwidth proportional to $1/\Delta$.

\begin{equation}
\lambda = \frac{\text{Interaction Remainder} - \text{Resonant Tax}}{\text{System Capacity}} = \frac{(\sigma - \chi) - \frac{1}{\Delta}}{\nu + \sigma + \chi}
\end{equation}

\begin{equation}
\lambda = \frac{3 - \frac{1}{43}}{23} = \frac{2.976744}{23} \approx \mathbf{\HiggsLambdaVal}
\end{equation}

\begin{itemize}
    \item \textbf{Experimental Value:} \HiggsLambdaExperimentalValue
    \item \textbf{Accuracy:} \HiggsLambdaAccText
\end{itemize}







\subsection{Stabilizing Governor (\texorpdfstring{$G_H$}{G}): The Quartic Potential}
While $\lambda$ represents the Coordination Protocol (bandwidth allocation), the quartic term $\lambda|\phi|^4$ in the potential acts as the \textbf{Stabilizing Governor}. This geometric bounding potential prevents the field from diverging to infinity under the negative mass pressure.

In Informational Energetics, the Governor enforces the topological boundary constraint ($\chi=2$). For the Higgs, this manifests as the requirement that the potential possess exactly two stable minima (the "Mexican hat" structure):

\begin{equation}
V(\phi) = -\mu^2|\phi|^2 + \lambda|\phi|^4
\end{equation}

The derived value $\lambda \approx 0.129$ is the precise structural stiffness required to enforce this boundary condition against vacuum pressure, ensuring the potential stabilizes at the thermodynamic minimum $|\phi| = v/\sqrt{2}$.

Without the quartic term, the tachyonic mass ($-\mu^2$) would cause runaway condensation. The Governor caps this divergence, implementing the fundamental constraint that persistent systems must have finite capacity.



\subsection{Temporal Tax (\texorpdfstring{$T_H$}{T}): The Instability Factor (\texorpdfstring{$\mu^2$}{mu})}
The Higgs field requires a source of instability to drive Spontaneous Symmetry Breaking. In IE, this is the \textbf{Temporal Tax ($T$)}, the metabolic cost of maintaining a broken symmetry state (the "False Vacuum") distinct from the origin.

We identify the dimensionless instability factor as twice the self-coupling bandwidth:
\begin{equation}
T_H = 2\lambda \approx 0.259
\end{equation}

The factor of 2 arises because the Higgs doublet contains two independent complex fields (four real degrees of freedom), each contributing a $\lambda$ term to the vacuum instability rate. This recovers the Standard Model relation $\mu^2 = \lambda v^2$ from geometric first principles.

This solves the "Tachyonic Mass" problem: the negative mass parameter $-\mu^2$ is simply the manifestation of this tax acting on the VEV capacity:
\begin{equation}
\mu^2 = \lambda v^2 = \frac{T_H}{2} v^2 = \frac{2\lambda}{2} v^2 = \lambda v^2
\end{equation}

\subsection{Output: The Higgs Mass (\texorpdfstring{$m_H$}{mH})}
With the Capacity ($v$) and Protocol ($\lambda$) established, the mass of the scalar excitation is the closure of the geometric system:

\begin{equation}
m_H = \sqrt{2\lambda} v_{phys}
\end{equation}

\begin{equation}
m_H = \sqrt{2 (0.12942)} \cdot (\HiggsVEVVal \text{ GeV}) \approx \mathbf{\HiggsMassVal \text{ GeV}}
\end{equation}

\begin{itemize}
    \item \textbf{Experimental Value:} \HiggsMassExperimentalValue
    \item \textbf{Accuracy:} \HiggsMassAccText
\end{itemize}

\subsection{Persistence Margin (\texorpdfstring{$PM_H$}{PM}): The Electron Yukawa (\texorpdfstring{$y_e$}{ye})}
The final component is the resolution floor of the regulator. The \textbf{Persistence Margin} ($PM$) derived in System II represents the smallest non-zero bit of mass the lattice can resolve against thermal noise. This sets the coupling of the lightest charged particle (the Electron).

\begin{equation}
    y_{e, bare} = PM_{geo} = \frac{1}{H_{full} \cdot (\sigma + 1) \cdot \Delta^2} \approx 2.9077 \times 10^{-6}
\end{equation}

Applying the standard electromagnetic self-energy correction $(1+\alpha)$:
\begin{equation}
    y_{e, phys} \approx 2.9077 \times 10^{-6} (1.00730) \approx \ElectronYukawaVal
\end{equation}

\textbf{Validation:} This matches the Standard Model value $y_e = \sqrt{2}m_e/v$ to within $0.2\%$, confirming the electron exists at the absolute limit of the vacuum's resolution.

\subsection{Closure: The Impedance Matching Condition}
If the Higgs is a functional regulator, its total internal impedance must match the aperture of the force it mediates (the Weak Interaction).

\begin{enumerate}
    \item \textbf{Weak Aperture:} From the System I invariants, the Weak Force acts through the aperture defined by Symmetry plus the Vacuum Unit:
    $$ \text{Aperture} = \sigma + 1 = 6 $$
    \item \textbf{Higgs Impedance ($Z_H$):} Defined by the ratio of Capacity ($v$) to Protocol ($\lambda$), modulated by the Temporal Tax ($e^{-T_H} = e^{-2\lambda}$):
    \begin{equation}
    Z_H(\lambda) = \frac{1}{\lambda} e^{-2\lambda}
    \end{equation}
\end{enumerate}

\textbf{The Test:} Substituting the derived geometric coupling $\lambda \approx 0.129424$:
\begin{equation}
Z_H \approx \frac{1}{0.129424} \cdot e^{-0.2588} \approx 7.726 \cdot 0.772 \approx \mathbf{5.966}
\end{equation}

The result $5.966 \approx 6$ (error $< 0.6\%$) confirms that the Higgs sector is not arbitrary. It is the unique geometric solution that \textbf{impedance-matches} the Lattice Resonance to the Weak Force Aperture.


\subsection{Validation: The Higgs Satisfies the Persistence Principle}

To confirm the Higgs is a legitimate persistent system, we verify it minimizes Entropic Action subject to the six-pillar constraints.

The Higgs action functional is:
\begin{equation}
S_H = \int d^4x \left[ |D_\mu\phi|^2 - V(\phi) \right]
\end{equation}

where the potential emerges from the IE constraints:
\begin{equation}
V(\phi) = -\mu^2|\phi|^2 + \lambda|\phi|^4
\end{equation}

The system minimizes $S_H$ by transitioning from the unstable false vacuum ($\phi=0$) to the true vacuum ($|\phi| = v/\sqrt{2}$). This transition occurs because:

\begin{enumerate}
    \item \textbf{Energy Vessel ($\Delta E_H$)}: The capacity $v$ is determined by lattice invariants (Eq. X).
    \item \textbf{Coordination Protocol ($MI_H$)}: The coupling $\lambda$ is fixed by bandwidth allocation (Eq. Y).
    \item \textbf{Temporal Tax ($T_H$)}: The instability factor $T_H = 2\lambda$ drives SSB.
    \item \textbf{Stabilizing Governor ($G_H$)}: The quartic term prevents divergence, enforcing $\chi=2$.
\end{enumerate}

The vacuum condition $\partial V/\partial|\phi| = 0$ yields:
\begin{equation}
\mu^2 = \lambda v^2
\end{equation}