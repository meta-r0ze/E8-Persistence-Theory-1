\section{The Geometric Partition Ratio (Weak Mixing Angle) (\texorpdfstring{$\sin^2 \theta_W$}{sin2thetaW})} \label{sec:Partition_Ratio}
\CatchFileBetweenTags{\AlphaInvVal}{calculations/constants.tex}{AlphaInvVal}

\CatchFileBetweenTags{\WeakAngleVal}{calculations/constants.tex}{WeakAngleVal}

\CatchFileBetweenTags{\WBosonMassVal}{calculations/constants.tex}{WBosonMassVal}

\textbf{The Standard Model Ansatz:} The Weak Mixing Angle governs the unification of the electromagnetic and weak forces. In the Standard Model, this is a free parameter determined by experiment ($\sin^2\theta_W \approx 0.223$). In the $E_8$-Persistence framework, it is a \textbf{Capacity Partition Ratio}—the fraction of the lattice's information bandwidth allocated to the temporal resonance.

\subsection{The Geometric Interpretation}
The Weak Mixing Angle represents the relative "weight" of the Fundamental Resonance ($\Delta$) in the total system dynamics. To quantify this, we define the \textbf{Total System Capacity} ($N_{sys}$) as the sum of all structural degrees of freedom in the lattice projection:

\begin{enumerate}
    \item \textbf{Manifold Projection ($D\Delta$):} The lattice resonance $\Delta=43$ must be embedded in all $D=4$ spacetime dimensions. This creates $D \times \Delta = 172$ space-time degrees of freedom (4 coordinates $\times$ 43 lattice sites). This represents the cost of spatial propagation.
    
    \item \textbf{Chiral Capacity ($\nu \cdot \eta$):} The fermion manifold requires $\nu=16$ degrees of freedom to encode the chiral spinor structure (left-handed Weyl fermions). As a \textbf{Bulk Capacity} projected onto the discrete manifold, it is subject to Universal Manifold Friction. This represents the cost of matter storage.
    
    \item \textbf{Symmetry Generators ($\sigma$):} The interaction group possesses $\sigma=5$ independent generators (the rank of the gauge symmetry). This represents the cost of force mediation.
\end{enumerate}

These components represent orthogonal sectors of the system.
\begin{itemize}
    \item $D\Delta$ encodes \textbf{spacetime structure} (where signals propagate).
    \item $\nu$ encodes \textbf{matter content} (what exists at each point).
    \item $\sigma$ encodes \textbf{interaction rules} (how signals transform).
\end{itemize}

Since these sectors are independent and orthogonal, their capacities add linearly to define the total bandwidth of the vacuum:
\begin{equation}
N_{sys} = D\Delta + (\nu \cdot \eta) + \sigma = 172 + 15.907 + 5 \approx \mathbf{192.907}
\end{equation}

\subsection{The Partition Formula}

The Weak Mixing Angle is defined as the fraction of the total capacity that is allocated to the Fundamental Resonance (the "clock frequency" $\Delta$ of the lattice):

\begin{equation}
\sin^2 \theta_W = \frac{\text{Fundamental Resonance}}{\text{Total Capacity}} = \frac{\Delta}{N_{sys}}
\end{equation}

\textbf{Physical Meaning:} This ratio represents the probability that a random fluctuation in the system is a \textbf{time translation} (governed by the bare resonance $\Delta$) rather than a \textbf{spatial translation} (governed by the manifold $D$) or an \textbf{internal symmetry transformation} (governed by $\nu$ or $\sigma$).

\textbf{Note on the Asymmetry:} The resonance $\Delta$ appears both in the numerator (as the bare frequency) and in the denominator (as part of $D\Delta$, the spacetime-embedded frequency). This reflects the competition between temporal coherence and spatial extension—the more dimensions the signal must propagate through ($D$), the more diluted the temporal component becomes relative to the total system capacity.

\subsection{Connection to Standard Electroweak Theory}

In the Standard Model, the mixing angle determines how the unified electroweak symmetry breaks into the massive $Z$ boson and the massless photon. The photon corresponds to the unbroken $U(1)_{em}$ symmetry.

Our derivation identifies the $U(1)$ hypercharge as the \textbf{Temporal Component} of the electroweak symmetry (driven by $\Delta$), while the $SU(2)$ weak isospin corresponds to the \textbf{Spatial Components} (driven by the manifold embedding).

The standard relation is:
\begin{equation}
\sin^2\theta_W = \frac{g'^2}{g^2 + g'^2}
\end{equation}

The $E_8$-Persistence derivation reveals that this coupling ratio is not arbitrary; it is the ratio of the temporal resonance to the total system bandwidth:
\begin{equation}
\frac{\Delta}{D\Delta + \nu + \sigma} \leftrightarrow \frac{g'^2}{g^2 + g'^2}
\end{equation}

\subsection{Numerical Result}

Substituting the geometric invariants:
\begin{equation}
\sin^2 \theta_W = \frac{43}{192.907} \approx \mathbf{\WeakAngleVal}
\end{equation}

\begin{itemize}
    \item \textbf{Geometric Prediction:} \WeakAngleVal
    \item \textbf{Experimental (Direct Mass)}: $0.22291 \pm 0.00011$ \cite{navas_review_2024}
    \item \textbf{Experimental (Global Fit)}: $0.22354 \pm 0.00006$
    \item \textbf{Experimental ($\overline{MS}$ Running)}: $0.23122$
    \item \textbf{Experimental (On-Shell):} $\approx 0.223 \pm 0.001$ (Depends on renormalization scheme)
    \item \textbf{Agreement:} $0.09\%$ precision
\end{itemize}


This result confirms that the electroweak partition is not a free parameter, but the inevitable consequence of allocating a finite lattice resonance ($\Delta=43$) across a 4-dimensional chiral manifold.

\subsection{Physical Interpretation: Time vs. Space}

The geometric derivation reveals a profound connection between the weak mixing angle and the structure of spacetime itself. The ratio $\sin^2\theta_W \approx 0.223$ means that approximately 22\% of the vacuum's information bandwidth is allocated to \textbf{temporal coherence} (the Fundamental Resonance $\Delta$), while the remaining 78\% is allocated to \textbf{spatial structure} (the manifold embedding $D\Delta$, matter content $\nu$, and symmetry operations $\sigma$).

This explains why electromagnetism (the $U(1)$ sector) is the "timelike" force—it couples most strongly to the temporal resonance. The weak force (the $SU(2)$ sector), by contrast, is the "spacelike" force—it couples to the spatial degrees of freedom.

The Higgs field (Section X) breaks the electroweak symmetry by selecting a vacuum state that distinguishes these sectors. The mixing angle $\sin^2\theta_W = 43/193$ determines the precise orientation of this symmetry breaking, explaining why the photon remains massless (it couples to the temporal resonance, which is never obstructed) while the Z boson acquires mass (it couples to spatial structure, which requires energy to deform).

\subsection{Implications for Grand Unification}

Many Grand Unified Theories (GUTs) assume the gauge couplings unify at a high energy scale ($M_{GUT} \sim 10^{16}$ GeV), predicting $\sin^2\theta_W(M_Z) \approx 0.21$, which is inconsistent with precision measurements ($\approx 0.223$).

The $E_8$-Persistence framework explains this discrepancy: $\sin^2\theta_W$ is not a running coupling determined by energy scale, but a \textbf{fixed geometric ratio} ($43/193$) determined by the lattice structure. The apparent "non-unification" of gauge couplings is not a failure of nature; it is evidence that the forces arise from distinct geometric features ($\Delta$ vs. $D\Delta$ vs. $\sigma$) rather than a single unified symmetry group breaking spontaneously.


% remove this?
\subsection{Recursive Validation: The Cost of Time}
The geometric validity of the Weak Mixing Angle is reinforced by a structural link back to the Geometric Impedance derived in \cref{sec:GeometricImpedance}.

Recall the \textbf{Temporal Cost} ($T$), the Entropic Action of a state transition defined purely by lattice integers in Eq. \ref{eq:alpha_inverse}:
$$ T_{geo} \approx 1.185 \times 10^{-5} $$

We now observe that this tax corresponds to the second-order electromagnetic coupling ($\alpha^2$) modulated by the Weak Mixing Angle partition we just derived ($\sin^2 \theta_W = 43/193$).

\begin{equation}
T_{check} = \alpha^2 \sin^2 \theta_W
\end{equation}

Substituting the strictly derived values from this framework:
$$ T_{check} = \left(\frac{1}{\AlphaInvVal}\right)^2 \cdot \left(\frac{43}{193}\right) $$
$$ T_{check} \approx (5.325 \times 10^{-5}) \cdot (0.2228) \approx \mathbf{1.186 \times 10^{-5}} $$

\textbf{Conclusion:} The match (within 0.1\%) confirms internal consistency. The Temporal Cost $T$ appearing in the Fine-Structure Constant is not a random correction; it is the specific geometric cost of authorizing a Weak Interaction. The Weak Force is the mechanism of Time (state transition), and $\alpha^2 \sin^2 \theta_W$ is the toll paid to the vacuum to execute it.

\textbf{Conclusion:} The integer derivation ($43/193$) correctly identifies the On-Shell angle derived from the physical W-boson mass (hitting the lower bound of the $1\sigma$ range), distinguishing it from the continuous field geometry ($1/2\sqrt{5} \approx 0.2236$) detected in global fits.





\subsubsection{Validation: Resolution of the W Boson Mass Anomaly}

The specific geometric partition derived above ($\sin^2 \theta_W = 43/193$) has immediate predictive power regarding the mass of the W Boson, a parameter currently subject to significant experimental tension.

\textbf{The Anomaly:} The mass of the W boson is currently the subject of a $7\sigma$ tension between theoretical prediction and experimental measurement.
\begin{itemize}
    \item \textbf{Standard Model Prediction:} $80.357 \pm 0.006$ GeV.
    \item \textbf{CDF II Measurement (2022):} $80.4335 \pm 0.0094$ GeV.
    \item \textbf{ATLAS Measurement (2023):} $80.360 \pm 0.016$ GeV.
\end{itemize}
The CDF result suggests new physics, while ATLAS results remain consistent with the Standard Model.

\textbf{The $E_8$ Derivation:} In this framework, the $W$ and $Z$ bosons are coupled via the \textbf{Geometric Partition} derived in Paper I. The Weak Mixing Angle is not a free parameter but the ratio of the Fundamental Resonance ($\Delta$) to the total electroweak capacity ($193$).
\begin{equation}
\sin^2 \theta_W = \frac{43}{193} \approx 0.22280
\end{equation}

In the On-Shell renormalization scheme, the masses are strictly related by the cosine of this angle:
\begin{equation}
M_W = M_Z \cos \theta_W = M_Z \sqrt{1 - \sin^2 \theta_W}
\end{equation}

\textbf{Calculation:}
Using the experimental Z-boson mass ($M_Z = 91.1876$ GeV) as the volumetric reference scale:
\begin{equation}
M_W = 91.1876 \text{ GeV} \times \sqrt{1 - \frac{43}{192.907}}
\end{equation}
\begin{equation}
M_W = 91.1876 \times \sqrt{0.7771} \approx 80.38
\end{equation}
\begin{equation}
M_W \approx \mathbf{\WBosonMassVal}
\end{equation}

\begin{itemize}
    \item \textbf{Geometric Prediction:} \WBosonMassVal
    \item \textbf{Standard Model:} $80.36$ GeV
    \item \textbf{CDF II:} $80.43$ GeV
\end{itemize}

\textbf{Result:} The geometric prediction acts as the mediator. It confirms that the Standard Model prediction is too low (validating the direction of the CDF anomaly) while suggesting the CDF value may be slightly overestimated.

\textbf{Conclusion:} The tension in the W mass arises because the Standard Model fits $\sin^2 \theta_W$ to global data ($\approx 0.2235$), whereas the lattice dictates a rigid geometric partition ($\approx 0.2229$). The lattice geometry predicts a heavier W boson than the Standard Model, resolving the direction of the anomaly without requiring new particles.