\section{System IV-B: The Geometric Partition Ratio (Weak Mixing Angle \texorpdfstring{$\sin^2 \theta_W$}{sin2thetaW})} \label{sec:Partition_Ratio}
\CatchFileBetweenTags{\AlphaInvVal}{calculations/constants.tex}{AlphaInvVal}
\CatchFileBetweenTags{\WeakAngleVal}{calculations/constants.tex}{WeakAngleVal}
\CatchFileBetweenTags{\WBosonMassVal}{calculations/constants.tex}{WBosonMassVal}

\textbf{The Standard Model Ansatz:} The Weak Mixing Angle ($\sin^2\theta_W \approx 0.223$) governs the unification of electromagnetic and weak forces. In the Standard Model, it is a free parameter determined by global fits to Z-pole data, with no structural origin for its specific value or connection to the W/Z mass ratio.

\textbf{The E8-Persistence Derivation:} We derive the Weak Mixing Angle as the \textbf{Capacity Partition Ratio}, the fraction of the lattice's total information bandwidth allocated to temporal evolution versus spatial/internal transformations.

\subsection{The System Specification}
To derive the partition, we must quantify the total active bandwidth of the lattice. We define the components of the \textbf{Total System Capacity} ($N_{sys}$) by instantiating the geometric invariants:

\begin{enumerate}
    \item \textbf{Spacetime Structure ($D\Delta$):} The lattice resonance ($\Delta=43$) must be embedded in all $D=4$ spacetime dimensions to define the propagation metric. 
    $$ C_{space} = D \times \Delta = 4 \times 43 = 172 $$
    
    \item \textbf{Matter Content ($\nu \cdot \eta$):} The chiral spinor manifold requires $\nu=16$ degrees of freedom to encode fermion quantum numbers. As a \textbf{bulk capacity} that must be distributed across the discrete lattice volume ($D\Delta = 172$ nodes), it pays the Universal Manifold Friction ($\eta = 1 - 1/(D\Delta)$), representing the projection efficiency from discrete states to continuous spacetime.
    $$ C_{matter} = \nu \cdot \eta \approx 15.9069 $$
    
    \item \textbf{Interaction Rules ($\sigma$):} The interaction group possesses $\sigma=5$ independent generators (the rank of the geometric symmetry), representing the cost of force mediation.
    $$ C_{force} = \sigma = 5 $$
\end{enumerate}

These three sectors: Space, Matter, and Force are orthogonal and add linearly to define the Total System Capacity:
\begin{equation} \label{eq:system_capacity}
\begin{split}
N_{sys} &= C_{space} + C_{matter} + C_{force} \\
& = 172 + 15.907 + 5 \approx \mathbf{192.907}
\end{split}
\end{equation}

\subsection{Derivation A: The Partition Formula}
The Weak Mixing Angle is defined as the ratio of the \textbf{Temporal Resonance} (the bare frequency $\Delta$) to the \textbf{Total System Capacity} ($N_{sys}$).

\begin{equation}
\sin^2 \theta_W = \frac{\text{Temporal Resonance}}{\text{Total Capacity}} = \frac{\Delta}{N_{sys}}
\end{equation}

\textbf{Physical Interpretation:} This ratio ($\approx 22\%$) represents the fraction of vacuum bandwidth allocated to temporal coherence ($\Delta$), with the remaining $78\%$ allocated to spatial structure and gauge operations. This explains why electromagnetism (the $U(1)$ sector coupling to temporal resonance) remains massless, while the weak force (the $SU(2)$ sector coupling to spatial structure) requires energy to deform the manifold.

\textbf{Numerical Calculation:}
Substituting the invariants:
\begin{equation}
\sin^2 \theta_W = \frac{43}{192.907} \approx \mathbf{\WeakAngleVal}
\end{equation}

\subsection{Implications for Grand Unification}
Many Grand Unified Theories (GUTs) predict $\sin^2\theta_W(M_Z) \approx 0.21$, inconsistent with precision measurements ($\approx 0.223$). The E8-Persistence framework resolves this: $\sin^2\theta_W$ is a \textbf{fixed geometric ratio} (~$43/193$), not a running coupling determined by energy scale. The apparent ``non-unification'' of gauge couplings is evidence that forces arise from distinct geometric features ($\Delta$ vs. $D\Delta$ vs. $\sigma$) rather than a single broken symmetry group.

\subsection{Physical Interpretation: Electroweak Structure and the Origin of Mass}

In the Standard Model, the weak mixing angle relates the gauge couplings of the $U(1)_Y$ hypercharge ($g'$) and the $SU(2)_L$ weak isospin ($g$) via the relation:
\begin{equation}
\sin^2\theta_W = \frac{g'^2}{g^2 + g'^2}
\end{equation}

The $E_8$-Persistence framework establishes a structural isomorphism between these couplings and the lattice geometry, mapping the arbitrary coupling ratio of QFT directly to the geometric bandwidth ratio of the substrate:
\begin{equation}
\frac{g'^2}{g^2 + g'^2} \longleftrightarrow \frac{\Delta}{D\Delta + \nu\eta + \sigma}
\end{equation}

This geometric partition ($\approx 22\%$ Temporal vs. $78\%$ Spatial) dictates the phenomenology of the gauge bosons:

\begin{itemize}
    \item \textbf{The Photon ($\gamma$, $U(1)$ Temporal):} The hypercharge sector ($g'$) couples to the \textbf{Fundamental Resonance} ($\Delta$), representing the causal update rate of the lattice. Because temporal evolution cannot be obstructed without violating causality, the photon must remain massless and propagate at $c$.
    
    \item \textbf{The W/Z Bosons ($W^{\pm}, Z^0$, $SU(2)$ Spatial):} The isospin sector ($g$) couples to the \textbf{Manifold Embedding} ($N_{sys} - \Delta$), representing the spatial infrastructure containing the resonance. To propagate through this structural density, these gauge fields must deform the lattice geometry, incurring an impedance cost that manifests as mass.
\end{itemize}

The Higgs field (System V) dynamically regulates this partition. The mixing angle $\sin^2\theta_W = 43/193$ determines the orientation of spontaneous symmetry breaking, ensuring the vacuum minimizes entropic action by preserving the temporal channel (photon) while confining the spatial channels ($W^{\pm}, Z$) below the electroweak scale ($v \approx 246$ GeV). This rigid geometric ratio provides a parameter-free prediction for the W-boson mass, testable against precision electroweak measurements.

\subsection{Validation: The Precision Electroweak Vector}

We test this geometric derivation against the most precise experimental constraints available.

\begin{itemize}
    \item \textbf{Geometric Prediction:} \WeakAngleVal
    \item \textbf{Experimental (Direct Mass)}: $0.22291 \pm 0.00011$ \cite{navas_review_2024}
    \item \textbf{Experimental (Global Fit)}: $0.22354 \pm 0.00006$
    \textbf{Result:} The geometric prediction ($M_W = $ \WBosonMassVal$ $ GeV) lies between $0.9\sigma$. The slight deviation from the Global Fit value reflects the difference between on-shell (mass-based) and $\overline{MS}$ (running coupling) renormalization schemes.
\end{itemize}

\subsection{Validation: Resolution of the W Boson Mass Tension}

The most stringent test of this derivation is the mass of the W Boson. The Standard Model prediction for $M_W$ is currently in tension with precise measurements.

\textbf{The Anomaly:}
\begin{itemize}
    \item \textbf{Standard Model:} $80.357 \pm 0.006$ GeV.
    \item \textbf{CDF II (2022):} $80.4335 \pm 0.0094$ GeV ($7\sigma$ tension).
    \item \textbf{ATLAS (2023):} $80.360 \pm 0.016$ GeV (SM-consistent).
\end{itemize}

\textbf{The Geometric Resolution:}
Using the derived angle $\sin^2 \theta_W = 43/193$ and the experimental Z-boson mass ($M_Z = 91.1876$ GeV), we calculate the W mass geometrically:

\begin{equation}
M_W = M_Z \sqrt{1 - \sin^2 \theta_W} = 91.1876 \times \sqrt{1 - \frac{43}{192.907}}
\end{equation}

\begin{equation}
M_W \approx 91.1876 \times \sqrt{0.7771} \approx \mathbf{\WBosonMassVal \text{ GeV}}
\end{equation}

\textbf{Result:} The geometric prediction ($M_W = \WBosonMassVal$ GeV) lies between the Standard Model ($80.357$ GeV) and CDF II ($80.434$ GeV), validating the \textbf{direction} of the CDF anomaly (favoring a heavier W-boson) while remaining consistent with ATLAS ($80.360$ GeV) within $1.6\sigma$. The lattice geometry predicts a heavier W than the SM global fit through a rigid structural partition ($43/193$) rather than parameter adjustment, suggesting the SM value may be systematically underestimated.

\subsection{Recursive Validation: The Cost of Time}
The structural consistency of the theory is confirmed by linking this partition back to the Fine-Structure Constant derived in System II.

Recall the \textbf{Temporal Cost ($T$)} derived in Eq. \ref{eq:alpha_inverse}, representing the entropic cost of a state transition:
$$ T_{geo} \approx 1.185 \times 10^{-5} $$

We now observe that this cost is exactly the second-order electromagnetic coupling ($\alpha^2$) modulated by the Weak Mixing Angle we just derived:
\begin{equation}
T_{check} = \alpha^2 \sin^2 \theta_W = \left(\frac{1}{\AlphaInvVal}\right)^2 \cdot \left(\frac{43}{192.907}\right)
\end{equation}
\begin{equation}
T_{check} \approx (5.325 \times 10^{-5}) \cdot (0.2228) \approx \mathbf{1.186 \times 10^{-5}}
\end{equation}

\textbf{Conclusion:} The match (within 0.1\%) confirms internal consistency. The Temporal Cost $T$ is the geometric toll paid for Weak Interaction (time evolution). This structural connection extends to flavor mixing, where the Cabibbo Angle and Weak Angle satisfy the Gatto-Sartori-Tonin relation (Paper III).