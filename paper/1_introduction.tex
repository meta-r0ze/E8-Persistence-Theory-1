\section{Introduction}

The Standard Model of particle physics presents a profound paradox. While it predicts interaction cross-sections with unprecedented precision, it relies on over 20 fundamental constants that are mathematically descriptive rather than predictive. They appear as arbitrary tuning parameters, empirically determined inputs rather than derived outputs.

We propose that these constants are not arbitrary, but are the \textbf{Geometric Impedances} of the vacuum itself, the unique structural solutions to a sequence of entropic constraints.

This work frames physics as a sub-discipline of \textbf{Informational Energetics} (IE). We reverse the standard order of model building. Instead of assuming a gauge group and fitting parameters, we apply the \textbf{Persistence Principle}: at each structural decision point, the system must select the configuration that minimizes Entropic Action while satisfying constraints of unitarity, causality, and solvency.

This is not a search over candidates. It is a deterministic descent through a decision tree with singular solutions at each node. Each step is not a hypothesis, it is the unique persistent solution when all alternatives are eliminated by entropic constraints. The Standard Model emerges not because we selected it, but because no other structure can maintain coherence against informational decay.

\textbf{This differs fundamentally from numerology.} Informational Energetics derives structure from constraints, not coincidences. Because our input set $\mathbb{S}$ consists exclusively of integers defined by mathematical theorems (e.g., Kneser, Heegner), the theory possesses \textbf{zero degrees of freedom} for tuning. The resulting agreement with experiment is a test of the structural logic, not a fit.

This proceeds as a \textbf{Thermodynamic Cascade}. We demonstrate that the Standard Model is the \textbf{Gibbs State} of the $E_8$ lattice—the unique vacuum configuration that maintains information conservation against the entropic flux of the substrate.

To validate this, we utilize the Standard Model not as a paradigm to be replaced, but as a blind test. If our substrate derivation is correct, known physics must emerge without adjustment; any free parameter would indicate structural error. This framework serves as the \textbf{Geometric Initialization} of Quantum Field Theory: while standard QFT treats couplings as inputs, we derive them as the sole surviving solutions to the entropic load of information propagation.

We achieve four objectives in this work:

\begin{enumerate}
    \item \textbf{The Geometric Control Architecture:} We derive the dimensionless coupling constants ($\alpha_s, \sin^2\theta_W, \theta_C$) not as arbitrary inputs, but as \textbf{Geometric Partition Coefficients}. These ratios represent the unique allocation of the finite lattice capacity ($\nu=16$) across orthogonal gauge sectors, \textbf{and their underlying geometry provides a structural resolution to the Strong CP problem}.

    \item \textbf{The Surface Regulator:} We identify the Higgs mechanism as the \textbf{Surface Regulator} of the lattice. We derive the Vacuum Expectation Value ($v$) and Higgs Mass ($m_H$) as the necessary impedance matching conditions required to couple the high-frequency lattice resonance to the weak interaction aperture.

    \item \textbf{The Bulk Regulator:} We identify Gravity as the \textbf{Bulk Regulator}, a nested system that stabilizes the lattice volume. The gravitational coupling emerges from the geometric attenuation of signals propagating from the lattice centroid, deriving the Planck Scale ($M_P$) and resolving the Hierarchy Problem as a function of lattice depth.
\end{enumerate}

Crucially, this derivation contains \textbf{zero free parameters}. Every output flows directly from the five geometric integers.

\subsection{Structure of the \texorpdfstring{$E_8$}{E8}-Persistence Theory Series}
This paper is part of a series that serves as a rigorous test of applying IE in the domain of physics. Each claim is developed with explicit derivations and falsification criteria.
