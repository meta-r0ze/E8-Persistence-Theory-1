\crefalias{section}{appendix}
\section{Transfer-Matrix Calculation of the Beta Function}
\label{app:beta_function}

To validate the dynamic behavior of the vacuum, we calculate the scaling evolution of the couplings using the \textbf{Transfer Matrix Method}. We treat the 4D spacetime manifold as a Tensor Network constructed from the contraction of the $E_8$ lattice unit cells.

The partition function $Z$ is expressed as the trace of the Transfer Matrix operator $\hat{T}$ raised to the power of the lattice depth $N$:
\begin{equation}
Z = \text{Tr}(\hat{T}^N) = \sum_i \lambda_i^N
\end{equation}

\subsection{Explicit Construction of \texorpdfstring{$\hat{T}(\beta)$}{TB}}
The transfer matrix acts on the configuration space of a single lattice layer. Let $|i\rangle$ denote a configuration state defined by the tuple of geometric quantum numbers $(D_i, \Delta_i, \nu_i, \sigma_i, \chi_i)$.

The matrix elements $T_{ij}$ describing the transition amplitude between adjacent layers are given by the Boltzmann weights of the potentials defined in Section V.A:
\begin{equation}
T_{ij}(\beta) = \exp\left(-\beta \sum_{k \in \{U, C, S, \sigma\}} V_k(i,j)\right) \cdot \delta_{ij}
\end{equation}
The Kronecker delta $\delta_{ij}$ arises because the geometric invariants are conserved quantities of the vacuum ground state (the ``ice rule").

\begin{itemize}
    \item \textbf{Ground State:} For the unique configuration satisfying all filters ($|\mathbb{S}\rangle = |4, 43, 16, 5, 2\rangle$), the potential sum is zero ($\sum V_k = 0$). Thus, the diagonal element is $T_{00} = 1$.
    \item \textbf{Excited States:} For any configuration violating a constraint, $\sum V_k > 0$. The element scales as $T_{ii} \to 0$ in the limit $\beta \to \infty$.
\end{itemize}
Consequently, the matrix possesses a non-degenerate maximal eigenvalue $\lambda_0 = 1$, corresponding to the stable vacuum, separated from the first excited state by a gap $\Delta_{gap} \sim e^{-\beta E_{gap}}$.

\subsection{Eigenvalues of the Color Sector}
Within the ground state sector, we analyze the sub-structure of the color interactions. The QCD Beta function coefficient $\beta_0$ emerges from the eigenvalues of the geometric operators acting on the $\sigma-\chi$ subspace.

\subsection{The Rigidity Eigenvalue (The ``11")}
The Rigidity Operator $\hat{R}$ describes the resistance of the vacuum geometry to deformation. It is the trace of the stress tensor defined by the Spacetime Anchor and Interaction Pressure:
\begin{equation}
\lambda_R = \text{Tr}(\hat{R}) = D\chi + (\sigma - \chi) = 8 + 3 = \mathbf{11}
\end{equation}

\subsection{The Screening Eigenvalue (The ``2/3")}
The Screening Operator $\hat{S}$ describes the partitioning of the topological boundary charge ($\chi=2$) across the generation manifold ($n_{gen}=3$).
\begin{equation}
\lambda_S = \frac{\chi}{n_{gen}} = \mathbf{\frac{2}{3}}
\end{equation}

\textbf{Result:} The lattice $\beta$-function is identified as the difference between the Rigidity and Screening eigenvalues:
\begin{equation}
\beta_{lat} = \lambda_R - \lambda_S \cdot n_f = 11 - \frac{2}{3}n_f
\end{equation}

\subsection{Correlation Length and the Loop Expansion}
We calculate the correlation length $\xi(\beta)$ to map the lattice statistics to the perturbative expansion of QED. The second moment of the spin-spin correlator is given by:
\begin{equation}
\xi(\beta) = \sqrt{\frac{\sum x^2 \langle O(x)O(0) \rangle}{\sum \langle O(x)O(0) \rangle}}
\end{equation}

In the strong persistence limit, the correlation length expands as:
\begin{equation}
\xi(\beta) = c \Delta + \frac{\gamma}{\beta} + O(1/\beta^2)
\end{equation}
Identifying the inverse persistence $\beta^{-1}$ with the running coupling $\alpha$, this form reproduces the structure of the one-loop renormalization group equation.

\subsection{Matching to the \texorpdfstring{$\overline{MS}$}{MS} Scheme:}
The geometric derivation in Section VI.E identified the screening factor as $\gamma = 1/(3\pi)$. This value matches the standard QED one-loop coefficient when the lattice spacing $a \sim 1/\Delta$ is matched to the renormalization scale $\mu$ of the $\overline{MS}$ scheme.

\textit{Note:} A complete ab initio derivation of the factor $1/(3\pi)$ from the $E_8$ correlator requires evaluating the sum $\sum \langle O(x)O(0) \rangle$ explicitly on the lattice geometry. This calculation is deferred to future numerical work. The present result demonstrates structural consistency: the lattice expansion reproduces the functional form of the beta function, with coefficients that map to the geometric invariants of the substrate.

\textbf{Result:} The lattice $\beta$-function is identified as the difference between the Rigidity and Screening eigenvalues:
\begin{equation}
\beta_{lat} = \lambda_R - \lambda_S \cdot n_f = 11 - \frac{2}{3}n_f
\end{equation}

\subsection{Significance of the Structural Match}
The fact that the lattice eigenvalues ($11$, $2/3$) coincide exactly with the QCD beta function coefficients—which in standard field theory emerge from intricate loop integrals involving Casimir invariants—is non-trivial. In this framework, they arise from elementary counting: dimensions times boundary ($D\chi = 8$), symmetry remainder ($\sigma - \chi = 3$), and topological partition ($\chi/n_{gen} = 2/3$). The ab initio correlator calculation would confirm whether this is coincidence or identity.
