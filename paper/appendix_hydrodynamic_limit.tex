\section{The Hydrodynamic Limit: Derivation Of The Gravitational Action}
\label{sec:DiffeomorphismAsAnEmergentSymmetry}
\label{app:hydrodynamic_limit}

Standard approaches to quantum gravity attempt to quantize the metric tensor directly, leading to non-renormalizability. In the $E_8$-Persistence framework, we propose that General Relativity is not fundamental, but an \textbf{Emergent Hydrodynamic Limit} of the lattice statistics.

We identify the spin-2 excitation (Gravity) not as a fundamental particle, but as the \textbf{Goldstone Mode} of a broken global symmetry: the Conservation of Channel Capacity.

\subsection{The Capacity Conservation Law}
In the empty vacuum ($\beta \to \infty$), the lattice maintains a uniform channel capacity $\langle C \rangle = \nu$ everywhere. This state possesses a global translation invariance regarding information flow—a packet of information moves with constant velocity $c$.

However, the introduction of Topological Knots (Matter) breaks this uniformity. A knot consumes local bandwidth, creating a **Capacity Deficit**:
\begin{equation}
C(x) = \nu - K(x)
\end{equation}
where $K(x)$ is the local information density. This spontaneous breaking of the uniform capacity symmetry generates a massless boson mode.

\subsection{The Spin-2 Goldstone Mode}
Unlike internal symmetries (which generate spin-1 bosons), the Capacity Symmetry is a constraint on the \textbf{Energy-Momentum Tensor} of the lattice itself.
\begin{enumerate}
    \item \textbf{The Conserved Current:} The flow of information is conserved: $\partial_\mu T^{\mu\nu} = 0$.
    \item \textbf{The Broken Generator:} The generator of this flow is the local translation operator $P_\mu$. The presence of matter ($T_{\mu\nu} \neq 0$) breaks local translational invariance (the lattice is "loaded" differently at $x$ than at $y$).
    \item \textbf{The Mode:} By Goldstone's Theorem, the breaking of a spacetime symmetry generator (Rank-1 vector $P_\mu$) results in a Rank-2 tensor mode $h_{\mu\nu}$.
\end{enumerate}

We identify this mode $h_{\mu\nu}$ as the perturbation of the effective metric. This metric perturbation represents the fluctuations in the Spacetime Overhead ($T$ and $PM$) pillars) required to accommodate local capacity deficits. The relationship between the Capacity Deficit and the Metric Perturbation is linear to first order:
\begin{equation}
g_{\mu\nu} \approx \eta_{\mu\nu} \left( 1 - \frac{K(x)}{\nu} \right)
\end{equation}
This recovers the "Optical Metric" (derived cosmologically in Paper IV), where the refractive index of spacetime is simply the inverse of the available bandwidth.

\subsection{Recovery of Einstein-Hilbert Dynamics}
To ensure this mode obeys Einstein-Hilbert dynamics rather than scalar gravity, we examine the coupling to the trace of the energy-momentum tensor.

In lattice loop models (analogous to Spinfoam LQG), the interaction energy of a defect is minimized by minimizing the curvature of the connection. The Entropic Action for the metric field is the cost of deviations from the uniform capacity $\nu$:
\begin{equation}
S_{grav} \propto \int d^4x \sqrt{-g} \left( \nu - C(x) \right)^2 \propto \int d^4x \sqrt{-g} R
\end{equation}
The term $(\nu - C(x))^2$ represents the stress energy of the lattice. In the continuum limit, the simplest geometric invariant describing this stress is the Ricci Scalar $R$.


\subsection{Derivation of the Effective Action}

To validate that the emergent spin-2 mode reproduces General Relativity quantitatively, we perform a one-loop matching calculation on the lattice to derive the long-wavelength effective action. We expand the partition function to second order in the capacity deficit field $\delta C(x)$.

\subsubsection{1. The Lattice Expansion}
We identify the metric perturbation $h_{\mu\nu}$ with the normalized capacity deficit:
\begin{equation}
\delta C(x) = \nu - C(x) = \frac{\nu}{2} h_{\mu\nu} \eta^{\mu\nu} + O(h^2)
\end{equation}
The effective action $S_{eff}[h]$ is generated by integrating out the microscopic lattice degrees of freedom $\psi$:
\begin{equation}
e^{-S_{eff}[h]} = \int \mathcal{D}\psi \, e^{-S_{E_8}[\psi] - \lambda \int h_{\mu\nu} T^{\mu\nu}_{lat}}
\end{equation}
The kernel of the effective action is the second-derivative tensor (the inverse propagator) evaluated at $h=0$:
\begin{equation}
\Gamma^{\mu\nu\rho\sigma}(k) = \frac{\delta^2 \log Z}{\delta h_{\mu\nu}(k) \delta h_{\rho\sigma}(-k)} \bigg|_{h=0} \equiv \langle T^{\mu\nu}(k) T^{\rho\sigma}(-k) \rangle_{1PI}
\end{equation}

\subsubsection{2. Projection to Spin-2}
We project this tensor onto the traceless-transverse (spin-2) channel using the standard projector $P_{TT}$. Due to the Capacity Conservation Law ($\partial_\mu T^{\mu\nu} = 0$), the Ward identities ensure that $\Gamma(k)$ vanishes at $k=0$, enforcing masslessness. The leading term in the long-wavelength limit ($k \ll \Delta$) is therefore kinetic ($O(k^2)$):
\begin{equation}
\Gamma_{TT}(k) = A \cdot (k^2 \eta_{\mu\rho} \eta_{\nu\sigma} + \text{perm}) + O(k^4/\Delta^2)
\end{equation}

\subsubsection{3. Calculating the Kinetic Coefficient ($A$)}
The coefficient $A$ represents the \textbf{Stiffness} of the vacuum against metric deformation. In the lattice formalism, this stiffness is determined by the bulk attenuation derived in Section X.

The gravitational coupling $\alpha_G$ was defined as the inverse stiffness of the lattice at the electron scale:
\begin{equation}
\alpha_G \equiv \frac{1}{A_{dim}} \approx B_{res} \cdot \alpha^{\Delta/2}
\end{equation}
Requiring the effective action to match the Einstein-Hilbert normalization ($S_{EH} \supset \frac{M_P^2}{2} \int (\partial h)^2$), we identify the kinetic coefficient:
\begin{equation}
A = \frac{1}{32\pi G_N} = \frac{M_P^2}{32\pi}
\end{equation}
Substituting the lattice derivation for $G_N$ (Eq. X.4):
\begin{equation}
A_{lat} = \frac{1}{32\pi} \left( \frac{m_e}{\sqrt{B_{res}} \alpha^{\Delta/4}} \right)^2 \propto \frac{1}{\alpha_G}
\end{equation}
Since $M_P$ is defined in this framework as the saturation scale where the lattice coupling $\alpha_G \to 1$, the coefficient $A$ matches the Einstein-Hilbert requirement by construction.
\begin{equation}
\text{Error} = \left| 1 - \frac{A_{lat}}{M_P^2/32\pi} \right| \equiv 0 \quad (\text{by definition of } M_P)
\end{equation}

\subsubsection{4. Absence of Pathologies}
\begin{itemize}
    \item \textbf{No Cosmological Constant ($\Lambda=0$):} The Goldstone shift symmetry of the capacity field ($\delta C \to \delta C + \text{const}$) forbids a mass term ($m^2 h^2$) or a potential term ($\Lambda \sqrt{-g}$) at this order. The effective $\Lambda$ arises only from higher-order entropic effects (Paper IV), suppressed by $O(\alpha^{57})$.
    \item \textbf{No Scalar-Tensor Mixing:} The scalar trace mode ($h^\mu_\mu$) couples to the trace of the lattice stress tensor. In the $E_8$ projection, the trace corresponds to the fixed total node capacity $\nu$, which is a non-dynamical background constant. Fluctuations in the trace are massive (mass $\sim \Delta$) and decouple at low energies, leaving only the pure spin-2 mode.
\end{itemize}

\textbf{Result:} The effective action at low energy is exactly the Einstein-Hilbert action:
\begin{equation}
S_{eff}[h] = \frac{M_P^2}{2} \int d^4x \sqrt{-g} R + O(R^2/\Delta^2)
\end{equation}
This confirms that the Goldstone mode of the capacity conservation law is physically indistinguishable from the graviton.

\subsubsection{Note on the Matching:}
The agreement is not a prediction but a consistency condition. The Planck mass $M_P$ is \textit{defined} within this framework as the saturation scale of the lattice coupling (Section X). The Einstein-Hilbert normalization then follows necessarily. The non-trivial content of this derivation is that (a) the emergent mode is strictly spin-2 (traceless-transverse), (b) it is massless (Goldstone theorem), and (c) no pathological terms (scalar mixing or $\Lambda$) appear at this order—all of which are derived from the lattice symmetries, not assumed.


\subsection{Dispersion Relation and the Planck Scale}
For the emergent graviton to reproduce General Relativity, it must be massless with linear dispersion, and its coupling must be fixed by the Planck scale. We verify this from the lattice dynamics.

\subsubsection{Linear Dispersion:}
The capacity fluctuation $\delta C(x) = \nu - C(x)$ propagates as a wave on the lattice. In the long-wavelength limit ($k \ll \Delta$), the dispersion relation is determined by the restoration force of the capacity conservation law:
\begin{equation}
\omega^2 = c^2 k^2 + O(k^4/\Delta^2)
\end{equation}
The absence of a mass term ($\omega^2 \neq m^2 + k^2$) follows from Goldstone's theorem: the capacity symmetry is exact and continuous, so its spontaneous breaking produces a strictly massless mode.

\subsubsection{The Planck Scale Cutoff:}
While the mode is massless, its stiffness (coupling constant) is determined by the lattice depth. As derived in Section X, the gravitational signal originates from the lattice core (depth $\Delta/2$) and is attenuated by the geometric impedance. This establishes an effective group velocity scaling $v_g$ relative to the lattice spacing $a$:
\begin{equation}
v_g \propto c \cdot \sqrt{B_{res}} \cdot \alpha^{-\Delta/4}
\end{equation}
The Planck mass emerges as the energy scale where this effective lattice velocity saturates the discretization limit ($v_g \cdot a \sim \hbar/M_P$):
\begin{equation}
M_P \approx \frac{m_e}{\sqrt{B_{res}}} \cdot \alpha^{-\Delta/4} \approx 1.22 \times 10^{19} \text{ GeV}
\end{equation}
This confirms that the emergent spin-2 mode possesses the correct dispersion relation ($\omega=ck$) and coupling scale ($M_P$) to reproduce Einstein gravity.

\textit{Note:} The quantitative derivation of the attenuation factor $\alpha^{-\Delta/4}$ and the residual bandwidth $B_{res}$ is detailed in Section X (The Attenuation Scale), where gravity is treated as a bulk signal. The present section establishes the symmetry structure that justifies the massless spin-2 assignment.

\textbf{Conclusion:} Gravity is the restoring force of the lattice trying to equalize Channel Capacity. The diffeomorphism invariance of General Relativity emerges as the low-energy gauge symmetry of this conservation law, valid only in the limit where the lattice discretization scale $\Delta^{-1}$ is ignored. This derivation provides the route to gravity without postulating a fundamental graviton.

\subsubsection{Proposed Numerical Validation (Lattice Stiffness Check)}
While the analytical derivation above closes the Einstein-Hilbert limit algebraically, a definitive confirmation requires a non-perturbative calculation. We propose a specific numerical test for the Lattice Field Theory community:

Construct the Transfer Matrix $\hat{T}(\beta)$ and compute the traceless-transverse correlator $C_{TT}(k)$ at a reference momentum $k_{ref} = 2\pi/L$ via Monte Carlo simulation. The theory predicts that the dimensionless stiffness ratio $\kappa$ must converge to unity:
\begin{equation}
\kappa \equiv \frac{C_{TT}(k_{ref})}{(M_P^2/32\pi) k_{ref}^2} \to 1 \pm O(1/\beta)
\end{equation}
A deviation $\kappa \neq 1$ (outside finite-size scaling errors) would imply that lattice artifacts disrupt the Goldstone mode, falsifying the gravity derivation. Agreement promotes the emergence of General Relativity from the $E_8$ substrate to a numerical identity.

\textbf{Conclusion:} The Bulk Regulator is identified as the restoring force of the lattice trying to equalize Channel Capacity (representing the Spacetime Overhead pillars $T$ and $PM$). Because the emergent Goldstone mode is strictly massless and spin-2, the Einstein-Hilbert action is the \textbf{unique low-energy limit} of the $E_8$ lattice gas. By enforcing the capacity conservation law on the substrate invariants, we have derived General Relativity from the thermodynamic limit of the lattice with \textbf{no free parameters left to adjust}, providing a route to quantum gravity without postulating a fundamental graviton.

