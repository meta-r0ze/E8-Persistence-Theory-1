\section{The Partition Ratio: Weak Mixing Angle (\texorpdfstring{$\sin^2 \theta_W$}{sin2thetaW})} \label{sec:Partition_Ratio}

\textbf{The Standard Model Ansatz:} The Weak Mixing Angle is a free parameter defining the rotation between the $U(1)_Y$ and $SU(2)_L$ sectors. It determines the partition of the unified electroweak force into the electromagnetic and weak components. Standard physics relies on the Weak Mixing Angle as an empirical input ($s^2_W \approx 0.223$) to satisfy the mass ratio $M_W/M_Z$. However, it offers no geometric origin for this specific magnitude, nor a mechanism to resolve the experimental tension between the value derived from direct mass measurements ($0.22291 \pm 0.00011$) \cite{PhysRevD.110.030001} and the value derived from global electroweak fits ($0.22354 \pm 0.00006$) \cite{PhysRevD.110.030001}.

\textbf{The $E_8$-Persistence Derivation:} The Weak Mixing Angle is derived as the \textbf{Resonance Partition Fraction}. To minimize Entropic Action, the lattice must efficiently route information between the Resonant Core (Lattice, the Signal) and the Geometric Infrastructure (Projection, the Carrier).

\subsection{The Allocation Principle}
The mixing angle represents the fraction of the total coordination budget ($D\Delta + \nu + \sigma$) that is dedicated to the active resonance ($\Delta$).

\begin{enumerate}
    \item \textbf{The Signal ($\Delta$):} The Heegner Resonance ($\Delta = 43$). This represents the fundamental frequency of the lattice. The Weak interaction, being the force of transmutation and decay, couples directly to this resonant heartbeat of the vacuum.
    \item \textbf{The Infrastructure:} The total geometric overhead required to support that signal.
    \begin{itemize}
        \item \textbf{Manifold Projection ($D\Delta$):} The projection of the resonance onto the 4D spacetime manifold ($4 \times 43 = 172$).
        \item \textbf{Chiral Channels ($\nu$):} The available channel capacity for fermion information ($\nu = 16$).
        \item \textbf{Interaction Order ($\sigma$):} The symmetry overhead required for unification ($\sigma = 5$).
    \end{itemize}

\end{enumerate}

The Persistence Principle requires that the electroweak force partitions itself according to this structural ratio to maintain impedance matching:

\begin{equation}
\sin^2 \theta_W = \frac{\text{Active Resonance}}{\text{Total Infrastructure}} = \frac{\Delta}{D\Delta + \nu + \sigma}
\end{equation}

\subsection{Numerical Result}
\begin{equation}
\sin^2 \theta_W = \frac{43}{4(43) + 16 + 5} = \frac{43}{193} \approx \mathbf{\ExecuteMetaData[src/results.tex]{WeakAngleVal}}
\end{equation}

\begin{itemize}
    \item \textbf{Geometric Prediction:} \ExecuteMetaData[src/results.tex]{WeakAngleVal}
    \item \textbf{Experimental (Direct Mass)}: $0.22291 \pm 0.00011$ \cite{PhysRevD.110.030001}
    \item \textbf{Experimental (Global Fit)}: $0.22354 \pm 0.00006$
    \item \textbf{Experimental ($\overline{MS}$ Running)}: $0.23122$

    \item \textbf{Experimental Value (On-Shell):} $\approx 0.223$ (Depends on renormalization scheme)
    \item \textbf{Precision:} Matches the on-shell definition to within \textbf{0.05\%}.
\end{itemize}

\textbf{Physical Interpretation:} The tension between the On-Shell value ($\approx 0.223$) and the $\overline{MS}$ value ($\approx 0.231$) is resolved. The lattice invariants define the \textbf{Bare Geometric Partition} (On-Shell), while the effective running includes dynamic loops.



\subsection{Recursive Validation: The Cost of Time}
The geometric validity of the Weak Mixing Angle is reinforced by a structural link back to the Vacuum Impedance derived in \cref{sec:Persistence_Condition}..

Recall the \textbf{Temporal Tax} ($T$), the metabolic cost of a state transition defined purely by lattice integers in Eq. \ref{eq:alpha_inverse}:
$$ T_{geo} \approx 1.185 \times 10^{-5} $$

We now observe that this tax corresponds to the second-order electromagnetic coupling ($\alpha^2$) modulated by the Weak Mixing Angle partition we just derived ($\sin^2 \theta_W = 43/193$).

\begin{equation}
T_{check} = \alpha^2 \sin^2 \theta_W
\end{equation}

Substituting the strictly derived values from this framework:
$$ T_{check} = \left(\frac{1}{\ExecuteMetaData[src/results.tex]{AlphaInvVal}}\right)^2 \cdot \left(\frac{43}{193}\right) $$
$$ T_{check} \approx (5.325 \times 10^{-5}) \cdot (0.2228) \approx \mathbf{1.186 \times 10^{-5}} $$

\textbf{Conclusion:} The match (within 0.1\%) confirms internal consistency. The "Time Tax" $T$ appearing in the Fine-Structure Constant is not a random correction; it is the specific geometric cost of authorizing a Weak Interaction. The Weak Force is the mechanism of Time (state transition), and $\alpha^2 \sin^2 \theta_W$ is the toll paid to the vacuum to execute it.

\textbf{Conclusion:} The integer derivation ($43/193$) correctly identifies the On-Shell angle derived from the physical W-boson mass (hitting the lower bound of the $1\sigma$ range), distinguishing it from the continuous field geometry ($1/2\sqrt{5} \approx 0.2236$) detected in global fits.

\subsection{Physical Interpretation}
The derivation matches the \textbf{On-Shell definition} ($s^2_W \equiv 1 - M_W^2/M_Z^2$) to within 0.05\%. This identification is structurally required: because the $W$ and $Z$ bosons are fundamental lattice resonances, their mass ratio is fixed by the static integer invariants ($\Delta, \sigma$). The higher values observed in effective schemes (like $\overline{MS} \approx 0.231$) include dynamic vacuum polarization loops ("running"). The lattice invariants define the \textbf{Bare Geometric Ratio} at the pole mass, which corresponds physically to the On-Shell scheme.