\section{The Vacuum Regulator: The Higgs Sector (\texorpdfstring{$v, \lambda, \mu^2, m_H, y_e$}{vlambdamu2mhye})} 
\label{sec:Vacuum_Regulator}
\CatchFileBetweenTags{\AlphaInvVal}{calculations/constants.tex}{AlphaInvVal}
\CatchFileBetweenTags{\MeMeVPrint}{calculations/constants.tex}{MeMeVPrint}

% Delta E
\CatchFileBetweenTags{\HiggsVEVVal}{calculations/constants.tex}{HiggsVEVVal}
\CatchFileBetweenTags{\HiggsVEVExperimentalValue}{calculations/constants.tex}{HiggsVEVExperimentalValue}
\CatchFileBetweenTags{\HiggsVEVAccText}{calculations/constants.tex}{HiggsVEVAccText}

\CatchFileBetweenTags{\FermiConstVal}{calculations/constants.tex}{FermiConstVal}
\CatchFileBetweenTags{\FermiConstExperimentalValue}{calculations/constants.tex}{FermiConstExperimentalValue}
\CatchFileBetweenTags{\FermiConstAccText}{calculations/constants.tex}{FermiConstAccText}

% MI
\CatchFileBetweenTags{\HiggsLambdaVal}{calculations/constants.tex}{HiggsLambdaVal}
\CatchFileBetweenTags{\HiggsLambdaExperimentalValue}{calculations/constants.tex}{HiggsLambdaExperimentalValue}
\CatchFileBetweenTags{\HiggsLambdaAccText}{calculations/constants.tex}{HiggsLambdaAccText}

% T
\CatchFileBetweenTags{\HiggsMassVal}{calculations/constants.tex}{HiggsMassVal}
\CatchFileBetweenTags{\HiggsMassExperimentalValue}{calculations/constants.tex}{HiggsMassExperimentalValue}
\CatchFileBetweenTags{\HiggsMassAccText}{calculations/constants.tex}{HiggsMassAccText}

% PM
\CatchFileBetweenTags{\ElectronYukawaVal}{calculations/constants.tex}{ElectronYukawaVal}
\CatchFileBetweenTags{\ElectronYukawaExperimentalValue}{calculations/constants.tex}{ElectronYukawaExperimentalValue}
\CatchFileBetweenTags{\ElectronYukawaAccText}{calculations/constants.tex}{ElectronYukawaAccText}

\textbf{The Standard Model Ansatz:} In the Standard Model, the electroweak sector is parameterized by independent inputs ($v, \lambda, \mu^2$) to generate the Higgs potential $V(\phi) = -\mu^2|\phi|^2 + \lambda|\phi|^4$. While effective, this offers no structural reason for the specific energy scale ($v \approx 246$ GeV) or the coupling strength ($\lambda \approx 0.13$).

\textbf{The $E_8$-Persistence Derivation:} We have established the Lattice Hardware (System I) and the Geometric Impedance (System III). However, a raw feed from the lattice resonance ($\Delta$) is too energetic to couple directly to matter. The universe requires a \textbf{Step-Down Transformer} to convert the high-frequency lattice potential into a stable mass scale.

We identify the Higgs Field not merely as a boson, but as a \textbf{Nested Persistent System} a fractal replica of the vacuum architecture designed to regulate the electroweak scale. It replicates the six pillars of Informational Energetics to create a stable ``energy vessel'' ($v$) within the larger lattice.


\subsection{Energy Vessel (\texorpdfstring{$\Delta E_H$}{dE}): The Vacuum Expectation Value (\texorpdfstring{$v$}{v})}
The VEV represents the capacity of the subsystem. Because the Electron ($\Delta^0$) is the unique Unitary Ground State ($N=0$) of the lattice, it acts as the fundamental \textbf{Mass Unit} against which the vacuum potential is normalized.

\subsubsection{Step 1: The Bare Geometric Floor (\texorpdfstring{$v_{geo}$}{vgeo})}
We first calculate the static potential minimum defined by the lattice invariants:
\begin{equation}
v_{geo} = (\chi \Delta^2 - I_s) \cdot \alpha^{-1} \cdot m_e 
\end{equation}

\noindent \textbf{Structural Overhead ($I_s$):}
$$ I_s = (\Delta \cdot D) + \nu = (43 \times 4) + 16 = \mathbf{188} $$

\noindent Substituting the invariants:

\begin{equation}
v_{geo} = (2 \cdot 43^2 - 188) \cdot \AlphaInvVal \cdot \MeMeVPrint \text{ MeV}
\end{equation}
\begin{equation}
v_{geo} \approx 245.789 \text{ GeV}
\end{equation}

\subsubsection{Step 2: Radiative Correction (Topological Screening)}
The field is screened by the electromagnetic topology. The screening medium consists of the spatial manifold ($D=4$) plus the topological boundary charge ($\chi=2$) distributed across the full spherical phase space of the gauge field ($4\pi$).
\begin{equation}
D_{eff} = D + \frac{\chi}{4\pi} \approx 4.15915
\end{equation}

\begin{equation}
v_{screened} = v_{geo} \left( 1 + \frac{\alpha}{D_{eff}} \right) \approx 246.2201 \text{ GeV}
\end{equation}

\subsubsection{Step 3: The Thermodynamic Noise Floor}
Finally, we account for the finite resolution of the lattice. As derived in System II, the vacuum possesses a \textbf{Persistence Margin} ($PM$) representing the minimum fluctuation amplitude. This noise reduces the effective depth of the potential well.
Because the vacuum stability floor supports $n_{gen}=3$ generations ($\sigma - \chi = 3$), the noise is partitioned linearly across the generation manifold.

\begin{equation}
v_{phys} = v_{screened} \left( 1 - \frac{PM}{3} \right)
\end{equation}

\textbf{Calculation:}
Substituting $PM \approx 2.91 \times 10^{-6}$:
\begin{equation}
v_{phys} = 246.2201 \text{ GeV} \times (1 - 9.7 \times 10^{-7}) \approx \textbf{246.219876} \text{ GeV}
\end{equation}

\begin{itemize}
    \item \textbf{Geometric Prediction:} $\HiggsVEVVal$ GeV
    \item \textbf{Experimental Value:} \HiggsVEVExperimentalValue
    \item \textbf{Accuracy:} \HiggsVEVAccText
\end{itemize}

\subsubsection*{Derived Limit: The Fermi Constant (\texorpdfstring{$G_F$}{GF})}
$G_F$ is the inverse squared cross-section of this stability floor. The normalization factor $\sqrt{2}$ is identified as the square root of the Topological Boundary ($\chi=2$):
\begin{equation}
G_F = \frac{1}{\sqrt{\chi} v^2} \approx \mathbf{\FermiConstVal \text{ GeV}^{-2}}
\end{equation}
(\textbf{Accuracy:} \FermiConstAccText)



\subsection{Information Model (\texorpdfstring{$\Delta I_H$}{dI}): The Scalar Charge (\texorpdfstring{$Y$}{Y})}
The \textbf{Information Model} defines the identity signature of the system within the gauge group. For the Higgs field, this corresponds to its Hypercharge ($Y$).

As established in \textbf{Appendix \cref{sec:OriginOfHypercharg}}, quantum numbers in this framework are geometric ratios. The Higgs boson is the scalar excitation of the vacuum's Unitary Ground State ($\Delta^0 = 1$). Consequently, its Hypercharge is derived as the inverse of this ground state resonance:

\begin{equation}
Y_H = \frac{1}{\Delta^0} = 1
\end{equation}

This unitary charge ($Y=1$) combined with the topological boundary constraint ($\chi=2$) mandates the $SU(2)_L$ doublet structure ($\mathbf{2}$) required for the Information Model to interface with the Chiral Diode ($\nu=16$).





\subsection{Coordination Protocol (\texorpdfstring{$MI_H$}{MI}): Self-Coupling (\texorpdfstring{$\lambda$}{lambda})}
The self-coupling $\lambda$ represents the Coordination Protocol of the Higgs system. In Informational Energetics, coupling constants are bandwidth allocations. $\lambda$ determines what fraction of the total system capacity is reserved for the scalar field to maintain its own coherence (self-interaction).

\begin{equation}
\lambda = \frac{\text{Interaction Remainder} - \text{Resonant Tax}}{\text{System Capacity}} 
\end{equation}

We derive this coupling as the Net Available Bandwidth normalized by the Total Systemic Channel.

\subsubsection{1. The Net Available Bandwidth}
The bandwidth available for the scalar sector starts with the Interaction Remainder (the surplus symmetry capacity, $\sigma - \chi = 3$).

However, the Higgs field is not static; it is a resonant excitation oscillating at the lattice frequency $\Delta$. To maintain phase coherence across the fundamental time period, the system must pay a Resonant Tax of one unit of inverse-bandwidth ($1/\Delta$). This acts as the "synchronization cost" or "clock cycle overhead" of the regulator.

\begin{equation}
\text{Net Bandwidth} = (\sigma - \chi) - \frac{1}{\Delta} = 3 - \frac{1}{43} \approx 2.9767
\end{equation}

\subsubsection{2. The Systemic Channel (\texorpdfstring{$H_{sys}$}{Hsys})}
This net bandwidth is normalized by the total active degrees of freedom in the system ($H_{sys}$), representing the full pipe through which the coupling must act.
\begin{equation}
H_{sys} = \nu + \sigma + \chi = 16 + 5 + 2 = 23
\end{equation}

\subsubsection{3. The Coupling Derivation}
The self-coupling is the ratio of the net available bandwidth to the total channel capacity:

\begin{equation}
\lambda = \frac{(\sigma - \chi) - 1/\Delta}{H_{sys}}
\end{equation}

\begin{equation}
\lambda = \frac{3 - \frac{1}{43}}{23} = \frac{2.976744}{23} \approx \mathbf{\HiggsLambdaVal}
\end{equation}

\begin{itemize}
    \item \textbf{Experimental Value:} \HiggsLambdaExperimentalValue
    \item \textbf{Accuracy:} \HiggsLambdaAccText
\end{itemize}

\textbf{Physical Interpretation:} The Higgs self-coupling is not an arbitrary number. It is the specific fraction of vacuum bandwidth ($\approx 13\%$) remaining for self-regulation after paying the entropic tax for temporal synchronization ($1/\Delta$). The high precision of this derivation ($0.3\%$ vs $1.1\%$ without the tax) confirms that the Higgs is a dynamic, resonant system, distinct from static geometric apertures like the Cabibbo angle.








\subsection{Stabilizing Governor (\texorpdfstring{$G_H$}{G}): The Quartic Potential}
While $\lambda$ represents the Coordination Protocol (bandwidth allocation), the quartic term $\lambda|\phi|^4$ in the potential acts as the \textbf{Stabilizing Governor}. This geometric bounding potential prevents the field from diverging to infinity under the negative mass pressure.

In Informational Energetics, the Governor enforces the topological boundary constraint ($\chi=2$). For the Higgs, this manifests as the requirement that the potential possess exactly two stable minima (the "Mexican hat" structure):

\begin{equation}
V(\phi) = -\mu^2|\phi|^2 + \lambda|\phi|^4
\end{equation}

The derived value $\lambda \approx 0.129$ is the precise structural stiffness required to enforce this boundary condition against vacuum pressure, ensuring the potential stabilizes at the thermodynamic minimum $|\phi| = v/\sqrt{2}$.

Without the quartic term, the tachyonic mass ($-\mu^2$) would cause runaway condensation. The Governor caps this divergence, implementing the fundamental constraint that persistent systems must have finite capacity.



\subsection{Temporal Cost (\texorpdfstring{$T_H$}{T}): The Instability Factor (\texorpdfstring{$\mu^2$}{mu})}
The Higgs field requires a source of instability to drive Spontaneous Symmetry Breaking. In IE, this is the \textbf{Temporal Cost ($T$)}, the Entropic Action of maintaining a broken symmetry state (the "False Vacuum") distinct from the origin.

We identify the dimensionless instability factor as twice the self-coupling bandwidth:
\begin{equation}
T_H = 2\lambda \approx 0.259
\end{equation}

The factor of 2 arises because the Higgs doublet contains two independent complex fields (four real degrees of freedom), each contributing a $\lambda$ term to the vacuum instability rate. This recovers the Standard Model relation $\mu^2 = \lambda v^2$ from geometric first principles.

This solves the "Tachyonic Mass" problem: the negative mass parameter $-\mu^2$ is simply the manifestation of this tax acting on the VEV capacity:
\begin{equation}
\mu^2 = \lambda v^2 = \frac{T_H}{2} v^2 = \frac{2\lambda}{2} v^2 = \lambda v^2
\end{equation}

\subsection{Output: The Higgs Mass (\texorpdfstring{$m_H$}{mH})}
With the Capacity ($v$) and Protocol ($\lambda$) established, the mass of the scalar excitation is the closure of the geometric system:

\begin{equation}
m_H = \sqrt{2\lambda} v_{phys}
\end{equation}

\begin{equation}
m_H = \sqrt{2 (0.12942)} \cdot (\HiggsVEVVal \text{ GeV}) \approx \mathbf{\HiggsMassVal \text{ GeV}}
\end{equation}

\begin{itemize}
    \item \textbf{Experimental Value:} \HiggsMassExperimentalValue
    \item \textbf{Accuracy:} \HiggsMassAccText
\end{itemize}




\subsection{Persistence Margin (\texorpdfstring{$PM$}{PM}): The Electron Yukawa (\texorpdfstring{$y_e$}{ye})}

The final component of the vacuum architecture is the resolution floor. The Persistence Margin ($PM$) represents the smallest non-zero bit of mass the lattice can resolve against thermal noise. This determines the coupling of the lightest charged particle (the Electron).

We derive this scale as the Unit Bit (1) diluted over the Total Configuration Space Volume of the resonant system.

\begin{equation}
y_{e,bare} = \frac{1}{V_{config}} = \frac{1}{H_{full} \cdot (\text{Aperture}) \cdot (\text{Area})}
\end{equation}

\begin{itemize}
    \item \textbf{Operational Budget ($H_{full} = 31$):} The total number of degrees of freedom required to define a persistent state (\cref{eq:hfull}).
    \item \textbf{Weak Aperture ($\sigma + 1 = 6$):} The electron couples via the weak interaction symmetry ($\sigma=5$) plus the vacuum unit ($1$).
    \item \textbf{Resonant Area ($\Delta^2 = 43^2$):} The geometric area of the fundamental lattice resonance ($1849$ lattice sites).
\end{itemize}

\begin{equation}
y_{e,bare} = \frac{1}{31 \cdot 6 \cdot 43^2} = \frac{1}{343,914} \approx 2.9077 \times 10^{-6}
\end{equation}

\subsubsection{Radiative Correction}
The bare geometric value represents the coupling of the naked topological knot. However, a charged particle cannot exist in isolation; it is surrounded by an electromagnetic field. We apply the self-energy correction factor $(1+\alpha)$ to account for the total effective coupling of the particle plus its field:

\begin{equation}
y_{e,phys} \approx y_{e,bare} \cdot (1 + \alpha)
\end{equation}

Substituting $\alpha \approx 1/\AlphaInvVal$\dots:
\begin{equation}
y_{e,phys} \approx 2.9077 \times 10^{-6}  \cdot 1.00730 \approx \mathbf{\ElectronYukawaVal}
\end{equation}

\subsubsection{Validation}
We compare this geometric derivation to the Standard Model definition of the Electron Yukawa coupling:
\begin{equation}
y_e^{SM} = \frac{\sqrt{2} m_e}{v} = \frac{\sqrt{2} \cdot 0.511 \text{ MeV}}{246.22 \text{ GeV}} \approx 2.93 \times 10^{-6}
\end{equation}

\textbf{Result:} Our geometric prediction matches the Standard Model value to within 0.2\%.

\subsubsection{Physical Interpretation}
The electron exists at the absolute limit of the vacuum's resolution, the "noise floor" of the universe. Any mass coupling smaller than $y_e \approx 3 \times 10^{-6}$ cannot be distinguished from background fluctuations and will not form a stable charged particle.

This geometric floor explains several fundamental features of the particle spectrum:
\begin{itemize}
    \item \textbf{Why the electron is stable:} It sits at the minimum resolvable mass. There is no "lower shelf" to decay to; the configuration space volume sets a hard lower bound for charged knots.
    \item \textbf{Why neutrinos are so light:} Neutrinos (if massive) must have Yukawa couplings $y_\nu \ll y_e$, pushing them below the resolution floor. This implies they cannot acquire mass through the standard Higgs mechanism (which stops at $y_e$) but require a different mechanism (explored in Paper II).
    \item \textbf{Absence of lighter particles:} The configuration space volume ($V_{config} \approx 3.4 \times 10^5$) prohibits the existence of any charged particle lighter than the electron. Anything smaller dissolves into the lattice geometry.
\end{itemize}

\subsubsection{Closure: The Impedance Matching Condition}
If the Higgs is a functional regulator, its total internal impedance must match the aperture of the force it mediates (the Weak Interaction).

\begin{enumerate}
    \item \textbf{Weak Aperture:} From the System I invariants, the Weak Force acts through the aperture defined by Symmetry plus the Vacuum Unit:
    $$ \text{Aperture} = \sigma + 1 = 6 $$
    \item \textbf{Higgs Impedance ($Z_H$):} Defined as the \textbf{Inverse Protocol} ($1/\lambda$) modulated by the \textbf{Temporal Cost} ($e^{-T_H} = e^{-2\lambda}$), representing the effective resistance of the field after accounting for instability losses:
    \begin{equation}
    Z_H(\lambda) = \frac{1}{\lambda} e^{-2\lambda}
    \end{equation}
\end{enumerate}

\textbf{The Test:} Substituting the derived geometric coupling $\lambda \approx 0.129424$:
\begin{equation}
Z_H \approx \frac{1}{0.129424} \cdot e^{-0.2588} \approx 7.726 \cdot 0.772 \approx \mathbf{5.966}
\end{equation}

The result $5.966 \approx 6$ (error $< 0.6\%$) confirms that the Higgs sector is not arbitrary. It is the unique geometric solution that \textbf{impedance-matches} the Lattice Resonance to the Weak Force Aperture.






\subsection{Validation: The Higgs Satisfies the Persistence Principle}

To confirm the Higgs is a legitimate persistent system, we verify it minimizes Entropic Action subject to the six-pillar constraints.

The Higgs action functional is:
\begin{equation}
S_H = \int d^4x \left[ |D_\mu\phi|^2 - V(\phi) \right]
\end{equation}

where the potential emerges from the IE constraints:
\begin{equation}
V(\phi) = -\mu^2|\phi|^2 + \lambda|\phi|^4
\end{equation}

The system minimizes $S_H$ by transitioning from the unstable false vacuum ($\phi=0$) to the true vacuum ($|\phi| = v/\sqrt{2}$). This transition occurs because:

\begin{enumerate}
    \item \textbf{Energy Vessel ($\Delta E_H$)}: The capacity $v$ is determined by lattice invariants (Eq. X).
    \item \textbf{Coordination Protocol ($MI_H$)}: The coupling $\lambda$ is fixed by bandwidth allocation (Eq. Y).
    \item \textbf{Temporal Cost ($T_H$)}: The instability factor $T_H = 2\lambda$ drives SSB.
    \item \textbf{Stabilizing Governor ($G_H$)}: The quartic term prevents divergence, enforcing $\chi=2$.
\end{enumerate}

The vacuum condition $\partial V/\partial|\phi| = 0$ yields:
\begin{equation}
\mu^2 = \lambda v^2
\end{equation}