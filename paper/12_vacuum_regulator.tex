\section{System V: The Surface Regulator (The Higgs Sector) (\texorpdfstring{$v, \lambda, m_H, y_e$}{vlambdamhye})}
\label{sec:Vacuum_Regulator}

\CatchFileBetweenTags{\AlphaInvVal}{calculations/constants.tex}{AlphaInvVal}
\CatchFileBetweenTags{\MeMeVPrint}{calculations/constants.tex}{MeMeVPrint}

% Delta E

\CatchFileBetweenTags{\HiggsVEVStepOneVal}{calculations/constants.tex}{HiggsVEVStepOneVal}
\CatchFileBetweenTags{\HiggsVEVStepTwoVal}{calculations/constants.tex}{HiggsVEVStepTwoVal}
\CatchFileBetweenTags{\HiggsVEVVal}{calculations/constants.tex}{HiggsVEVVal}
\CatchFileBetweenTags{\HiggsVEVExperimentalValue}{calculations/constants.tex}{HiggsVEVExperimentalValue}
\CatchFileBetweenTags{\HiggsVEVAccText}{calculations/constants.tex}{HiggsVEVAccText}

\CatchFileBetweenTags{\FermiConstVal}{calculations/constants.tex}{FermiConstVal}
\CatchFileBetweenTags{\FermiConstExperimentalValue}{calculations/constants.tex}{FermiConstExperimentalValue}
\CatchFileBetweenTags{\FermiConstAccText}{calculations/constants.tex}{FermiConstAccText}

% MI
\CatchFileBetweenTags{\HiggsLambdaVal}{calculations/constants.tex}{HiggsLambdaVal}
\CatchFileBetweenTags{\HiggsLambdaExperimentalValue}{calculations/constants.tex}{HiggsLambdaExperimentalValue}
\CatchFileBetweenTags{\HiggsLambdaAccText}{calculations/constants.tex}{HiggsLambdaAccText}

% T
\CatchFileBetweenTags{\HiggsMassVal}{calculations/constants.tex}{HiggsMassVal}
\CatchFileBetweenTags{\HiggsMassExperimentalValue}{calculations/constants.tex}{HiggsMassExperimentalValue}
\CatchFileBetweenTags{\HiggsMassAccText}{calculations/constants.tex}{HiggsMassAccText}

% PM
\CatchFileBetweenTags{\ElectronYukawaVal}{calculations/constants.tex}{ElectronYukawaVal}
\CatchFileBetweenTags{\ElectronYukawaStepOneVal}{calculations/constants.tex}{ElectronYukawaStepOneVal}
\CatchFileBetweenTags{\ElectronYukawaExperimentalValue}{calculations/constants.tex}{ElectronYukawaExperimentalValue}
\CatchFileBetweenTags{\ElectronYukawaAccText}{calculations/constants.tex}{ElectronYukawaAccText}

\CatchFileBetweenTags{\HiggsImpedanceVal}{calculations/constants.tex}{HiggsImpedanceVal}
\CatchFileBetweenTags{\WeakApertureProjVal}{calculations/constants.tex}{WeakApertureProjVal}
 
\subsection{The Standard Model Ansatz}
In the Standard Model, the electroweak sector is parameterized by independent inputs ($v, \lambda, \mu^2$) to generate the Higgs potential $V(\phi) = -\mu^2|\phi|^2 + \lambda|\phi|^4$. While effective, this offers no structural reason for the specific energy scale ($v \approx 246$ GeV) or the coupling strength ($\lambda \approx 0.13$). It requires manual tuning to stabilize the mass against radiative corrections (The Hierarchy Problem).

\subsection{The E8-Persistence Derivation}
We identify the Higgs Field not merely as a boson, but as the \textbf{Surface Regulator} of the lattice. It is a nested persistent system designed to impedance-match the high-frequency lattice resonance ($\Delta$) to the specific aperture of the Weak Interaction. It acts as a ``voltage regulator,'' stepping down the energy density to create the stable \textbf{Matter Scale}.

\subsection{The System Specification}
We define the Higgs architecture by instantiating the six pillars of persistence:

\begin{itemize}
    \item \textbf{Capacity ($\Delta E$): The Vacuum Expectation Value ($v$).} \\
    The energy depth of the regulator. It acts as the fundamental \textbf{Mass Unit} against which the vacuum potential is normalized.
    
    \item \textbf{Identity ($\Delta I$): Scalar Hypercharge ($Y=1$).} \\
    The address of the regulator. It corresponds to the inverse of the lattice Unitary Ground State ($\Delta^0=1$), mandating an $SU(2)$ doublet structure.
    
    \item \textbf{Protocol ($MI$): Self-Coupling ($\lambda$).} \\
    The bandwidth fraction allocated for self-interaction. It determines the stiffness of the field.
    
    \item \textbf{Governor ($G$): The Quartic Potential ($\lambda|\phi|^4$).} \\
    The geometric constraint enforcing the topological boundary ($\chi=2$) to prevent field divergence under negative pressure.
    
    \item \textbf{Temporal Cost ($T$): Instability Factor ($\mu^2$).} \\
    The entropic cost of maintaining the broken symmetry state (The False Vacuum).
    
    \item \textbf{Resolution Floor ($PM$): The Electron Yukawa ($y_e$).} \\
    The minimum mass bit resolvable against thermal noise, setting the floor for the fermion spectrum.
\end{itemize}


\subsection{Derivation A: Capacity (The VEV)}
We calculate the static potential minimum defined by the lattice invariants.

\subsubsection{The Bare Geometric Floor}
The VEV is derived from the Lattice Resonance ($\Delta$) scaled by the structural overhead ($H_{struct}$) and the vacuum impedance ($\alpha^{-1}$).
\begin{equation}
v_{geo} = (\chi \Delta^2 - H_{struct}) \cdot \alpha^{-1} \cdot m_e 
\end{equation}
Substituting the invariants ($H_{struct} = 188$ from System I):

\begin{equation}
\begin{split}
v_{geo} &= (2 \cdot 43^2 - 188) \cdot \AlphaInvVal \cdot \MeMeVPrint \text{ MeV} \\
&\approx \HiggsVEVStepOneVal
\end{split}
\end{equation}

\subsubsection{Geometric Dilution (Topology Correction)}
The regulator energy is diluted by the topology of the charge boundaries. We define the \textbf{Effective Dimension} ($D_{eff}$) of the space as the manifold rank ($D=4$) augmented by the topological boundary ($\chi=2$) distributed over the spherical gauge geometry ($4\pi$).

\begin{equation}
D_{eff} = D + \frac{\chi}{4\pi} \approx 4.15915
\end{equation}

The effective capacity is the geometric floor diluted by this fractional dimension acting through the vacuum coupling:
\begin{equation}
v_{topo} = v_{geo} \left( 1 + \frac{\alpha}{D_{eff}} \right) \approx  \HiggsVEVStepTwoVal
\end{equation}


\subsubsection{The Thermodynamic Noise Floor}
Finally, we account for the \textbf{Persistence Margin} ($PM$). Since the vacuum stability floor supports $n_{gen}=3$ generations ($\sigma - \chi = 3$), the noise amplitude per generation is partitioned linearly:
\begin{equation}
v_{phys} = v_{topo} \left( 1 - \frac{PM}{3} \right)
\end{equation}
Substituting $PM \approx \ElectronYukawaStepOneVal$:
\begin{equation}
v_{phys} \approx \textbf{$\HiggsVEVVal$}
\end{equation}

\begin{itemize}
    \item \textbf{Geometric Prediction:} $\HiggsVEVVal$
    \item \textbf{Experimental Value:} \HiggsVEVExperimentalValue
    \item \textbf{Accuracy:} \HiggsVEVAccText
\end{itemize}

\subsubsection*{Derived Limit: The Fermi Constant (\texorpdfstring{$G_F$}{GF})}
$G_F$ is the inverse squared cross-section of this stability floor, normalized by the topological boundary ($\chi=2$):
\begin{equation}
G_F = \frac{1}{\sqrt{\chi} v^2} \approx \mathbf{\FermiConstVal}
\end{equation}

\textbf{Experimental Value:} \FermiConstExperimentalValue

\textbf{Accuracy:} \FermiConstAccText

\subsection{Identity (\texorpdfstring{$\Delta I$}{DeltaI}): Scalar Hypercharge (\texorpdfstring{$Y=1$}{Y1})} 
The address of the regulator, corresponding to the inverse of the lattice Unitary Ground State ($\Delta^0 =1$). This unitary charge, combined with the topological boundary ($X=2$), mandates an $SU(2)$ doublet structure to interface with the chiral matter sector.


\subsection{Derivation B: Coordination Protocol (The Coupling \texorpdfstring{$\lambda$}{lambda})}
In Informational Energetics, coupling constants are bandwidth allocations. The self-coupling $\lambda$ is the ratio of the \textbf{Net Available Bandwidth} to the \textbf{Total Systemic Channel Capacity} .

\subsubsection{The Net Bandwidth}
The bandwidth available for the scalar sector is the Interaction Remainder ($\sigma - \chi$) minus the \textbf{Resonant Cost} ($1/\Delta$).
\begin{equation}
\text{Net} = (\sigma - \chi) - \frac{1}{\Delta} = 3 - \frac{1}{43} \approx 2.9767
\end{equation}
The factor $1/\Delta$ represents the fractional bandwidth consumed by temporal synchronization. Without this reservation, the field would drift out of phase with the lattice update cycle, causing decoherence.

\subsubsection{The Systemic Channel (\texorpdfstring{$H_{sys}$}{Hsys})}
The total active degrees of freedom in the system:
\begin{equation}
H_{sys} = \nu + \sigma + \chi = 16 + 5 + 2 = 23
\end{equation}

\subsubsection{The Coupling Calculation}
\begin{equation}
\lambda = \frac{(\sigma - \chi) - 1/\Delta}{H_{sys}} = \frac{2.976744}{23} \approx \mathbf{\HiggsLambdaVal}
\end{equation}
\begin{itemize}
    \item \textbf{Experimental Value:} \HiggsLambdaExperimentalValue
    \item \textbf{Accuracy:} \HiggsLambdaAccText
\end{itemize}

\textbf{Physical Interpretation:} The Higgs self-coupling is not arbitrary; it is the specific fraction of vacuum bandwidth ($\approx 13\%$) remaining for self-regulation after paying the entropic cost for temporal synchronization.

\subsection{Governor (\texorpdfstring{$G$}{G}): The Quartic Potential (\texorpdfstring{$\lambda|\phi|^4$}{lambdaphi4}).}
The stabilizing constraint enforcing the topological boundary ($\chi=2$). It caps the tachyonic instability ($-\mu^2$) to prevent runaway vacuum condensation, ensuring the potential creates a closed, stable manifold.


\subsection{Derivation C: Temporal Cost (Instability \texorpdfstring{$\mu^2$}{mu})}
The instability factor $\mu^2$ drives Spontaneous Symmetry Breaking. In IE, this is the entropic cost of maintaining the ``False Vacuum'' state. The $SU(2)$ doublet structure ($\chi=2$) requires two independent complex fields. The total instability ($T_H$) is the sum of their individual contributions:
\begin{equation}
T_H = \chi \cdot \lambda = 2\lambda \approx 0.259
\end{equation}
This recovers the Standard Model relation for the tachyonic mass parameter from geometric first principles:
\begin{equation}
\mu^2 = \lambda v^2 = \frac{T_H}{2} v^2 = \frac{2\lambda}{2} v^2 = \lambda v^2
\end{equation}
This solves the ``Tachyonic Mass" problem: the negative mass parameter $-\mu^2$ is simply the manifestation of this cost acting on the VEV capacity.

\subsection{Output: The Higgs Mass (\texorpdfstring{$m_H$}{mH})}
With the Capacity ($v$) and Protocol ($\lambda$) established, the mass of the scalar excitation is the closure of the geometric system.

\begin{equation}
m_H = \sqrt{2\lambda} v_{phys}
\end{equation}
\begin{equation}
\begin{split}
m_H = \sqrt{2 (0.12942)} \cdot (\HiggsVEVVal) \approx \\
\mathbf{\HiggsMassVal}
\end{split}
\end{equation}

\begin{itemize}
    \item \textbf{Experimental Value:} \HiggsMassExperimentalValue
    \item \textbf{Accuracy:} \HiggsMassAccText
\end{itemize}


\subsection{Derivation D: Resolution Floor (The Electron Yukawa \texorpdfstring{$y_e$}{ye})}

The Persistence Margin ($PM$) represents the smallest non-zero bit of mass the lattice can resolve against thermal noise. This determines the coupling of the lightest charged particle (the Electron). We derive this scale as the Unit Bit (1) diluted over the \textbf{Total Configuration Space Volume}.

\begin{equation}
y_{e,bare} = \frac{1}{V_{config}} = \frac{1}{\underbrace{H_{full}}_{\text{Budget}} \cdot \underbrace{(\sigma+1)}_{\text{Aperture}} \cdot \underbrace{\Delta^2}_{\text{Area}}}
\end{equation}

\subsubsection{Derivation E: The Geometric Self-Energy Correction}

Standard field theory treats the difference between bare and physical parameters as ``Renormalization" arising from virtual particle loops. In the $E_8$-Persistence framework, this difference is the \textbf{Geometric Stress} of embedding the knot's internal symmetry into the spacetime manifold.

A charged particle possesses an interaction structure of rank $\sigma=5$ (from System I), but it must exist within a manifold of rank $D=4$. This dimensional mismatch ($\sigma > D$) creates an irreducible ``Overpressure" on the surrounding lattice.

We define the \textbf{Projection Coefficient} ($\mathcal{P}$) as the ratio of the internal load to the manifold capacity:
\begin{equation}
\mathcal{P} = \frac{\text{Interaction Rank}}{\text{Manifold Rank}} = \frac{\sigma}{D} = \frac{5}{4} = 1.25
\end{equation}

The total effective coupling ($y_{phys}$) is the bare geometric floor ($y_{bare}$) enhanced by this projection stress acting through the vacuum impedance ($\alpha$):

\begin{equation}
y_{e,phys} = y_{e,bare} \cdot \left(1 + \mathcal{P} \cdot \alpha \right) = y_{e,bare} \cdot \left(1 + \frac{\sigma}{D}\alpha \right)
\end{equation}

\textbf{Numerical Validation:}
Substituting the invariants ($H_{full}=31, \sigma+1=6, \Delta=43$) and the derived $\alpha$:

\begin{equation}
y_{e,phys} \approx \left(\frac{1}{31 \cdot 6 \cdot 43^2}\right) \cdot \left(1 + 1.25 \cdot \frac{1}{137.036}\right)
\end{equation}

\begin{equation}
y_{e,phys} \approx \ElectronYukawaStepOneVal \cdot (1.00912) \approx \mathbf{2.9342 \times 10^{-6}}
\end{equation}

\begin{itemize}
    \item \textbf{Geometric Prediction:} $\ElectronYukawaVal$
    \item \textbf{Experimental Value (SM):} \ElectronYukawaExperimentalValue
    \item \textbf{Accuracy:} \ElectronYukawaAccText
\end{itemize}

\subsubsection{Physical Consequences: The Spectral Floor}

This geometric floor establishes the lower bound of information storage in the vacuum, explaining several fundamental features of the particle spectrum:

\begin{itemize}
    \item \textbf{Why the electron is stable:} It sits at the minimum resolvable mass. There is no ``lower shelf'' to decay to; the configuration space volume sets a hard lower bound for charged knots. Any topological knot with coupling $y < y_e$ falls below the vacuum resolution floor and dissolves.
    \item \textbf{Why neutrinos are so light:} Massive neutrinos possess Yukawa couplings $y_\nu \ll y_e$, pushing them below the resolution floor. This structurally prohibits them from acquiring mass via the standard Higgs mechanism (which terminates at $y_e$), necessitating the alternative phonon mechanism derived in Paper II.
    \item \textbf{Absence of lighter particles:} The configuration space volume ($V_{config} \approx 3.4 \times 10^5$) prohibits the existence of any charged particle lighter than the electron. Anything smaller dissolves into the lattice geometry.
\end{itemize}

\textbf{Forward Link:} This derivation establishes the electron as the lightest possible charged particle. Its physical mass is recovered via $m_e = y_e \cdot v/\sqrt{2} \approx 0.511$ MeV. The masses of all heavier fermions are derived in Paper II via the Residual-Lifetime Power Law.

\subsection{Validation: The Impedance Matching Condition}
The ultimate structural test of the Higgs sector is whether it minimizes entropic drag. If the Higgs is a functional regulator, its total internal impedance ($Z_H$) must match the \textbf{Projected Aperture} of the force it mediates (the Weak Interaction).

\begin{enumerate}
    \item \textbf{Higgs Impedance ($Z_H$):} Defined as the Inverse Protocol ($1/\lambda$) modulated by the Temporal Cost ($e^{-2\lambda}$):
    \begin{equation}
    Z_H = \frac{1}{\lambda} e^{-2\lambda} \approx \frac{1}{0.129424} (0.7719) \approx \mathbf{\HiggsImpedanceVal}
    \end{equation}
    
    \item \textbf{Manifold Friction cost ($\eta$):} This projection occurs across the physical manifold, so it must also pay the Manifold Friction cost ($\eta$)
    
    \begin{equation}
    \begin{split}
    A_{weak} = & (\sigma + 1) \cdot \eta \approx \mathbf{\WeakApertureProjVal}
    \end{split}
    \end{equation}
\end{enumerate}

\textbf{Result:} The regulator impedance (\HiggsImpedanceVal) matches the projected aperture (\WeakApertureProjVal) to within \textbf{0.01\%}. 

This structural lock confirms the Higgs scale is derived from the lattice resonance. The remaining deviation represents the operational tolerance of the regulator. In Informational Energetics, a system with zero tolerance cannot process information; the mismatch is the necessary overhead for dynamic operation.