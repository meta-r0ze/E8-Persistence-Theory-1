\section{System IV: Architecture of Matter}
With the static geometry of the vacuum established (System I, System II, System III, System IV), the subsequent papers in this series will derive the resonant excitations of this substrate. Table \ref{tab:system_ArchitectureOfMatter} outlines this Architecture of Matter, demonstrating how the constants derived here serve as the construction rules for the stable particle spectrum.

\begin{table*}[h]
\centering
\caption{\textbf{System V: Architecture of Matter.} The emergence of stable particles, atoms, and nuclei as resonant solutions of the Effective Field Limits.}
\label{tab:system_ArchitectureOfMatter}
\renewcommand{\arraystretch}{1.5}
\setlength{\tabcolsep}{6pt}
\begin{tabularx}{\textwidth}{l|l|l|r|l}
\toprule
\textbf{IE Pillar} & \textbf{Component} & \textbf{Construction Rule} & \textbf{Archetype} & \textbf{System Function} \\
\midrule
\textbf{Substrate ($S$)} & Invariant Substrate & System I ($\mathbb{S}$) & \textbf{$E_8$} & Metric constraints of the 4D projection. \\
\textbf{Substrate ($S$)} & Input Impedance & System II ($\mathbb{O}$) & \textbf{$\alpha^{-1}$} & The Baseline Cost \\
\textbf{Substrate ($S$)} & Field Limits & System III ($\mathbb{C}$) & \textbf{Standard Model} & \textbf{The Runtime Rules} \\
\midrule
\textbf{Energy Vessel ($\Delta E$)} & Lattice Phonon & Inverse Resonance ($1/\Delta^n$) & \textbf{Neutrino} & Energy Sink (Thermodynamic Balance) \\
\textbf{Energy Vessel ($\Delta E$)} & Vacuum Resonance & Harmonic ($2\alpha^{-1}$) & \textbf{Pion} & Nuclear Binding (Glue). \\
\addlinespace
\textbf{Info. Model ($\Delta I$)} & Resonant Knot & Geometric Lock ($\Delta^2 - \pi D$) & \textbf{Proton} & Baryonic Identity (Stable Memory) \\
\textbf{Info. Model ($\Delta I$)} & Isospin Anchor & Symmetry Half-Step ($\sigma/2$) & \textbf{Neutron} & Charge Neutralization. \\
\textbf{Info. Model ($\Delta I$)} & Physical Radius & Projection Scale ($D \cdot \lambda_C$) & \textbf{Charge Radius} & Spatial Extent (Interaction Volume) \\
\addlinespace
\textbf{Protocol ($MI$)} & Ground State & Zero-Entropy Address ($\Delta^0$) & \textbf{Electron} & Charge Carrier (Chemical Agent) \\
\textbf{Protocol ($MI$)} & Atomic Orbital & Impedance Matching ($\alpha^2 m_e$) & \textbf{Hydrogen} & Bonding Interface (Rydberg). \\
\addlinespace
\multicolumn{5}{l}{\textit{The Stabilizing Governor ($G$) --- Nuclear Limits (Magic Numbers)}} \\
\textbf{Governor ($G$)} & Magic Number 2 & Boundary ($\chi$) & \textbf{Helium (2)} & Minimal Topological Closure. \\
\textbf{Governor ($G$)} & Magic Number 8 & Manifold Double ($2D$) & \textbf{Oxygen (8)} & Spinor Capacity. \\
\textbf{Governor ($G$)} & Magic Number 20 & Projection ($D \cdot \sigma$) & \textbf{Calcium (20)} & Symmetric Packing. \\
\textbf{Governor ($G$)} & Magic Number 28 & Capacity ($\mathbb{H_{sys} + \sigma}$) & \textbf{Nickel (28)} & System Saturation. \\
\addlinespace
\textbf{Governor ($G$)} & Magic Number 50 & Harmonic ($\Delta + \sigma + \chi$) & \textbf{Tin (50)} & Resonant Stability. \\
\textbf{Governor ($G$)} & Magic Number 82 & Harmonic ($2\Delta - D$) & \textbf{Lead (82)} & Heavy Saturation. \\
\textbf{Governor ($G$)} & Magic Number 126 & Harmonic ($3\Delta - (\sigma - \chi)$) & \textbf{Shell (126)} & Interaction Limit. \\
\addlinespace
\multicolumn{5}{l}{\textit{The Thermodynamic Taxes (Manifestation in Matter)}} \\
\textbf{Temporal Tax ($T$)} & Fine Structure & Spin-Orbit Coupling ($\alpha^4$) & \textbf{Splitting} & Entropic cost of orbital movement \\
\textbf{Margin ($PM$)} & Mass Defect & Binding Ratio ($\chi\Delta / D\sigma$) & \textbf{Deuteron} & Energy released to purchase stability \\
\bottomrule
\end{tabularx}
\end{table*}



\section{The Formal Mapping Function: From Lattice to Observable} \label{sec:mapping}

To ensure the $E_8$-Persistence Theory is a computable framework rather than a purely interpretive one, we  define the formal mapping from the abstract lattice geometry to the world of physical observables. This function acts as the definitive recipe for calculating the fundamental constants from the derived invariants.

\paragraph{Input:} The set of five geometric invariants, $\mathbb{S} = \{D, \Delta, \nu, \sigma, \chi\}$, which are the necessary outputs of the stable $E_8 \to D_4 \oplus D_4$ projection derived in the preceding sections.

\paragraph{Output:} The fundamental physical constants are computed as rational functions of the elements in $\mathbb{S}$. The primary derivations are summarized below, with references to their detailed proofs.

\begin{table}[h!]
\centering
\caption{The Geometric Mapping of Fundamental Constants}
\label{tab:mapping_function}
\begin{tabular}{@{}llc@{}}
\toprule
\textbf{Observable} & \textbf{Geometric Formula (Schematic)} & \textbf{Section} \\
\midrule
Vacuum Impedance & $\alpha^{-1} = f(\pi, \Delta, \chi, D, \sigma, \nu)$ & \S\ref{sec:Vacuum_Impedance} \\
Strong Coupling & $\alpha_s = (\nu + 1/D)/\alpha^{-1}$ & \S\ref{sec:Saturation_Limit} \\
Weak Mixing Angle & $\sin^2\theta_W = \Delta/(D\Delta + \nu + \sigma)$ & \S\ref{sec:Partition_Ratio} \\
Higgs VEV Scale & $v \propto (\chi\Delta^2 - (D\Delta+\nu)) \cdot \alpha^{-1}$ & \S\ref{sec:Structural_Floor} \\
\bottomrule
\end{tabular}
\end{table}

\paragraph{Principle of Sufficiency:} Each physical observable is the manifestation of a specific geometric constraint of the lattice, its impedance, saturation limit, partition ratio, or structural floor. The five invariants $\mathbb{S}$ are hereby posited to be \textit{necessary and sufficient} to derive the complete set of dimensionless constants governing the Standard Model and cosmology. No additional free parameters are required or permitted within this framework.

\section{Conclusion: The Invariant Substrate}

In this work, we have established that the fundamental constants of nature are not arbitrary tuning parameters, but the necessary boundary conditions of a discrete $E_8$ gauge theory projected onto a 4D manifold.

We have demonstrated a \textbf{Derivation Hierarchy} where a single master equation ($\alpha^{-1}$) encodes the complete geometric specification of the vacuum. As summarized in the System Specification (Section III), all fundamental limits derive from the lattice invariants:
\begin{equation}
\mathbb{S} = \{ D=4, \Delta=43, \sigma=5, \nu=16, \chi=2 \}
\end{equation}

This suggests that the universe does not choose these structures; the lattice architecture selects them as the only solvent methods of information propagation. By replacing free parameters with geometric necessities, we move from a descriptive model of physics to a predictive one.

This work serves as a proof-of-concept for Informational Energetics. The fact that standard system constraints (Bandwidth, Protocol, Overhead) map 1:1 to the constants of nature suggests that the laws of physics may be emergent properties of information processing limits.

\begin{acknowledgments}
The author is an independent researcher and received no external funding for this work. 

I would like to thank my friends and family for their patience and support throughout decades of discussions as I tried to understand everything I encountered.
\end{acknowledgments}

\paragraph{Code Availability:} The computational scripts reproducing all numerical 
results in this paper are available at \url{https://github.com/r0ze-at-github/E8-Persistence-Theory-I}.