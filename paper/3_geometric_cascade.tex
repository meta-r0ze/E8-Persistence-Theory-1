\section{The Systems Specifications: The Geometric Cascade}

\CatchFileBetweenTags{\InvHSys}{calculations/results.tex}{InvHSys}
\CatchFileBetweenTags{\InvHFull}{calculations/results.tex}{InvHFull}
\CatchFileBetweenTags{\InvN}{calculations/results.tex}{InvN}

\CatchFileBetweenTags{\AlphaInvVal}{calculations/results.tex}{AlphaInvVal}

\CatchFileBetweenTags{\AlphaSVal}{calculations/results.tex}{AlphaSVal}
\CatchFileBetweenTags{\AlphaSEq}{calculations/results.tex}{AlphaSEq}

\CatchFileBetweenTags{\HiggsVEVVal}{calculations/results.tex}{HiggsVEVVal}
\CatchFileBetweenTags{\HiggsVEVEq}{calculations/results.tex}{HiggsVEVEq}

\CatchFileBetweenTags{\FermiConstVal}{calculations/results.tex}{FermiConstVal}
\CatchFileBetweenTags{\FermiConstEq}{calculations/results.tex}{FermiConstEq}

\CatchFileBetweenTags{\HiggsMassVal}{calculations/results.tex}{HiggsMassVal}
\CatchFileBetweenTags{\HiggsMassEq}{calculations/results.tex}{HiggsMassEq}

\CatchFileBetweenTags{\ElectronYukawaVal}{calculations/results.tex}{ElectronYukawaVal}
\CatchFileBetweenTags{\ElectronYukawaEq}{calculations/results.tex}{ElectronYukawaEq}

\CatchFileBetweenTags{\WeakAngleVal}{calculations/results.tex}{WeakAngleVal}
\CatchFileBetweenTags{\WeakAngleEq}{calculations/results.tex}{WeakAngleEq}

\CatchFileBetweenTags{\PlanckMassVal}{calculations/results.tex}{PlanckMassVal}
\CatchFileBetweenTags{\PlanckMassEq}{calculations/results.tex}{PlanckMassEq}

\CatchFileBetweenTags{\GravCouplingVal}{calculations/results.tex}{GravCouplingVal}
\CatchFileBetweenTags{\GravCouplingEq}{calculations/results.tex}{GravCouplingEq}

\CatchFileBetweenTags{\HiggsLambdaVal}{calculations/results.tex}{HiggsLambdaVal}
\CatchFileBetweenTags{\HiggsLambdaEq}{calculations/results.tex}{HiggsLambdaEq}

\CatchFileBetweenTags{\JarlskogVal}{calculations/results.tex}{JarlskogVal}
\CatchFileBetweenTags{\JarlskogEq}{calculations/results.tex}{JarlskogEq}

Before deriving the Lagrangian dynamics or the specific coupling values, we present the System Specification: the immutable integer constraints outputted from The Invariant Substrate, from which all subsequent physics emerges.
\begin{equation}
\mathbb{S} = \{ D=4, \Delta=43, \sigma=5, \nu=16, \chi=2 \}
\end{equation}
The following tables outline the architectural layers of the $E_8$ lattice projection, moving from invariant constraints to observable matter. Importantly only the unique set is used in the fundamental constants of nature, spanning 47 orders of magnitude from the gravitational coupling ($10^{-45}$) to the Planck mass ($10^{19}$ GeV).

\begin{itemize}
    \item \textbf{\hyperref[tab:system_InvariantSubstrate]{System I}:} The Invariant Substrate (5 integers).
    \item \textbf{\hyperref[tab:system_EntropicAction]{System II}:} The Entropic Action (Lagrangian).
    \item \textbf{\hyperref[tab:system_GeometricImpedance]{System III}:} The Geometric Impedance ($\alpha^{-1}$).
    \item \textbf{\hyperref[tab:system_EffectiveFieldLimits]{System IV}:} The Effective Field Limits (Standard Model Constants).
\end{itemize}

\begin{table*}[h]
\centering
\caption{\textbf{System I: The Invariant Substrate.} The Universe as the projection of the $E_8$ lattice onto a 4D Manifold.}
\label{tab:system_InvariantSubstrate}
\renewcommand{\arraystretch}{1.5}
\setlength{\tabcolsep}{6pt}
\begin{tabular}{@{} l l c r l @{}} 
\toprule
\textbf{IE Pillar} & \textbf{Parameter} & \textbf{Derivation} & \textbf{Value} & \textbf{Systemic Function} \\
\midrule
\textbf{Substrate ($S$)} & Dimension & $D$ & \textbf{4} & Manifold Rank ($|-1| + |3|$) \\
\midrule
\textbf{Energy Vessel ($\Delta E$)} & Lattice Rank & $N_{E8}$ & \textbf{2D (8)} & Parent Capacity ($E_8 \to D_4 \oplus D_4$) \\
\textbf{Energy Vessel ($\Delta E$)} & Resonance & $\Delta$ & \textbf{43} & Fundamental Frequency (Heegner Number) \\
\textbf{Info. Model ($\Delta I$)} & Interaction & $\sigma$ & \textbf{5} & Symmetry Order ($SU(5)$ Precursor) \\
\textbf{Protocol ($MI$)} & Channel & $\nu$ & \textbf{16} & Chiral Diode (Enforcing Arrow of Time) \\
\textbf{Governor ($G$)} & Boundary & $\chi$ & \textbf{2} & Topological Closure (Gauss-Bonnet) \\
\addlinespace
\multicolumn{5}{l}{\textit{The Metric Signature Components (Time/Space)}} \\
\textbf{Temporal Tax ($T$)} & Causality & Sig($-$) & \textbf{$-1$} & \textbf{Time:} Irreversible state update direction. \\
\textbf{Persistent Margin ($PM$)} & Existence & Sig($+$) & \textbf{+3} & \textbf{Space:} Volumetric storage for knots. \\
\midrule
\multicolumn{5}{c}{\textbf{Active Invariant Set} $\mathbb{S} = \{ \Delta=43, \nu=16, \sigma=5, D=4, \chi=2 \}$} \\
\multicolumn{5}{c}{\textit{These 5 integers are the sole inputs exported to System II.}} \\
\bottomrule

\multicolumn{5}{l}{\textit{The Derived Capacities}} \\
\textbf{Substrate ($S$)} & Systemic Channel & $H_{sys} = \nu+\sigma+\chi$ & \textbf{\InvHSys} & \textbf{Active Bandwidth:} Sum of active pillars. \\
\textbf{Substrate ($S$)} & Full Budget & $H_{full} = H_{sys} + 2D$ & \textbf{\InvHFull} & \textbf{Total Load:} Including spacetime overhead. \\
\textbf{Substrate ($S$)} & State Space & $N = 2\nu$ & \textbf{\InvN} & \textbf{Bit Depth:} Total available node addresses. \\
\end{tabular}
\end{table*}


\begin{table*}[h]
\centering
\caption{\textbf{System II: The Geometric Impedance ($\alpha^{-1}$).} Geometric costs required to sustain a coherent signal against the entropy of the manifold. }
\label{tab:system_GeometricImpedance}
\renewcommand{\arraystretch}{1.5}
\setlength{\tabcolsep}{6pt}
\begin{tabular}{@{} l l c r l @{}} 
\toprule
\textbf{IE Pillar} & \textbf{Parameter} & \textbf{Derivation} & \textbf{Value} & \textbf{Physical Function} \\
\midrule
\textbf{Substrate ($S$)} & Invariant Substrate & System I & $\mathbb{S}$ & Metric constraints of the 4D projection. \\
\textbf{Substrate ($S$)} & Golden Ideal & $D\sigma\phi^4$ & $+137.082$ & The frictionless geometric baseline. \\
\midrule
\textbf{Energy Vessel ($\Delta E$)} & Circumference & $\pi \Delta$ & $+135.088$ & Radial-to-Gauge flux conversion. \\
\textbf{Info. Model ($\Delta I$)} & Boundary & $\chi$ & $+2.000$ & Distinguishes Particle from Vacuum \\
\textbf{Protocol ($MI$)} & Alignment & $\frac{-1}{D\Delta-\sigma}$ & $-0.006$ & Drag reduction via symmetry alignment \\
\textbf{Governor ($G$)} & Metric Shear & $-\frac{\chi}{\Delta}$ & $-0.047$ & Vacuum pressure preventing UV divergence \\
\addlinespace
\multicolumn{5}{l}{\textit{The Substrate Costs}} \\
\textbf{Temporal Tax ($T$)} & Entropy & $T_{geo}$ (Eq. \ref{eq:alpha_inverse}) & $+10^{-5}$ & The entropy cost of Weak State transitions. \\
\textbf{Margin ($PM$)} & Resolution & $PM_{geo}$ (Eq. \ref{eq:alpha_inverse}) & $+10^{-6}$ & Minimum energy to define a mass state. \\
\midrule
\multicolumn{5}{c}{\textbf{Active Output Set} $\mathbb{O} = \{ \alpha^{-1} \approx \AlphaInvVal, \ T, \ PM \}$} \\
\multicolumn{5}{c}{\textit{This impedance acts as an input for the Effective Field Limits in System IV.}} \\
\bottomrule
\end{tabular}
\end{table*}

\begin{table*}[h]
\centering
\caption{\textbf{System IV: The Effective Field Limits.} The physical constants of the Standard Model derived as the operational outputs of the Vacuum Impedance on the $E_8$ lattice.}
\label{tab:system_EffectiveFieldLimits}
\renewcommand{\arraystretch}{1.5}
\setlength{\tabcolsep}{6pt}
\begin{tabular}{@{} l l c r l @{}} 
\toprule
\textbf{IE Pillar} & \textbf{Parameter} & \textbf{Derivation} & \textbf{Value} & \textbf{Physical Function} \\
\midrule
\textbf{Substrate ($S$)} & Invariant Substrate & System I & $\mathbb{S}$ & Metric constraints of the 4D projection. \\
\textbf{Substrate ($S$)} & Input Impedance ($\alpha^{-1}_{geo}$) & System II & $\mathbb{O}$ & \textbf{Baseline Cost:} The vacuum resistance. \\
\midrule
\textbf{Energy Vessel ($\Delta E$)} & Gravity ($\alpha_G$) &
    $\GravCouplingEq$ & $\GravCouplingVal$ & \textbf{Bulk Attenuation:} Signal loss across lattice depth. \\
\textbf{Energy Vessel ($\Delta E$)} & Planck Mass ($M_P$) &
    $\PlanckMassEq$ & $\mathbf{\PlanckMassVal}$ GeV & \textbf{Unity Threshold:} The scale where $\alpha_G \to 1$ \\
\addlinespace
\textbf{Info. Model ($\Delta I$)} & Higgs VEV ($v$) & 
    $\HiggsVEVEq$ & $\mathbf{\HiggsVEVVal}$ GeV & \textbf{Stability Floor:} The potential minimum \\
\textbf{Info. Model ($\Delta I$)} & Fermi Const. ($G_F$) &
    $\FermiConstEq$ & $\mathbf{\FermiConstVal}$ & \textbf{Interaction Area:} The weak cross-section \\
\textbf{Info. Model ($\Delta I$)} & Higgs Mass ($m_H$) &
    $\HiggsMassEq$ & $\mathbf{\HiggsMassVal}$ GeV & \textbf{Scalar Mass:} The excitation of the floor \\
\addlinespace
\textbf{Protocol ($MI$)} & Strong Force ($\alpha_s$) &
    $\AlphaSEq$ & $\mathbf{\AlphaSVal}$ & \textbf{Saturation:} Full channel occupancy load \\
\textbf{Protocol ($MI$)} & Weak Angle ($\theta_W$) &
    $\WeakAngleEq$ & $\mathbf{\WeakAngleVal}$ & \textbf{Partition:} Resonance fraction of total bandwidth \\
\textbf{Protocol ($MI$)} & W Boson Mass ($M_W$) & $M_Z \sqrt{1 - \sin^2\theta_W}$ & $\mathbf{80.395}$ GeV & \textbf{Weak Range:} Protocol screening length. \\
\textbf{Protocol ($MI$)} & Cabibbo Angle ($\theta_C$) & $\arcsin(\sqrt{m_d/m_s})$ & $\mathbf{0.2248}$ & \textbf{Flavor Aperture:} Leakage between generations. \\
\textbf{Governor ($G$)} & Self-Coupling ($\lambda$) &
    $\HiggsLambdaEq$ & $\mathbf{\HiggsLambdaVal}$ & \textbf{Vacuum Rigidity:} Resistance to field deformation \\
\textbf{Governor ($G$)} & QCD Beta Func. ($\beta_0$) & $11 - \frac{2}{3}n_f$ & $\mathbf{Function}$ & \textbf{Field Rigidity:} Vacuum anti-screening limit. \\
\addlinespace
\multicolumn{5}{l}{\textit{The Thermodynamic Taxes}} \\
\textbf{Temporal Tax ($T$)} & Jarlskog Inv. ($J$) &
    $\JarlskogEq$ & $\mathbf{\JarlskogVal}$ & \textbf{Time Asymmetry:} Cost of the Chiral Diode. \\
\textbf{Margin ($PM$)} & Yukawa ($y_e$) &
    $\ElectronYukawaEq$ & $\mathbf{\ElectronYukawaVal}$ & \textbf{Resolution Floor:} Minimum coupled mass. \\
\midrule
\multicolumn{5}{c}{\textbf{Active Output Set} $\mathbb{C} = \{ \alpha_G, M_P, v, \alpha_s, \theta_W,  m_H, \theta_C, \lambda, J, y_e \}$} \\
\multicolumn{5}{c}{\textit{These constants form the boundary conditions for the Matter Spectrum in System IV.}} \\
\bottomrule
\end{tabular}
\end{table*}