\CatchFileBetweenTags{\CabibboAngleVal}{calculations/constants.tex}{CabibboAngleVal}
\CatchFileBetweenTags{\CabibboAngleExperimentalValue}{calculations/constants.tex}{CabibboAngleExperimentalValue}
\CatchFileBetweenTags{\CabibboAngleEq}{calculations/constants.tex}{CabibboAngleEq}

\subsection{The Flavor Aperture: Cabibbo Angle (\texorpdfstring{$\theta_C$}{thetaC})}
While the Weak Angle ($\theta_W$) governs the partition of forces, the Cabibbo Angle ($\theta_C$) governs the partition of generations (Flavor Mixing). In the $E_8$-Persistence framework, this represents the **Geometric Leakage** between the resonant tiers of the lattice.

The leakage is determined by the ratio of the circular interface ($\pi$) to the active channel width of the flavor sector. The flavor width is the Total Chiral Capacity ($\nu$) minus the Topological Boundary ($\chi$):

\begin{equation}
\sin \theta_C = \frac{\text{Interface}}{\text{Flavor Channel}} = \frac{\pi}{\nu - \chi}
\end{equation}

Substituting $\nu=16$ and $\chi=2$:
\begin{equation}
\sin \theta_C = \frac{\pi}{14} \approx \mathbf{0.224399}
\end{equation}

\begin{itemize}
    \item \textbf{Geometric Prediction:} \CabibboAngleExperimentalValue
    \item \textbf{Experimental Value:} $0.2250 \pm 0.0007$
    \item \textbf{Agreement:} Within $1\sigma$.
\end{itemize}

\textbf{Physical Function:} Flavor mixing is not a random quantum rotation. It is the inevitable geometric result of fitting a circular interaction ($\pi$) into a linear channel of width 14. The angle $\pi/14$ is the "Aperture" that allows matter to transition between generations.