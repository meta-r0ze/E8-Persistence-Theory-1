\section{System IV-C: The Geometric leakage: Cabibbo Angle (\texorpdfstring{$\theta_C$}{thetaC})}

\CatchFileBetweenTags{\CabibboAngleVal}{constants.tex}{CabibboAngleVal}
\CatchFileBetweenTags{\CabibboAngleExperimentalValue}{constants.tex}{CabibboAngleExperimentalValue}
\CatchFileBetweenTags{\CabibboAngleEq}{constants.tex}{CabibboAngleEq}
\CatchFileBetweenTags{\CabibboAngleAccText}{constants.tex}{CabibboAngleAccText}

\textbf{The Standard Model Ansatz:} The Cabibbo Angle ($\theta_C \approx 13.0^\circ$) parameterizes the mixing between the first and second generations of quarks. In the Standard Model, it is an empirical rotation angle required to explain why quarks can transition between flavors. Its specific value is unexplained.

\textbf{The E8-Persistence Derivation:} We identify the Cabibbo Angle as the \textbf{Geometric Aperture} of the flavor sector. While the Weak Angle partitions the forces, the Cabibbo Angle represents the \textbf{Geometric Leakage} between the resonant tiers (generations) of the lattice.

\subsection{The System Specification}
To derive the leakage probability, we must define the geometry of the inter-generational interface.

\begin{enumerate}
    \item \textbf{The Interface ($\pi$):} Transitions between generations occur across the circular boundary of the lattice nodes. The geometric cost of traversing this interface is $\pi$.
    
    \item \textbf{The Active Flavor Width ($\nu - \chi$):} The total chiral capacity is $\nu=16$. However, the topological boundary ($\chi=2$) serves as the ``Identity Lock" that maintains particle stability. Therefore, the ``free" bandwidth available for inter-generational mixing is the remainder:
    $$ W_{flavor} = \nu - \chi = 16 - 2 = 14 $$
\end{enumerate}

\subsection{Derivation: The Leakage Ratio}
The mixing angle is defined by the ratio of the Interface to the Active Width. This determines the maximum angle at which a signal can ``slip" from one generation to the next without breaking the topological lock.

\begin{equation}
\sin \theta_C = \frac{\text{Interface}}{\text{Flavor Width}} = \frac{\pi}{\nu - \chi}
\end{equation}

\textbf{Numerical Calculation:}
\begin{equation}
\sin \theta_C = \frac{\pi}{14} \approx \mathbf{\CabibboAngleVal}
\end{equation}
\begin{equation}
\theta_C = \arcsin\left(\frac{\pi}{14}\right) \approx 12.97^\circ
\end{equation}

\textbf{Physical Meaning:} This ratio represents the quantum tunneling amplitude between generational resonances. While the boundary ($\chi$) protects the particle's identity, the open channels ($14$) allow for a precise, geometrically defined leakage between adjacent generation tiers.

\subsection{Validation: Precision Flavor Physics}
We compare this geometric derivation to the world average from kaon and hyperon decays.

\begin{itemize}
    \item \textbf{Geometric Prediction ($\sin \theta_C$):} \CabibboAngleVal
    \item \textbf{Experimental Value (PDG):} \CabibboAngleExperimentalValue
    \item \textbf{Agreement:} \CabibboAngleAccText
\end{itemize}

\textbf{Conclusion:} The Cabibbo Angle is not arbitrary; it is the geometric aspect ratio ($\pi/14$) of a chiral channel constrained by a topological boundary.

\subsection{Geometric Consequence: The GST Relation}
It is notable that the derived Cabibbo mixing ($\sin \theta_C \approx 0.224$) is nearly identical to the Weak Mixing Angle derived in the previous section ($\sin^2 \theta_W \approx 0.223$). 

This confirms the \textbf{Gatto-Sartori-Tonin (GST) Relation} from first principles: the geometry that partitions the forces ($\Delta/N_{sys}$) is structurally coupled to the geometry that mixes the masses ($\pi/(\nu-\chi)$). Both are manifestations of the same underlying lattice capacity, allocating bandwidth between Temporal Coherence (Forces) and Generational Leakage (Matter).