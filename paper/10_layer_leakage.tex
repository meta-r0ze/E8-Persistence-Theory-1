\CatchFileBetweenTags{\CabibboAngleVal}{calculations/constants.tex}{CabibboAngleVal}
\CatchFileBetweenTags{\CabibboAngleExperimentalValue}{calculations/constants.tex}{CabibboAngleExperimentalValue}
\CatchFileBetweenTags{\CabibboAngleEq}{calculations/constants.tex}{CabibboAngleEq}

\subsection{The Flavor Aperture: Cabibbo Angle (\texorpdfstring{$\theta_C$}{thetaC})}
While the Weak Angle ($\theta_W$) governs the partition of forces, the Cabibbo Angle ($\theta_C$) governs the partition of generations (Flavor Mixing). In the Standard Model, this is an empirical parameter ($\sin\theta_C \approx 0.225$).

In the $E_8$-Persistence framework, this angle represents the Geometric Leakage between the first and second resonant tiers of the lattice. It is determined by the ratio of the circular interface ($\pi$) to the active channel width ($\nu - \chi$):

\begin{equation}
    \sin \theta_C = \frac{\pi}{\nu - \chi} + \frac{\alpha}{\nu}
\end{equation}

\begin{equation}
    \sin \theta_C = \frac{\pi}{14} + \frac{\alpha}{16} \approx \mathbf{}
\end{equation}

\begin{itemize}
    \item \textbf{Experimental Value:} \CabibboAngleExperimentalValue
    \item \textbf{Accuracy:} The geometric prediction captures \textbf{99.9\%} of the observed mixing.
\end{itemize}

\textbf{Implication:} This confirms that flavor mixing is not random; it is the inevitable ``crosstalk" between lattice layers defined by the channel invariants.