\crefalias{section}{appendix}
\section{Geometric Origin of Hypercharge}
\label{app:OriginOfHypercharge}

In the main text, we derived the structure of the non-Abelian gauge groups $SU(3)_C$ and $SU(2)_L$ from the integer invariants $\sigma$ and $\chi$. This appendix extends that logic to the Abelian group $U(1)_Y$, demonstrating that the specific hypercharge quantum numbers of the Standard Model are not arbitrary assignments but are the precise values required to maintain consistency between a particle's electric charge and its geometric role within the lattice.

We use the Gell-Mann--Nishijima relation, $Q = I_3 + Y/2$, as a constraint equation. By deriving Weak Isospin ($I_3$) geometrically from the topological boundary $\chi=2$, we can calculate the necessary hypercharge $Y = 2(Q - I_3)$ for each particle and show that it corresponds to a simple ratio of the geometric invariants $\{D, \sigma, \chi\}$.

\subsection{Geometric Isospin (\texorpdfstring{$I_3$}{I3}):}
The $SU(2)_L$ symmetry arises from the topological boundary $\chi=2$, which mandates a doublet structure for left-handed particles. We therefore make the following geometric assignments:
\begin{itemize}
    \item Particles in a left-handed doublet are assigned $I_3 = \pm 1/2$.
    \item Particles that are right-handed singlets are assigned $I_3 = 0$.
\end{itemize}

\subsection{Derivation of Fermion Hypercharges:}
The following table demonstrates that the calculated hypercharge for every Standard Model fermion perfectly matches a unique, simple ratio of the geometric invariants. This provides a powerful confirmation of the framework, linking quantum numbers directly to the geometry of the substrate.

\begin{table}[h!]
\centering
\caption{Derivation of Fermion Hypercharges from Geometric Ratios}
\label{tab:hypercharge_derivation}
\renewcommand{\arraystretch}{1.2}
\begin{tabular}{@{}lccccc@{}}
\toprule
\textbf{Particle} & \textbf{$SU(2)_L$ Rep.} & \textbf{$I_3$} & \textbf{$Q$} & \textbf{Calculated $Y$} & \textbf{Geometric Ratio} \\
\midrule
Neutrino ($L_L$)    & Doublet & $+1/2$ & $0$    & $-1$ & $-D/D$ \\
Electron ($L_L$)    & Doublet & $-1/2$ & $-1$   & $-1$ & $-D/D$ \\
Electron ($e_R$)    & Singlet & $0$    & $-1$   & $-2$ & $-\chi$ \\
\addlinespace
Up Quark ($Q_L$)    & Doublet & $+1/2$ & $+2/3$ & $+1/3$ & $1/(\sigma-\chi)$ \\
Down Quark ($Q_L$)  & Doublet & $-1/2$ & $-1/3$ & $+1/3$ & $1/(\sigma-\chi)$ \\
Up Quark ($u_R$)    & Singlet & $0$    & $+2/3$ & $+4/3$ & $D/(\sigma-\chi)$ \\
Down Quark ($d_R$)  & Singlet & $0$    & $-1/3$ & $-2/3$ & $-\chi/(\sigma-\chi)$ \\
\bottomrule
\end{tabular}
\end{table}

\subsection{Derivation of Higgs Hypercharge:}
The Higgs boson doublet contains a neutral component ($H^0$) and a charged component ($H^+$). For the charged component, $Q=+1$ and $I_3=+1/2$. Applying the same convention as the fermion sector yields its hypercharge:
\[
Y_H = 2(Q - I_3) = 2(1 - 1/2) = +1
\]
Geometrically, this unitary value corresponds to the scalar ground state of the lattice:
\[
Y_H = \frac{1}{\Delta^0} = 1
\]

\subsection{Physical Interpretation of the Ratios:}
The specific ratios assigned to each particle class are not arbitrary; they reflect the particle's fundamental coupling to the geometric structures of the vacuum.
\begin{itemize}
    \item \textbf{Leptons:} As color-singlets, their hypercharges are determined by their coupling to the fundamental manifold topology ($D$) and its boundary ($\chi$), not the color sector ($\sigma-\chi$).
    \item \textbf{Quarks:} As color-triplets, their hypercharges are necessarily normalized by the Interaction Remainder ($\sigma-\chi=3$), reflecting their fundamental coupling to the $SU(3)_C$ gauge structure.
    \item \textbf{Higgs:} As the scalar field that stabilizes the electroweak boundary, its hypercharge is derived from the unitary scalar ground state ($\Delta^0=1$), signifying its foundational role.
\end{itemize}

\textbf{Conclusion:} The hypercharge assignments are not random. They are the unique rational numbers that ensure the electric charge of a particle is consistent with its geometric isospin (determined by $\chi$), its relationship to the color sector (via ratios involving the remainder $\sigma-\chi=3$), and its role within the manifold ($D=4$). The entire charge structure of the Standard Model is shown to be a direct consequence of the five geometric invariants.