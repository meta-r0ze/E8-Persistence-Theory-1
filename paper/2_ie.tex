\section{Theoretical Context: Informational Energetics}
\label{sec:IE}
A subset of complex adaptive systems, persistent systems must optimize for persistence against entropy. 

This framework synthesizes insights from non-equilibrium thermodynamics, algorithmic information theory, and robust control theory, bridging them with empirical principles from evolutionary biology, computational neuroscience, and high-energy physics, and applying them through the practical lenses of institutional economics, quantitative finance, and reliability engineering.

% TODO citations to add, so many possiblities, options...
% Non-equilibrium thermodynamics → cite Prigogine or England
% Algorithmic information theory → cite Kolmogorov, Chaitin, or Solomonoff
% Robust control theory → cite Doyle or Csete & Doyle (2002) on biological robustness
% Evolutionary biology → cite Kauffman (self-organization) or England (dissipation-driven adaptation)

\subsection{The Axiom of Persistence}

The fundamental imperative of persistence is to maximize existence duration against environmental entropy. We formalize this as the \textbf{Persistence Principle}: the minimization of \textbf{Entropic Action ($S_\Phi$)} relative to structural complexity. This action represents the metabolic cost of maintaining a distinct identity. To satisfy this axiom, any persistent entity must implement a specific architecture comprising four structural pillars for information management, plus the thermodynamic overhead of operation.

The Principle of Persistence dictates that a system must minimize the loss of structural information. This creates a bridge between disciplines:
\begin{itemize}
    \item \textbf{To a Physicist:} This is the \textit{Principle of Least Action} applied to geometry.
    \item \textbf{To a Computer Scientist:} This is \textit{Loss Function Minimization} in a distributed network.
    \item \textbf{To a Biologist:} This is the minimization of \textit{Metabolic Drag} required to maintain homeostasis.
\end{itemize}

\subsection{The Structural Pillars}
The set of structural requirements that make up all \textbf{Persistence Systems} ($P$).

\begin{equation}
\label{eq:IE_pillars}
\begin{split}
{} & P = \\ 
&  \underbrace{\Delta E}_{\text{Capacity}}
+ \underbrace{\Delta I}_{\text{Identity}}
- \underbrace{MI}_{\text{Efficiency}}
- \underbrace{G}_{\text{Stability}}
+ \underbrace{T}_{\text{Overhead}}
+ \underbrace{PM}_{\text{Margin}}
\end{split}
\end{equation}

\noindent The six components represent the universal structural requirements of persistence. To manifest, these abstract requirements must map to specific features of a system. These six components represent the minimal complete set: fewer leaves the system unable to persist; additional components reduce to combinations of these. Positive terms represent entropic costs; negative terms represent efficiency gains that reduce the persistence burden.

These structural pillars constitute the universal architecture of any persistent entity:

\begin{enumerate}
    \item \textbf{The Energy Vessel ($\Delta E$):} \textit{The Capacity.} 
    The physical infrastructure required to acquire resources and perform work. It defines the maximum bandwidth, storage limit, or energy throughput of the system. Without a vessel, the system lacks the agency to act.

    \item \textbf{The Information Model ($\Delta I$):} \textit{The Identity.} 
    The internal logic or topological structure that distinguishes the system from the environment. It functions as the predictive engine, encoding the system's configuration to reduce environmental uncertainty. Without a model, the system acts blindly.
    
    \item \textbf{The Coordination Protocol ($MI$):} \textit{The Efficiency.} 
    The communicative glue regulating the flow between the Vessel and the Model. It ensures coherence and minimizes the entropic loss of signal transmission. Without a protocol, the system fragments into isolated parts.
    
    \item \textbf{The Stabilizing Governor ($G$):} \textit{The Stability.} 
    The constraint mechanism that prevents unbounded divergence. It enforces the operational boundaries necessary to maintain structural integrity against internal pressure. Without a governor, the system consumes itself or explodes.
    
    \item \textbf{The Temporal Cost ($T$):} \textit{The Overhead.} 
    The entropic cost of state transitions. It represents the irreversible energy expenditure required to update the system's configuration, enforcing the arrow of time.
    
    \item \textbf{The Persistence Margin ($PM$):} \textit{The Floor.} 
    The buffer for existence. It represents the minimum resolution limit required to distinguish a signal from thermal background noise, or the reserve capacity required to survive fluctuations.
\end{enumerate}

\noindent We distinguish between \textbf{Persistent Systems}, which maintain a distinct identity ($\Delta E > 0$) against the environment, and \textbf{Coordination Signals}, which serve as the transient transmission mechanism for the Protocol ($MI$). A System exists in time; a Signal propagates through space.

In the following sections, we demonstrate that every persistent layer of the universe, from the vacuum substrate to the Higgs mechanism and Gravity is a distinct instantiation of this single architectural blueprint.