\section{Theoretical Context: Informational Energetics}\label{:IE}
A subset of complex adaptive systems, persistent systems must optimize for persistence against entropy. 

This framework synthesizes insights from non-equilibrium thermodynamics, algorithmic information theory, and robust control theory, bridging them with empirical principles from evolutionary biology, computational neuroscience, and high-energy physics, and applying them through the practical lenses of institutional economics, quantitative finance, and reliability engineering.

% TODO citations to add, so many possiblities, options...
% Non-equilibrium thermodynamics → cite Prigogine or England
% Algorithmic information theory → cite Kolmogorov, Chaitin, or Solomonoff
% Robust control theory → cite Doyle or Csete & Doyle (2002) on biological robustness
% Evolutionary biology → cite Kauffman (self-organization) or England (dissipation-driven adaptation)

\subsection{The Axiom of Persistence}

The fundamental imperative of persistence is to maximize existence duration against environmental entropy. We formalize this as the \textbf{Persistence Principle}: the minimization of \textbf{Entropic Action ($S_\Phi$)} relative to structural complexity. This action represents the metabolic cost of maintaining a distinct identity. To satisfy this axiom, any persistent entity must implement a specific architecture comprising four structural pillars for information management, plus the thermodynamic overhead of operation.

\subsection{The Structural Pillars}
The set of structural requirements that make up all \textbf{Persistence Systems} ($P$).

\begin{equation}
\label{eq:IE_pillars}
\begin{split}
{} & P = \\ 
&  \underbrace{\Delta E}_{\text{Capacity}}
+ \underbrace{\Delta I}_{\text{Identity}}
- \underbrace{MI}_{\text{Efficiency}}
- \underbrace{G}_{\text{Stability}}
+ \underbrace{T}_{\text{Overhead}}
+ \underbrace{PM}_{\text{Margin}}
\end{split}
\end{equation}

\noindent The six components represent the universal structural requirements of persistence. To manifest, these abstract requirements must map to specific features of a system. These six components represent the minimal complete set: fewer leaves the system unable to persist; additional components reduce to combinations of these. Positive terms represent metabolic costs; negative terms represent efficiency gains that reduce the persistence burden.

To provide immediate physical context, we highlight the mapping between these abstract requirements and the concrete geometry of the vacuum. When projecting the $E_8$ lattice onto the derived 4D manifold (detailed in Section \ref{sec:DerivationOfTheSubstrate}), these universal structural requirements manifest physically as the following geometric invariants:

\begin{enumerate}
    \item \textbf{The Energy Vessel ($\Delta E$):} \textit{The Capacity.} The structure that holds state (Resonant Bandwidth). In System I, this maps to \textbf{The Fundamental Resonance} ($\Delta=43$).
    \textbf{Mechanism:} In a wave-based substrate, storage capacity is strictly limited by the fundamental non-repeating frequency (Heegner Number) required to prevent aliasing.

    \item \textbf{The Information Model ($\Delta I$):} \textit{The Identity.} The structure used to interact with the environment (Boundary Topology). In System I, this maps to \textbf{The Interaction Symmetry} ($\sigma=5$).
    \textbf{Mechanism:} The complexity of a particle's identity is bounded by the rank of its internal symmetry group ($SU(5)$ precursor).
    
    \item \textbf{The Coordination Protocol ($MI$):} \textit{The Efficiency.} The channel regulating flow (Gauge Alignment). In System I, this maps to \textbf{The Chiral Capacity} ($\nu=16$).
    \textbf{Mechanism:} Efficiency requires directed information flow; geometrically, this requires truncating the total degrees of freedom to the chiral rank (Chiral Projection).
    
    \item \textbf{The Stabilizing Governor ($G$):} \textit{The Stability.} The constraint preventing divergence (Renormalization Cutoff). In System I, this maps to \textbf{The Topological Boundary} ($\chi=2$).
    \textbf{Mechanism:} Infinite dissipation is prevented by enforcing topological closure on the energy vessel (Gauss-Bonnet limit).
    
    \item \textbf{The Temporal Cost ($T$):} \textit{The Overhead.} The cost of updates. In System I, this maps to \textbf{Metric Time} ($-1$).
    \textbf{Mechanism:} State transitions require an irreversible metric signature component to enforce the arrow of time.
    
    \item \textbf{The Persistence Margin ($PM$):} \textit{The Floor.} The buffer for existence. In System I, this maps to \textbf{Metric Space} ($+3$).
    \textbf{Mechanism:} Volumetric capacity is required to store the knots defined by the other pillars.
\end{enumerate}

\noindent \textbf{The Finite Capacity Constraint:} Standard Quantum Field Theory assumes a vacuum of infinite capacity, leading to divergences. By mapping the \textbf{Stabilizing Governor} to the topological boundary ($\chi=2$), we impose a physical mechanism that prevents the Energy Vessel from diverging. The Persistence Principle acts as the selection filter, ensuring only geometric configurations with this functioning Governor survive.

We distinguish between \textbf{Persistent Systems}, which maintain a distinct identity ($\Delta E > 0$) against the vacuum, and \textbf{Coordination Signals} (like the Photon), which serve as the transmission mechanism for the \textbf{Protocol} ($M_I$) pillar. A System exists in time; a Signal propagates through space.

With the universal architecture of persistence established, we now derive the specific geometric invariants of the vacuum substrate.