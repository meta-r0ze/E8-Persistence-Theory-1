\section{System IV: Architecture of Matter}
With the static geometry of the vacuum established (System I, System II, System III, System IV), the subsequent papers in this series will derive the resonant excitations of this substrate. Table \ref{tab:system_ArchitectureOfMatter} outlines this Architecture of Matter, demonstrating how the constants derived here serve as the construction rules for the stable particle spectrum.

\CatchFileBetweenTags{\AlphaSEq}{calculations/constants.tex}{AlphaSEq}
\CatchFileBetweenTags{\WeakAngleEq}{calculations/constants.tex}{WeakAngleEq}
\CatchFileBetweenTags{\HiggsVEVEq}{calculations/constants.tex}{HiggsVEVEq}

\begin{table*}[h]
\centering
\caption{\textbf{System VII: Architecture of Matter.} The emergence of stable particles, atoms, and nuclei as resonant solutions of the Effective Field Limits.}
\label{tab:system_ArchitectureOfMatter}
\renewcommand{\arraystretch}{1.5}
\setlength{\tabcolsep}{6pt}
\begin{tabularx}{\textwidth}{l|l|l|r|l}
\toprule
\textbf{IE Pillar} & \textbf{Component} & \textbf{Construction Rule} & \textbf{Archetype} & \textbf{System Function} \\
\midrule
\textbf{Substrate ($S$)} & Invariant Substrate & System I ($\mathbb{S}$) & \textbf{$E_8$} & Metric constraints of the 4D projection \\
\textbf{Substrate ($S$)} & Geometric Impedance & System II ($\mathbb{O}$) & \textbf{$\alpha^{-1}$} & The Baseline Cost \\
\textbf{Substrate ($S$)} & Field Limits & System III ($\mathbb{C}$) & \textbf{Standard Model} & \textbf{The Runtime Rules} \\
\midrule
\textbf{Energy Vessel ($\Delta E$)} & Lattice Phonon & Inverse Resonance ($1/\Delta^n$) & \textbf{Neutrino} & Energy Sink (Thermodynamic Balance) \\
\textbf{Energy Vessel ($\Delta E$)} & Vacuum Resonance & Harmonic ($2\alpha^{-1}$) & \textbf{Pion} & Nuclear Binding (Glue) \\
\addlinespace
\textbf{Info. Model ($\Delta I$)} & Resonant Knot & Geometric Lock ($\Delta^2 - \pi D$) & \textbf{Proton} & Baryonic Identity (Stable Memory) \\
\textbf{Info. Model ($\Delta I$)} & Isospin Anchor & Symmetry Half-Step ($\sigma/2$) & \textbf{Neutron} & Charge Neutralization \\
\textbf{Info. Model ($\Delta I$)} & Physical Radius & Projection Scale ($D \cdot \lambda_C$) & \textbf{Charge Radius} & Spatial Extent (Interaction Volume) \\
\addlinespace
\textbf{Protocol ($MI$)} & Ground State & Zero-Entropy Address ($\Delta^0$) & \textbf{Electron} & Charge Carrier (Chemical Agent) \\
\textbf{Protocol ($MI$)} & Atomic Orbital & Impedance Matching ($\alpha^2 m_e$) & \textbf{Hydrogen} & Bonding Interface (Rydberg) \\
\addlinespace
\multicolumn{5}{l}{\textit{The Stabilizing Governor ($G$) --- Nuclear Limits (Magic Numbers)}} \\
\textbf{Governor ($G$)} & Magic Number 2 & Boundary ($\chi$) & \textbf{Helium (2)} & Minimal Topological Closure \\
\textbf{Governor ($G$)} & Magic Number 8 & Manifold Double ($2D$) & \textbf{Oxygen (8)} & Spinor Capacity \\
\textbf{Governor ($G$)} & Magic Number 20 & Projection ($D \cdot \sigma$) & \textbf{Calcium (20)} & Symmetric Packing \\
\textbf{Governor ($G$)} & Magic Number 28 & Capacity ($H_{sys} + \sigma$) & \textbf{Nickel (28)} & System Saturation \\
\addlinespace
\textbf{Governor ($G$)} & Magic Number 50 & Harmonic ($\Delta + \sigma + \chi$) & \textbf{Tin (50)} & Resonant Stability \\
\textbf{Governor ($G$)} & Magic Number 82 & Harmonic ($2\Delta - D$) & \textbf{Lead (82)} & Heavy Saturation \\
\textbf{Governor ($G$)} & Magic Number 126 & Harmonic ($3\Delta - (\sigma - \chi)$) & \textbf{Shell (126)} & Interaction Limit \\
\addlinespace
\multicolumn{5}{l}{\textit{The Thermodynamic Taxes (Manifestation in Matter)}} \\
\textbf{Temporal Cost ($T$)} & Fine Structure & Spin-Orbit Coupling ($\alpha^4$) & \textbf{Splitting} & Entropic cost of orbital movement \\
\textbf{Margin ($PM$)} & Mass Defect & Binding Ratio ($\chi\Delta / D\sigma$) & \textbf{Deuteron} & Energy released to purchase stability \\
\bottomrule
\end{tabularx}
\end{table*}



\section{The Formal Mapping Function: From Lattice to Observable} \label{sec:mapping}

To ensure the $E_8$-Persistence Theory is a computable framework rather than a purely interpretive one, we  define the formal mapping from the abstract lattice geometry to the world of physical observables. This function acts as the definitive recipe for calculating the fundamental constants from the derived invariants.

\paragraph{Input:} The set of five geometric invariants, $\mathbb{S} = \{D, \Delta, \nu, \sigma, \chi\}$, which are the necessary outputs of the stable $E_8 \to D_4 \oplus D_4$ projection derived in the preceding sections.

\paragraph{Output:} The fundamental physical constants are computed as rational functions of the elements in $\mathbb{S}$. The primary derivations are summarized below, with references to their detailed proofs.

\begin{table}[h!]
\centering
\caption{The Geometric Mapping of Fundamental Constants}
\label{tab:mapping_function}
\begin{tabular}{@{}llc@{}}
\toprule
\textbf{Observable} & \textbf{Geometric Formula (Schematic)} & \textbf{Section} \\
\midrule
Geometric Impedance & $\alpha^{-1} = f(\pi, \Delta, \chi, D, \sigma, \nu)$ & \S\ref{sec:GeometricImpedance} \\
Strong Coupling & $\alpha_s = \AlphaSEq$ & \S\ref{sec:Saturation_Limit} \\
Weak Mixing Angle & $\sin^2\theta_W = \WeakAngleEq$ & \S\ref{sec:Partition_Ratio} \\
Higgs VEV Scale & $\HiggsVEVEq$ & \S\ref{sec:Vacuum_Regulator} \\
\bottomrule
\end{tabular}
\end{table}

\paragraph{Principle of Sufficiency:} Each physical observable is the manifestation of a specific geometric constraint of the lattice, its impedance, saturation limit, partition ratio, or structural floor. The five invariants $\mathbb{S}$ are hereby posited to be \textit{necessary and sufficient} to derive the complete set of dimensionless constants governing the Standard Model and cosmology. No additional free parameters are required or permitted within this framework.


\section{Conclusion: The Invariant Substrate}

In this work, we have established that the fundamental constants of nature are not arbitrary tuning parameters, but the necessary boundary conditions of a discrete $E_8$ gauge theory projected onto a 4D manifold. These constants emerge inevitably from five geometric integers—no fitting, no tuning, no free parameters.

We have demonstrated a \textbf{Derivation Hierarchy} all fundamental limits derive from the lattice invariants:
\begin{equation}
\mathbb{S} = \{ D=4, \Delta=43, \sigma=5, \nu=16, \chi=2 \}
\end{equation}

The derivation proceeds through four integrated systems, transforming the raw lattice geometry into the observable physical universe:

\begin{itemize}
    \item \textbf{System I (The Invariant Substrate):} The five geometric integers define the computational hardware and causal limits of the vacuum.
    \item \textbf{System II (The Entropic Dynamics):} The Standard Model Lagrangian emerges as the unique solution minimizing Entropic Action ($S_\Phi$).
    \item \textbf{System III (The Geometric Impedance):} The master equation $\alpha^{-1} = f(\mathbb{S})$ encodes the baseline resistance of the vacuum to information propagation.
    \item \textbf{System IV (The Surface Regulator):} The Higgs field functions as a nested subsystem that impedance-matches the Fundamental Resonance to the weak interaction aperture, establishing the stable Electroweak Scale ($v$).
    \item \textbf{System V (The Bulk Regulator):} Gravity functions as a nested subsystem that attenuates bulk signals across the lattice depth, establishing the structural Geometry Scale ($M_P$) and prohibiting intermediate forces.
    \item \textbf{System VI (The Effective Field Architecture):} The complete set of boundary conditions, partition ratios and mass scales, forms a globally impedance-matched regulatory system.
\end{itemize}

Each system builds necessarily on the previous, creating a derivation cascade with zero free parameters. From these five integers, we have derived:
\begin{itemize}
    \item The complete gauge structure: $SU(3) \times SU(2) \times U(1)$
    \item All primary coupling constants: $\alpha^{-1}, \alpha_s, \theta_W, \theta_C, J$
    \item All fundamental mass scales: $M_P, v, m_H, m_e$
    \item The generation count: $n_{gen} = 3$ (structurally enforced)
\end{itemize}

By replacing free parameters with geometric necessities, we move from a descriptive model of physics to a predictive one.

\subsection{The Descent of Scales: A Universal Architecture}

The organization of this work reflects the physical hierarchy of the universe. We observe a \textbf{Derivation Cascade} where each layer acts as a regulator for the next, stepping down the infinite capacity of the lattice to the finite resolution of matter:

\begin{enumerate}
    \item \textbf{The Substrate (System I):} The raw lattice geometry ($E_8 \to D_4 \oplus D_4$) defines the fundamental resonance $\Delta = 43$ and establishes the information-processing limits $\nu = 16$.
    
    \item \textbf{The Geometric Impedance (System III):} The master equation $\alpha^{-1} \approx 137$ emerges from the interplay of all five invariants, setting the characteristic resistance of the vacuum.
    
    \item \textbf{The Bulk Regulator (nested within System IV):} This system attenuates bulk signals by $\alpha^{\Delta/2}$, establishing the \textbf{Unity Threshold} at the Planck Scale ($M_P \approx 10^{19}$ GeV).
    
    \item \textbf{The Surface Regulator (nested within System IV):} The Higgs field impedance-matches to the weak aperture ($Z_H \approx 6$), establishing the \textbf{Electroweak Scale} ($v \approx 246$ GeV).
    
    \item \textbf{The Resolution Floor (System II):} The Persistence Margin ($PM \approx 10^{-6}$) defines the noise floor, establishing the \textbf{Minimum Resolvable Mass} at the Electron Scale ($m_e \approx 0.5$ MeV).
\end{enumerate}

This hierarchical structure is not arbitrary, each scale emerges necessarily from the impedance matching requirements of the layer above. Standard physics views these scales ($M_P, v, m_e$) as independent random inputs. Informational Energetics reveals them as resonant harmonics of a single, unified substrate. The fundamental constants are the \textbf{impedance ratios} of this hierarchical regulatory architecture, relating each energy scale to the next via geometric factors like $\alpha^{\Delta/2}$ and $Z_H \approx 6$.

This work serves as a proof-of-concept for Informational Energetics. The fact that standard system constraints (Bandwidth, Protocol, Overhead) map 1:1 to the constants of nature suggests that the laws of physics are not fundamental axioms but emergent properties of information processing limits. The Standard Model is the thermodynamic ground state of a finite-capacity lattice, the unique persistent solution to the constraints of Informational Energetics.

\begin{table*}[t]
\centering
\caption{\textbf{The Universal Architecture of Persistence (The Rosetta Stone).} 
This table demonstrates the isomorphism between the principles of Informational Energetics (Rows) and the physical layers of the E8-Persistence Theory (Columns). Each physical constant is identified not as an arbitrary input, but as a specific structural component: Capacity, Identity, Protocol, or Governor, required to minimize Entropic Action at that specific scale.}
\label{tab:rosetta_stone}
\resizebox{\textwidth}{!}{%
\begin{tabular}{@{}lccccccc@{}}
\toprule
\textbf{IE Pillar} & \textbf{Sys I: Substrate} & \textbf{Sys II: Impedance} & \textbf{Sys III: Dynamics} & \textbf{Sys IV: Architecture} & \textbf{Sys V: Higgs} & \textbf{Sys VI: Gravity} & \textbf{Sys VII: Matter} \\ 
\textit{(Function)} & \textit{(Invariants)} & \textit{(Geometric Cost)} & \textit{(Lagrangian)} & \textit{(Effective Limits)} & \textit{(Surface Regulator)} & \textit{(Bulk Regulator)} & \textit{(Resonant Spectrum)} \\ \midrule
\textbf{Capacity ($\Delta E$)} & Resonance & Circumference & Mass Term & Total Bandwidth & VEV & Planck Mass & Knot Impedance \\
\textit{The Vessel} & $\Delta = 43$ & $\pi \Delta$ & $\bar{\psi} m \psi$ & $\nu = 16$ & $v \approx 246$ GeV & $M_P \approx 10^{19}$ GeV & $m_{fermion}$ \\ \addlinespace
\textbf{Identity ($\Delta I$)} & Symmetry Rank & Topology Cost & Dirac Operator & Gauge Group & Hypercharge & Tensor Mode & Quantum Numbers \\
\textit{The Model} & $\sigma = 5$ & $+ \chi$ & $i \gamma^\mu D_\mu$ & $SU(3)\times SU(2)\times U(1)$ & $Y = 1$ & Spin-2 ($h_{\mu\nu}$) & Spin, Charge, Color \\ \addlinespace
\textbf{Protocol ($MI$)} & Chiral Rank & Alignment & Gauge Kinetic & Strong Coupling & Self-Coupling & Grav. Coupling & Interaction Vertex \\
\textit{Coordination} & $\nu = 16$ & $\frac{-1}{D\Delta - \sigma}$ & $-\frac{1}{4}F_{\mu\nu}F^{\mu\nu}$ & $\alpha_s = \frac{\nu + 1/D}{\alpha^{-1}}$ & $\lambda \approx 0.13$ & $\alpha_G \approx 10^{-45}$ & Gauge Charges ($g$) \\ \addlinespace
\textbf{Governor ($G$)} & Boundary & Stability Pot. & Higgs Potential & Weak Partition & Quartic Term & Diffeomorphism & Exclusion Principle \\
\textit{Stability} & $\chi = 2$ & $- \chi / \Delta$ & $V(\phi)$ & $\sin^2 \theta_W = \frac{\Delta}{N_{sys}}$ & $\lambda |\phi|^4$ & $\nabla_\mu T^{\mu\nu} = 0$ & Pauli Blocking \\ \addlinespace
\textbf{Overhead ($T$)} & Causality & Entropic Cost & Metric & Time Asymmetry & Instability & Stiffness & Oscillation \\
\textit{Temporal Cost} & Sig $(-1)$ & $T_{geo} \approx 10^{-5}$ & $g_{\mu\nu}$ & $J \approx 3 \times 10^{-5}$ & $\mu^2 = \lambda v^2$ & $ds^2 \ge 0$ & Mixing Matrices \\ \addlinespace
\textbf{Margin ($PM$)} & Manifold & Resolution & Vacuum Energy & Flavor Aperture & Yukawa Floor & Planck Length & Stability Gap \\
\textit{Existence Floor} & $D = 4$ & $PM_{geo} \approx 10^{-6}$ & $\Lambda \approx 0$ & $\theta_C \approx \pi/(\nu-\chi)$ & $y_e \approx 3 \times 10^{-6}$ & $\ell_P$ & Particle Lifetime \\ \bottomrule
\end{tabular}%
}
\end{table*}


\begin{acknowledgments}
The author is an independent researcher and received no external funding for this work. 

I would like to thank my friends and family for their patience and support throughout decades of discussions as I tried to understand everything I encountered.
\end{acknowledgments}

\paragraph{Code Availability:} The computational scripts reproducing all numerical 
results in this paper are available at \url{https://github.com/meta-r0ze/E8-Persistence-Theory-1}.